
\subsection{Двумерная задача просачивания с источником на границе}
Рассмотрим двумерную область пористой среды, имеющую форму
прямоугольника. В начальный момент времени водой занято лишь
10\% области, а для нефти и газа задано периодическое распределение. Ось $x$ направлена слева-направо,
ось $y$ -- снизу-вверх. На левой границе в верхней половине области расположен источник воды.
На правой границе задано постоянное значение давления. Через верхнюю, нижнюю границы, вторую половину левой границы
может происходить просачивание. Учитывается влияние силы тяжести. Месторасположение источника нагревается. 
Пусть число расчетных узлов -- $N_x\cdot N_y$.\\
Начальные условия:
\begin{equation}
  \begin{aligned}
    &S_w=0.1,\\
    &S_n(x, y)=0.4 + 0.1 \cdot sin^2(x \cdot N_x + y \cdot N_y),\\
    &S_g(x, y)=0.4 + 0.1 \cdot cos^2(x \cdot N_x + y \cdot N_y),\\
    &P_w=P_\text{атм},\\
    &T=285K.
   \end{aligned}
\end{equation}
Граничные условия:
\begin{equation}
  \begin{aligned}
    &\left.T\right|_{x=0,\ 0.5 < y < 1}=320K,\quad \Biggl.\dfrac{\partial{T}}{\partial{x}}\Biggr|_{x=0,\ 0 < y < 0.5}=0,\quad \Biggl.\dfrac{\partial{T}}{\partial{x}}\Biggr|_{x=1}=0,\\
    &\Biggl.\dfrac{\partial{T}}{\partial{y}}\Biggr|_{y=0}=0,\quad \Biggl.\dfrac{\partial{T}}{\partial{y}}\Biggr|_{y=1}=0;\\
    &\left.P_w\right|_{x=0,\ 0.5 < y < 1}=1.1\cdot P_{\text{атм}},\quad \Biggl.\dfrac{\partial{P_w}}{\partial{x}}\Biggr|_{x=0,\ 0 < y < 0.5}=0,\quad \left.{P_w}\right|_{x=1}=P_{\text{атм}},\\
    &\Biggl.\dfrac{\partial{P_w}}{\partial{y}}\Biggr|_{y=0}=0,\quad \Biggl.\dfrac{\partial{P_w}}{\partial{y}}\Biggr|_{y=1}=0;\\
    &\left.S_w\right|_{x=0,\ 0.5 < y < 1}=0.6,\quad \Biggl.\dfrac{\partial{S_w}}{\partial{x}}\Biggr|_{x=0,\ 0 < y < 0.5}=0,\quad \Biggl.\dfrac{\partial{S_w}}{\partial{x}}\Biggr|_{x=1}=0,\\
    &\Biggl.\dfrac{\partial{S_w}}{\partial{y}}\Biggr|_{y=0}=0,\quad \Biggl.\dfrac{\partial{S_w}}{\partial{y}}\Biggr|_{y=1}=0;\\
    &\left.S_n\right|_{x=0,\ 0.5 < y < 1}=0.15,\quad \Biggl.\dfrac{\partial{S_n}}{\partial{x}}\Biggr|_{x=0,\ 0 < y < 0.5}=0,\quad \Biggl.\dfrac{\partial{S_n}}{\partial{x}}\Biggr|_{x=1}=0,\\
    &\Biggl.\dfrac{\partial{S_n}}{\partial{y}}\Biggr|_{y=0}=0,\quad \Biggl.\dfrac{\partial{S_n}}{\partial{y}}\Biggr|_{y=1}=0.
  \end{aligned}
\end{equation}

Результаты расчетов изображены на Рис.~\ref{t3_pic_start} - Рис.~\ref{t3_pic_end}, приведены распределения давления, температуры
и~насыщенностей трех фаз в три различные момента времени. Можно видеть, что фронты источника распространяются по
области и далее за ее границы, вода постепенно вытесняет нефть и газ.

\begin{figure}
\begin{center}
\begin{minipage}[h]{0.48\textwidth}
\begin{tikzpicture}
  \begin{axis}[view={0}{90}, enlargelimits=false, xlabel={$x$, м}, ylabel={$y$, м}, colorbar, width=0.8\textwidth, height=0.8\textwidth, point meta min=100000, point meta max=110000]
    \pgfplotstableread[skip first n=3]{data/test3/F-000000000100.tec}{\mytable}
    \addplot3 [surf, shader=interp] table [x=0, y=1, z=5] {\mytable};
  \end{axis}
\end{tikzpicture}
\caption{Давление $P_w$ в момент времени $t=100$с}
\label{t3_pic_start}
\end{minipage}
\hfill
\begin{minipage}[h]{0.48\textwidth}
\begin{tikzpicture}
  \begin{axis}[view={0}{90}, enlargelimits=false, xlabel={$x$, м}, ylabel={$y$, м}, colorbar, width=0.8\textwidth, height=0.8\textwidth, point meta min=285, point meta max=320]
    \pgfplotstableread[skip first n=3]{data/test3/F-000000000100.tec}{\mytable}
    \addplot3 [surf, shader=interp] table [x=0, y=1, z=6] {\mytable};
  \end{axis}
\end{tikzpicture}
\caption{Температура в момент времени $t=100$с}
\end{minipage}
\vfill
\begin{minipage}[h]{0.48\textwidth}
\begin{tikzpicture}
  \begin{axis}[view={0}{90}, enlargelimits=false, xlabel={$x$, м}, ylabel={$y$, м}, colorbar, width=0.8\textwidth, height=0.8\textwidth, point meta min=100000, point meta max=110000]
    \pgfplotstableread[skip first n=3]{data/test3/F-000000001000.tec}{\mytable}
    \addplot3 [surf, shader=interp] table [x=0, y=1, z=5] {\mytable};
  \end{axis}
\end{tikzpicture}
\caption{Давление $P_w$ в момент времени $t=1000$с}
\end{minipage}
\hfill
\begin{minipage}[h]{0.48\textwidth}
\begin{tikzpicture}
  \begin{axis}[view={0}{90}, enlargelimits=false, xlabel={$x$, м}, ylabel={$y$, м}, colorbar, width=0.8\textwidth, height=0.8\textwidth, point meta min=285, point meta max=320]
    \pgfplotstableread[skip first n=3]{data/test3/F-000000001000.tec}{\mytable}
    \addplot3 [surf, shader=interp] table [x=0, y=1, z=6] {\mytable};
  \end{axis}
\end{tikzpicture}
\caption{Температура в момент времени $t=1000$с}
\end{minipage}
\vfill
\begin{minipage}[h]{0.48\textwidth}
\begin{tikzpicture}
  \begin{axis}[view={0}{90}, enlargelimits=false, xlabel={$x$, м}, ylabel={$y$, м}, colorbar, width=0.8\textwidth, height=0.8\textwidth, point meta min=100000, point meta max=110000]
    \pgfplotstableread[skip first n=3]{data/test3/F-000000007000.tec}{\mytable}
    \addplot3 [surf, shader=interp] table [x=0, y=1, z=5] {\mytable};
  \end{axis}
\end{tikzpicture}
\caption{Давление $P_w$ в момент времени $t=7000$с}
\end{minipage}
\hfill
\begin{minipage}[h]{0.48\textwidth}
\begin{tikzpicture}
  \begin{axis}[view={0}{90}, enlargelimits=false, xlabel={$x$, м}, ylabel={$y$, м}, colorbar, width=0.8\textwidth, height=0.8\textwidth, point meta min=285, point meta max=320]
    \pgfplotstableread[skip first n=3]{data/test3/F-000000007000.tec}{\mytable}
    \addplot3 [surf, shader=interp] table [x=0, y=1, z=6] {\mytable};
  \end{axis}
\end{tikzpicture}
\caption{Температура в момент времени $t=7000$с}
\end{minipage}
\end{center}
\end{figure}
\begin{figure}
\begin{center}
\begin{minipage}[h]{0.48\textwidth}
\begin{tikzpicture}
  \begin{axis}[view={0}{90}, enlargelimits=false, xlabel={$x$, м}, ylabel={$y$, м}, colorbar, width=0.8\textwidth, height=0.8\textwidth, point meta min=0.1, point meta max=0.6]
    \pgfplotstableread[skip first n=3]{data/test3/F-000000000100.tec}{\mytable}
    \addplot3 [surf, shader=interp] table [x=0, y=1, z=2] {\mytable};
  \end{axis}
\end{tikzpicture}
\caption{Насыщенность $S_w$ в момент времени $t=100$с}
\end{minipage}
\hfill
\begin{minipage}[h]{0.48\textwidth}
\begin{tikzpicture}
  \begin{axis}[view={0}{90}, enlargelimits=false, xlabel={$x$, м}, ylabel={$y$, м}, colorbar, width=0.8\textwidth, height=0.8\textwidth, point meta min=0.1, point meta max=0.48]
    \pgfplotstableread[skip first n=3]{data/test3/F-000000000100.tec}{\mytable}
    \addplot3 [surf, shader=interp] table [x=0, y=1, z=3] {\mytable};
  \end{axis}
\end{tikzpicture}
\caption{Насыщенность $S_n$ в момент времени $t=100$с}
\end{minipage}
\vfill
\begin{minipage}[h]{0.48\textwidth}
\begin{tikzpicture}
  \begin{axis}[view={0}{90}, enlargelimits=false, xlabel={$x$, м}, ylabel={$y$, м}, colorbar, width=0.8\textwidth, height=0.8\textwidth, point meta min=0.1, point meta max=0.6]
    \pgfplotstableread[skip first n=3]{data/test3/F-000000001000.tec}{\mytable}
    \addplot3 [surf, shader=interp] table [x=0, y=1, z=2] {\mytable};
  \end{axis}
\end{tikzpicture}
\caption{Насыщенность $S_w$ в момент времени $t=1000$с}
\end{minipage}
\hfill
\hfill
\begin{minipage}[h]{0.48\textwidth}
\begin{tikzpicture}
  \begin{axis}[view={0}{90}, enlargelimits=false, xlabel={$x$, м}, ylabel={$y$, м}, colorbar, width=0.8\textwidth, height=0.8\textwidth, point meta min=0.1, point meta max=0.48]
    \pgfplotstableread[skip first n=3]{data/test3/F-000000001000.tec}{\mytable}
    \addplot3 [surf, shader=interp] table [x=0, y=1, z=3] {\mytable};
  \end{axis}
\end{tikzpicture}
\caption{Насыщенность $S_n$ в момент времени $t=1000$с}
\end{minipage}
\vfill
\begin{minipage}[h]{0.48\textwidth}
\begin{tikzpicture}
  \begin{axis}[view={0}{90}, enlargelimits=false, xlabel={$x$, м}, ylabel={$y$, м}, colorbar, width=0.8\textwidth, height=0.8\textwidth, point meta min=0.1, point meta max=0.6]
    \pgfplotstableread[skip first n=3]{data/test3/F-000000007000.tec}{\mytable}
    \addplot3 [surf, shader=interp] table [x=0, y=1, z=2] {\mytable};
  \end{axis}
\end{tikzpicture}
\caption{Насыщенность $S_w$ вТемпература момент времени $t=7000$с}
\end{minipage}
\hfill
\begin{minipage}[h]{0.48\textwidth}
\begin{tikzpicture}
  \begin{axis}[view={0}{90}, enlargelimits=false, xlabel={$x$, м}, ylabel={$y$, м}, colorbar, width=0.8\textwidth, height=0.8\textwidth, point meta min=0.1, point meta max=0.48]
    \pgfplotstableread[skip first n=3]{data/test3/F-000000007000.tec}{\mytable}
    \addplot3 [surf, shader=interp] table [x=0, y=1, z=3] {\mytable};
  \end{axis}
\end{tikzpicture}
\caption{Насыщенность $S_n$ в момент времени $t=7000$с}
\end{minipage}
\end{center}
\end{figure}
\begin{figure}
\begin{center}
\begin{minipage}[h]{0.48\textwidth}
\begin{tikzpicture}
  \begin{axis}[view={0}{90}, enlargelimits=false, xlabel={$x$, м}, ylabel={$y$, м}, colorbar, width=0.8\textwidth, height=0.8\textwidth, point meta min=0.1, point meta max=0.48]
    \pgfplotstableread[skip first n=3]{data/test3/F-000000000100.tec}{\mytable}
    \addplot3 [surf, shader=interp] table [x=0, y=1, z=4] {\mytable};
  \end{axis}
\end{tikzpicture}
\caption{Насыщенность $S_g$ в момент времени $t=100$с}
\end{minipage}
\vfill
\begin{minipage}[h]{0.48\textwidth}
\begin{tikzpicture}
  \begin{axis}[view={0}{90}, enlargelimits=false, xlabel={$x$, м}, ylabel={$y$, м}, colorbar, width=0.8\textwidth, height=0.8\textwidth, point meta min=0.1, point meta max=0.48]
    \pgfplotstableread[skip first n=3]{data/test3/F-000000001000.tec}{\mytable}
    \addplot3 [surf, shader=interp] table [x=0, y=1, z=4] {\mytable};
  \end{axis}
\end{tikzpicture}
\caption{Насыщенность $S_g$ в момент времени $t=1000$с}
\end{minipage}
\vfill
\begin{minipage}[h]{0.48\textwidth}
\begin{tikzpicture}
  \begin{axis}[view={0}{90}, enlargelimits=false, xlabel={$x$, м}, ylabel={$y$, м}, colorbar, width=0.8\textwidth, height=0.8\textwidth, point meta min=0.1, point meta max=0.48]
    \pgfplotstableread[skip first n=3]{data/test3/F-000000007000.tec}{\mytable}
    \addplot3 [surf, shader=interp] table [x=0, y=1, z=4] {\mytable};
  \end{axis}
\end{tikzpicture}
\caption{Насыщенность $S_g$ в момент времени $t=7000$с}
\label{t3_pic_end}
\end{minipage}
\end{center}
\end{figure}

