\documentclass[a4paper, draft]{article}
\usepackage[14pt]{extsizes}
\usepackage{cmap} % поиск русских слов в полученном pdf-файле
\usepackage[T2A]{fontenc} % Поддержка русских букв
\usepackage[utf8]{inputenc}
\usepackage{ucs}
\usepackage[english,russian]{babel}
\usepackage{makeidx} % пакет для создания алфавитного указателя
\usepackage{amsthm,amsfonts,amsmath,amssymb,amscd} % эти пакеты необходимы для набора формул
\usepackage{graphicx}
\usepackage{indentfirst}% Красная строка в первом абзаце
\usepackage{tabularx}
\usepackage{newclude}
\usepackage[labelsep=period]{caption} % пакет для управления заголовка рисунков и таблиц
\usepackage{subfig} % пакет для расположения нескольких под рисунков в одном окружении figure
\usepackage{longtable}
\usepackage{cite}
\usepackage{pgfplots}
\usepackage{pgfplotstable}
\usepackage{auto-pst-pdf}
\usepackage{wrapfig}
\usepackage{minted}
\usepackage[overload]{empheq}
\usepackage[nottoc]{tocbibind}
\usepackage{fancyhdr} % пакет для установки колонтитулов
\usepackage{geometry} % Меняем поля страницы
\usepackage{caption}
\usepackage[]{hyperref}
\geometry{left=2cm}% левое поле
\geometry{right=2cm}% правое поле
\geometry{top=3cm}% верхнее поле
\geometry{bottom=2cm}% нижнее поле
\sloppy % Избавляемся от переполнений
\clubpenalty=10000 % Запрещаем разрыв страницы после первой строки абзаца
\widowpenalty=10000 % Запрещаем разрыв страницы после последней строки абзаца
\captionsetup{figurewithin=none}
\linespread{1.0} % интервал
\graphicspath{{fig/}}
\makeindex % сделаем именной указатель, но печатаем его отдельной командой(если печатаем)
\frenchspacing %пробел после запятой не увеличивается - так принято в России
\righthyphenmin=2 % можно оставлять два символа на строке при переносе
\tolerance=400
\pagenumbering{arabic}

\DeclareCaptionLabelFormat{labelright}{\hfill {#1}{ }#2}
\captionsetup[table]{position=top, name={Таблица}, labelfont=it, labelsep = newline, labelformat=labelright, aboveskip=0px, belowskip=0px, justification=centering}
\captionsetup[figure]{name={Рис.}, labelfont=it, aboveskip=6px, belowskip=15px}


%%==========================================================================================
\begin{document}

\newcommand{\includepic}[2][]{\includegraphics[#1]{#2.png}}
\newcommand{\figref}[1]{рис.~\ref{#1}}

\renewcommand{\contentsname}{\hfill Оглавление \hfill}
\renewcommand{\refname}{Библиографический список}

\pagestyle{empty}
\begin{titlepage}
\begin{center}

{\bf Ордена Ленина \\
ИНСТИТУТ ПРИКЛАДНОЙ МАТЕМАТИКИ \\
имени М.В.~Келдыша \\
Российской академии наук \\
\par}

\vspace{50mm}

{\bf \large А.А.~Люпа, Е.Б.~Савенков\par}

\vspace{10mm}

{\bf \Large Модель двухфазной фильтрации с~релаксацией потока
и анализ эффективности применения явных схем
\par}

\end{center}

\vspace{\fill}

\begin{center}
{\bf Москва --- 2016}
\end{center}

\clearpage
\end{titlepage}
\newpage

{\noindent \textit{ \textbf {Люпа~А.А., Савенков~Е.Б.}}}


{\bf Модель двухфазной фильтрации \\ с~релаксацией потока \\
и анализ эффективности применения явных схем} \\


В работе рассматривается модификация классической модели
изотермической двухфазной фильтрации в пористых средах, полученная 
введением релаксации потока в уравнениях неразрывности.
Представлены результаты численного исследования предложенной модели,
а также применения явных численных методов со сниженным ограничением на
шаг по времени. \\


{\textit{ \textbf {Ключевые слова:}}} течение жидкости в пористой среде, явные конечно-разностные схемы \\ \\ \\


{\noindent \textit{ \textbf { Anastasiya Alexandrovna Lyupa, Evgeny Borisovich Savenkov}}}


{\bf Two-phase flow model with flow relaxation
and effectiveness analysis of the explicit schemes application} \\


The paper focuses on the modification of the classical model of isothermal two-phase 
flow in porous media by addition of flow relaxation to the continuity equation. 
The results of a numerical study of the proposed model are presented. 
Explicit numerical methods with a reduced restriction on time step are applied. \\


{\textit{ \textbf {Key words and phrases:}}} fluid flow in a porous medium, explicit finite difference schemes \\ \\ \\


Работа выполнена при поддержке Российского фонда фундаментальных исследований, проекты 16-29-15095-офи\_м, 15-01-03445-а, 15-01-03654-а. \\ \\ \\
\include*{intro}
\include*{model}
\include*{calc}
\include*{concl}
\bibliographystyle{utf8gost705u.bst}
\bibliography{kbib-utf8.bib}
%\nocite*{}
\tableofcontents

\end{document}
