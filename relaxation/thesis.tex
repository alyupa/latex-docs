\documentclass[a4paper,12pt]{article}
\usepackage{cmap} % поиск русских слов в полученном pdf-файле
\usepackage[T2A]{fontenc} % Поддержка русских букв
\usepackage[utf8]{inputenc}
\usepackage{ucs}
\usepackage[english,russian]{babel}
\usepackage{makeidx} % пакет для создания алфавитного указателя
\usepackage{amsthm,amsfonts,amsmath,amssymb,amscd} % эти пакеты необходимы для набора формул
\usepackage{graphicx}
\usepackage{indentfirst}% Красная строка в первом абзаце
\usepackage{tabularx}
\usepackage{newclude}
\usepackage[labelsep=period]{caption} % пакет для управления заголовка рисунков и таблиц
\usepackage{subfig} % пакет для расположения нескольких под рисунков в одном окружении figure
\usepackage{longtable}
\usepackage{cite}
\usepackage{pgfplots}
\usepackage{pgfplotstable}
\usepackage{wrapfig}
\usepackage{minted}
\usepackage[overload]{empheq}
\usepackage{geometry} % Меняем поля страницы
\geometry{left=3cm}% левое поле
\geometry{right=2cm}% правое поле
\geometry{top=2cm}% верхнее поле
\geometry{bottom=2cm}% нижнее поле
\sloppy % Избавляемся от переполнений
\clubpenalty=10000 % Запрещаем разрыв страницы после первой строки абзаца
\widowpenalty=10000 % Запрещаем разрыв страницы после последней строки абзаца
\newcommand{\includepic}[2][]{\includegraphics[#1]{#2.png}}
\renewcommand{\thefigure}{\thesection.\arabic{figure}}
%
\linespread{1.3} % интервал
\graphicspath{{fig/}}

\makeindex % сделаем именной указатель, но печатаем его отдельной командой(если печатаем)
\frenchspacing %пробел после запятой не увеличивается - так принято в России
\righthyphenmin=2 % можно оставлять два символа на строке при переносе

\definecolor{bg}{rgb}{0.97,0.97,0.97}
\usemintedstyle{colorful}
\renewcommand\listingscaption{Листинг}



%%==========================================================================================
\begin{document}
\pagenumbering{arabic}
\numberwithin{equation}{section}
\renewcommand{\theequation}{\thesection.\arabic{equation}}

\makeatletter
\renewcommand{\@biblabel}[1]{#1.\hfil}
\makeatother

\begin{titlepage}
\begin{center}

{\bf Ордена Ленина \\
ИНСТИТУТ ПРИКЛАДНОЙ МАТЕМАТИКИ \\
имени М.В.~Келдыша \\
Российской академии наук \\
\par}

\vspace{50mm}

{\bf \large А.А.~Люпа, Е.Б.~Савенков\par}

\vspace{10mm}

{\bf \Large Модель двухфазной фильтрации с~релаксацией потока
и анализ эффективности применения явных схем
\par}

\end{center}

\vspace{\fill}

\begin{center}
{\bf Москва --- 2016}
\end{center}

\clearpage
\end{titlepage}
\newpage

{\noindent \textit{ \textbf {Люпа~А.А., Савенков~Е.Б.}}}


{\bf Модель двухфазной фильтрации \\ с~релаксацией потока \\
и анализ эффективности применения явных схем} \\


В работе рассматривается модификация классической модели
изотермической двухфазной фильтрации в пористых средах, полученная 
введением релаксации потока в уравнениях неразрывности.
Представлены результаты численного исследования предложенной модели,
а также применения явных численных методов со сниженным ограничением на
шаг по времени. \\


{\textit{ \textbf {Ключевые слова:}}} течение жидкости в пористой среде, явные конечно-разностные схемы \\ \\ \\


{\noindent \textit{ \textbf { Anastasiya Alexandrovna Lyupa, Evgeny Borisovich Savenkov}}}


{\bf Two-phase flow model with flow relaxation
and effectiveness analysis of the explicit schemes application} \\


The paper focuses on the modification of the classical model of isothermal two-phase 
flow in porous media by addition of flow relaxation to the continuity equation. 
The results of a numerical study of the proposed model are presented. 
Explicit numerical methods with a reduced restriction on time step are applied. \\


{\textit{ \textbf {Key words and phrases:}}} fluid flow in a porous medium, explicit finite difference schemes \\ \\ \\


Работа выполнена при поддержке Российского фонда фундаментальных исследований, проекты 16-29-15095-офи\_м, 15-01-03445-а, 15-01-03654-а. \\ \\ \\

\tableofcontents
\newpage

\section{Введение}
Теория фильтрации, изучающая законы движения жидкостей, газов и их смесей в
пористой среде, имеет обширное практическое применение. На протяжении многих лет
фильтрационные расчеты занимают очень важное место при разработке технологий
добычи нефти и газа, при проектировании, постройке и эксплуатации
гидротехнических  и мелиоративных сооружений, в горном деле, в решении
экологических проблем. В настоящее время, с появлением мощных расчётных систем,
обладающих возможностями обрабатывать огромные объемы информации, задачи фильтрации
становятся все более и более актуальными.

Настоящая работа посвящена проблемам, связанным с течениями трехфазной жидкости 
сквозь пористую среду. Процессы напрямую зависят от свойств почв, их неоднородности. 
Три рассматриваемые фазы: вода, газ и NAPL(от английского Non-Aqueous Phase 
Liquids). К NAPL, например, относятся минеральное топливо, растворители, очищающие средства. 
В зависимости от того, как соотносится плотность вещества с плотностью воды, 
NAPL разделяются на легкие и плотные (первые называются Light NAPL, или LNAPL, их плотность меньше плотности
воды; вторые - Dense NAPL, или DNAPL, соответственно их плотность больше
плотности воды). Бензин, например, относится к первому типу, а тетрахлороэтилен
- ко второму. Эти жидкости не смешиваются с водой и с газом, поэтому при моделировании их течения
в почве говорят  о многофазной фильтрации.

Основной целью работы является получение распределения насыщенностей и давлений
трех фаз в пласте среды в зависимости от времени, начальных и граничных
условий, свойств и неоднородностей почв. 

Численные решения таких задач, также как и других задач газо- и гидродинамики,
требуют больших вычислительных затрат. В то же время существует необходимость в
наиболее быстром получении результатов. Использование
высокопроизводительных параллельных компьютеров с распределенной памятью
позволяет проводить вычисления в течение разумного времени. Поэтому созданные
вычислительные алгоритмы были адаптированы к расчетам на многопроцессорных
вычислительных комплексах. Для вычислений используются стандартная
библиотека MPI и технология Nvidia CUDA(для задействования в расчетах 
высокопроизводительных графических плат). При расчетах применяются явные разностные 
схемы, допускающие эффективное распараллеливание. Вычисления проводятся
на гибридном вычислительном комплексе K-100, ИПМ им. М.В.Келдыша РАН.

\newcommand*{\pd}[3][]{\ensuremath{\dfrac{\partial^{#1} #2}{\partial #3^{#1}}}}
\newcommand*{\dd}[3][]{\ensuremath{\dfrac{\mathrm d^{#1} #2}{\mathrm d #3^{#1}}}}
\newcommand{\grad}{\mathop{\mathrm{grad}}\nolimits}
\newcommand{\diver}{\mathop{\mathrm{div}}\nolimits}

\section{Классическая модель}
\label{classic_model}
Система уравнений для описания двухфазной неизотермической фильтрации в пористой среде
может быть представлена в виде~\cite{Aziz-Settari},~\cite{Basniev}:
\begin{subequations} \label{eq:classic_system}
  \begin{align}[left = \empheqlbrace\,]
    &\pd {(\phi \rho_i S_i)}{t} + \diver{(\rho_i \overrightarrow{u_i})} = q_i, \label{eq:classic_system1} \\
    &\overrightarrow{u_i} = -\frac{K k_i}{\mu_i}(\grad {P_i} - {\rho}_i g \grad {z}), \label{eq:classic_system2} \\
    &P_n = P_w + P_{c}(\overline{S}_w), \label{eq:classic_system3} \\
    &S_w + S_n = 1, \label{eq:classic_system4} \\
    &k_i = k_i(\overline{S}_w), \label{eq:classic_system5} \\
    &\rho_i = \rho_i(P_i), \label{eq:classic_system6} \\
    &i = w,n \nonumber .
  \end{align}
\end{subequations}

Здесь 
$P_i$ -- давление фазы,
$P_c$ -- капиллярное давление на границе раздела фаз,
$S_i$ -- насыщенность фазы,
${\rho}_i$ -- плотность фазы,
$q_i$ -- источниковые члены,
$K$ -- абсолютная проницаемость среды,
$\phi$ -- пористость среды,
$\overrightarrow{u_i}$ -- скорость фильтрации фазы,
$\mu_i$ -- вязкость фазы,
$k_i$ -- относительная фазовая проницаемость,
$g$ -- ускорение свободного падения,
\eqref{eq:classic_system1} -- уравнения неразрывности,
\eqref{eq:classic_system2} -- закон Дарси,
\eqref{eq:classic_system3} -- связь давлений фаз,
\eqref{eq:classic_system4} -- связь насыщенностей фаз,
\eqref{eq:classic_system5} -- уравнения относительных фазовых проницаемостей,
\eqref{eq:classic_system6} -- уравнения состояния.

\newpage
%%%% отсюда начинаем нумерацию страниц
\setcounter{page}{4}
\pagestyle{fancy} % смена стиля оформления страниц
\fancyhf{} % очистка текущих значений
\fancyhead[C]{\thepage} % установка верхнего колонтитула
\renewcommand{\headrulewidth}{0pt} % убрать разделительную линию

Для удобства дальнейших выкладок введем следующие обозначения.
Пусть $q_t = \dfrac{q_n}{\rho_n} + \dfrac{q_w}{\rho_w}$,
$c_n = \dfrac{1}{\rho_n} \dd{\rho_n}{P_n}$,
$c_w = \dfrac{1}{\rho_w} \dd{\rho_w}{P_w}$,
$P_{avg} = \dfrac{P_n + P_w}{2}$,
$P_n = P_{avg} + \dfrac{P_c}{2}$,
$P_w = P_{avg} - \dfrac{P_c}{2}$,
$\lambda_n = -\dfrac{K k_n}{\mu_n}$,
$\lambda_w = -\dfrac{K k_w}{\mu_w}$,
$\overrightarrow{u_t} = \overrightarrow{u_n} + \overrightarrow{u_w}$,
$f_w = \dfrac{\lambda_w}{\lambda_n + \lambda_w}$,
$h_w = - \dfrac{\lambda_n\lambda_w}{\lambda_n + \lambda_w} \dd{P_c}{S_w}$.

Для дальнейшего использования метода IMPES (Implicit Pressure Explicit Saturation) -- метода неявного по давлению,
явного по насыщенности, проводится преобразование уравнений модели~\eqref{eq:classic_system}~\cite{Peaceman}.
Разложим член $\pd {(\phi \rho_i S_i)}{t}$ на сумму слагаемых:
\begin{equation}
 \pd {(\phi \rho_i S_i)}{t} = \rho_i S_i \dd{\phi}{P_{avg}} \cdot \pd{P_{avg}}{t}
 + \phi S_i \dd{\rho_i}{P_i} \cdot \pd{P_i}{t} + \phi \rho_i \pd{S_i}{t}.
\end{equation}

Разделим уравнения неразрывности~\eqref{eq:classic_system1} на $\rho_i$ 
и сложим их. Также используем принятые выше обозначения.
Получим следующее уравнение для определения среднего давления:
\begin{equation} \label{eq:classic_pressure}
 \begin{gathered}
  \frac{1}{\rho_n} \diver{ (\rho_n\lambda_n\grad{ P_{avg}})} + \frac{1}{\rho_w} \diver{ (\rho_w\lambda_w\grad{ P_{avg}})} +\\
  + \frac{1}{2} \left[ \frac{1}{\rho_n} \diver{ (\rho_n\lambda_n\grad{ P_c})} + \frac{1}{\rho_w} \diver{ (\rho_w\lambda_w\grad{ P_c})}\right] + q_t =\\
  = \left[ \dd{\phi}{P_{avg}} + \phi (S_n c_n + S_w c_w) \right] \pd{P_{avg}}{t} +
  \frac{1}{2} \phi (S_n c_n - S_w c_w) \pd{P_{c}}{t} +\\
  + g \left[ \frac{1}{\rho_n} \diver{ (\rho_n^2\lambda_n\grad{ z})} + \frac{1}{\rho_w} \diver{ (\rho_w^2\lambda_w\grad{ z})} \right].
 \end{gathered}
\end{equation}

Получим уравнение для нахождения насыщенности $S_w$, исходя из предположения,
что давление $P_{avg}$ к этому моменту уже известно.
Из~\eqref{eq:classic_system3} следует, что $\grad{ P_c} = \grad{ P_n} - \grad{ P_w}$ или
$\lambda_n\lambda_w\grad{ P_c} = \lambda_n\lambda_w\grad{ P_n} - \lambda_n\lambda_w\grad{ P_w}$.
Используем~\eqref{eq:classic_system1}, ~\eqref{eq:classic_system2} и принятые обозначения.
Тогда
\begin{equation} \label{eq:classic_saturation}
 \begin{gathered}
  \diver{(\rho_w h_w \grad{ S_w})} - \diver{(\rho_w f_w[\overrightarrow{u_t} + \lambda_n(\rho_w - \rho_n)g \grad{ z}])} + \\
  + q_w = \pd{(\phi \rho_w S_w)}{t}.
 \end{gathered}
\end{equation} 
 
\section{Модифицированная модель}
\label{mod_model}

Запишем уравнение~\eqref{eq:classic_system1} кратко в виде
\begin{equation} \label{eq:short_system}
   \pd{(\phi \rho_i S_i)}{t} + \diver{ \overrightarrow{Q_i}} = q_i, i = w,n,
\end{equation}
где $\overrightarrow{Q_i}$ -- поток фазы.
Пусть $\overrightarrow{Q_i^D} = \rho_i \overrightarrow{u_i}$ -- поток Дарси.
В модели~\eqref{eq:classic_system} $\overrightarrow{Q_i} = \overrightarrow{Q_i^D}$.
Мы же введем релаксацию потока, положим 

\begin{equation}
 \overrightarrow{Q_i} = \overrightarrow{Q_i^D} - \tau \pd{\overrightarrow{Q_i}}{t},
\end{equation}
где $\tau$ -- параметр релаксации, довольно малая величина, характеризующая время установления равновесия в системе. Откуда следует, что
\begin{equation} \label{eq:div_relax}
 \diver{\overrightarrow{Q_i}} = \diver{\overrightarrow{Q_i^D}} - \tau \diver{\pd{\overrightarrow{Q_i}}{t}}.
\end{equation}

Продифференцируем по времени уравнения~\eqref{eq:short_system} и домножим на $\tau$:
\begin{equation}
  \tau \pd[2]{(\phi \rho_i S_i)}{t} + \tau \pd{(\diver{ \overrightarrow{Q_i}})}{t} = \tau \pd{q_i}{t}.
\end{equation}

С учетом~\eqref{eq:div_relax} от~\eqref{eq:short_system} перейдем к конечному виду модифицированного уравнения неразрывности
путем введения релаксации потока:

\begin{equation} \label{eq:mod_system}
  \tau \pd[2]{(\phi \rho_i S_i)}{t} + \pd{(\phi \rho_i S_i)}{t} + \diver{ \overrightarrow{Q_i^D}} = q_i + \tau \pd{q_i}{t}.
\end{equation}\\

Сделаем некоторые вспомогательные выкладки:
\begin{equation} \label{eq:calcs1}
 \begin{aligned}
  \pd[2]{(\phi \rho_i S_i)}{t} &= \rho_i S_i \pd[2]{\phi}{t} + \phi S_i \pd[2]{\rho_i}{t} + \phi \rho_i \pd[2]{S_i}{t} + \\
  &+ 2S_i \pd{\phi}{t} \cdot \pd{\rho_i}{t} + 2\rho_i \pd{\phi}{t} \cdot \pd{S_i}{t} + 2\phi \pd{\rho_i}{t} \cdot \pd{S_i}{t} = \\
  &= \rho_i S_i \left(  \dd[2]{\phi}{P_{avg}} \cdot {\left( \pd{P_{avg}}{t}\right) }^2 + \dd{\phi}{P_{avg}} \cdot \pd[2]{P_{avg}}{t} \right) +\\
  &+ \phi S_i \left( \dd[2]{\rho_i}{P_i} \cdot {\left( \pd{P_i}{t}\right) }^2 + \dd{\rho_i}{P_i} \cdot \pd[2]{P_i}{t} \right) + \phi \rho_i \pd[2]{S_i}{t} +\\
  &+ 2S_i \dd{\phi}{P_{avg}} \cdot \pd{P_{avg}}{t} \cdot \dd{\rho_i}{P_i} \cdot \pd{P_i}{t} + 2\rho_i \dd{\phi}{P_{avg}} \cdot \pd{P_{avg}}{t} \cdot \pd{S_i}{t} +\\
  &+ 2\phi \dd{\rho_i}{P_i} \cdot \pd{P_i}{t} \cdot \pd{S_i}{t},
 \end{aligned}
\end{equation}
\begin{equation} \label{eq:calcs2}
 \begin{aligned}
  \pd[2]{\phi}{t} &= \dd[2]{\phi}{P_{avg}} \cdot {\left( \pd{P_{avg}}{t}\right) }^2 + \dd{\phi}{P_{avg}} \cdot \pd[2]{P_{avg}}{t}, \\
  \pd[2]{\rho_i}{t} &= \dd[2]{\rho_i}{P_i} \cdot {\left( \pd{P_i}{t}\right) }^2 + \dd{\rho_i}{P_i} \cdot \pd[2]{P_i}{t}, \\
  \pd{q_i}{t} &= \dd{q_i}{P_{avg}} \cdot \pd{P_{avg}}{t},
 \end{aligned}
\end{equation}

Проведя преобразования, аналогичные выполненным в разделе~\ref{classic_model},
получим с учетом результатов~\eqref{eq:calcs1},~\eqref{eq:calcs2} следующие уравнения для определения
среднего давления и насыщенности водной фазы:
\begin{equation} \label{eq:full_pressure}
 \begin{gathered}
   \frac{1}{\rho_n} \diver{ (\rho_n\lambda_n\grad{ P_{avg}})} + \frac{1}{\rho_w} \diver{ (\rho_w\lambda_w\grad{ P_{avg}})}  +\\
   + \frac{1}{2} \left[ \frac{1}{\rho_n} \diver{ (\rho_n\lambda_n\grad{ P_c})} + \frac{1}{\rho_w} \diver{ (\rho_w\lambda_w\grad{ P_c})} \right]  +\\
   + q_t = \left[ \dd{\phi}{P_{avg}} + \phi (S_n c_n + S_w c_w) \right] \left( \pd{P_{avg}}{t} + \tau \pd[2]{P_{avg}}{t} \right) +\\
   + \frac{1}{2} \phi (S_n c_n - S_w c_w) \left( \pd{P_{c}}{t} + \tau \pd[2]{P_c}{t} \right) +\\
   + g \left[ \frac{1}{\rho_n} \diver{ (\rho_n^2\lambda_n\grad{ z})} + \frac{1}{\rho_w} \diver{ (\rho_w^2\lambda_w\grad{ z})} \right] +\\
   + \tau \left[ \dd[2]{\phi}{P_{avg}} + \phi \left( \frac{1}{\rho_n} S_n \dd[2]{\rho_n}{P_n} +
   + \frac{1}{\rho_w} S_w \dd[2]{\rho_w}{P_w} \right) \right. + \\
   \left. + 2(S_n c_n + S_w c_w) \dd{\phi}{P_{avg}} \right] {\left( \pd{P_{avg}}{t}\right) }^2 +\\
   + \tau \left[ \phi \left( \frac{1}{\rho_n} S_n \dd[2]{\rho_n}{P_n} - \frac{1}{\rho_w} S_w \dd[2]{\rho_w}{P_w} \right) \right. +\\
   \left. + (S_n c_n - S_w c_w) \dd{\phi}{P_{avg}} \right] \pd{P_{avg}}{t} \cdot \pd{P_c}{t} +\\
   + \frac{1}{4} \tau \phi \left( \frac{1}{\rho_n} S_n \dd[2]{\rho_n}{P_n} + \frac{1}{\rho_w} S_w \dd[2]{\rho_w}{P_w} \right) {\left( \pd{P_c}{t}\right) }^2 -\\
  - \tau \left(\frac{1}{\rho_n}\dd{q_n}{P_{avg}} + \frac{1}{\rho_w}\dd{q_w}{P_{avg}}\right) \pd{P_{avg}}{t} +\\
   + 2 \tau \phi \left(c_n \pd{S_n}{t} + c_w \pd{S_w}{t} \right) \pd{P_{avg}}{t}+ \\
   + \tau \phi \left(c_n \pd{S_n}{t} - c_w \pd{S_w}{t} \right) \pd{P_c}{t},
 \end{gathered}
\end{equation}
\begin{equation} \label{eq:full_saturation}
 \begin{gathered}
  \diver{(\rho_w h_w \grad{ S_w})} - \diver{(\rho_w f_w[\overrightarrow{u_t} + \lambda_n(\rho_w - \rho_n)g \grad{ z}])} + q_w =\\
  = \rho_i S_i \dd{\phi}{P_{avg}} \cdot \pd{P_{avg}}{t} + \phi S_i \dd{\rho_i}{P_i} \cdot \pd{P_i}{t} + \phi \rho_i \pd{S_i}{t} +\\
  + \tau \pd[2]{(\phi \rho_w S_w)}{t} - \tau \dd{q_w}{P_{avg}} \cdot \pd{P_{avg}}{t}.
 \end{gathered}
\end{equation} 

Поскольку целью данной работы является проверка возможности в принципе снизить ограничение на шаг по времени при
использовании явного численного метода для решения уравнения для среднего давления
с помощью введения релаксации потока в уравнения неразрывности~\eqref{eq:classic_system1}, то на данном этапе
будем исходить из дополнительных предположений, позволяющих значительно упростить уравнения для среднего давления
и насыщенности, не изменяя существенно их характера в целом. В частности, пористость среды положим постояной,
уравнения состояния - линейными, капиллярными эффектами пренебрежем:
\begin{equation}
\phi = const,\; \dd[2]{\rho_i}{P_i} = 0,\;P_w = P_n = P_{avg}.
\end{equation}
При перечисленных допущениях, действуя как в разделе~\ref{classic_model}, получим уравнение для определения давления:
\begin{equation} \label{eq:mod_pressure}
 \begin{gathered}
  \frac{1}{\rho_n} \diver{ (\rho_n\lambda_n\grad{ P_{avg}})} + \frac{1}{\rho_w} \diver{ (\rho_w\lambda_w\grad{ P_{avg}})} + q_t =\\
  = \phi (S_n c_n + S_w c_w) \left( \pd{P_{avg}}{t} + \tau \pd[2]{P_{avg}}{t} \right) +\\
  + g \left[ \frac{1}{\rho_n} \diver{ (\rho_n^2\lambda_n\grad{ z})} + \frac{1}{\rho_w} \diver{ (\rho_w^2\lambda_w\grad{ z})} \right]  -\\
  -  \tau \left(\frac{1}{\rho_n}\dd{q_n}{P_{avg}} + \frac{1}{\rho_w}\dd{q_w}{P_{avg}}\right) \pd{P_{avg}}{t} +\\
  + 2 \tau \phi \left(c_n \pd{S_n}{t} + c_w\pd{S_w}{t}\right) \pd{P_{avg}}{t}.
 \end{gathered}
\end{equation}

Несмотря на то, что полностью избавиться от насыщенности в уравнении не получилось, в силу малости слагаемого
\begin{equation*}
 2 \tau \phi (c_n \pd{S_n}{t} + c_w\pd{S_w}{t}) \pd{P_{avg}}{t}
\end{equation*}
при вычислениях будем считать допустимым задание $\pd{S_i}{t}$ явно --- будем использовать данные
значения с предыдущего временного слоя.

А для определения насыщенности достаточно использовать исходное уравнение неразрывности~\eqref{eq:mod_system},
раскрыв слагаемые, в которых присутствуют частные производные по времени от произведения переменных:
\begin{equation} \label{eq:mod_saturation}
 \begin{gathered}
  \diver{ (\rho_w\lambda_w \grad{ P_{avg}})} + q_w = g \cdot \diver{ (\rho_w^2\lambda_w \grad{ z})} + \phi \rho_w S_w c_w  \pd{P_{avg}}{t} +\\
  + \phi \rho_w \pd{S_w}{t} + \tau \phi \rho_w \pd[2]{S_w}{t} + \tau \phi \rho_w S_w c_w \pd[2]{P_{avg}}{t} +\\
  + 2 \tau \phi \rho_w c_w \pd{P_{avg}}{t} \cdot \pd{S_w}{t} - \tau \dd{q_w}{P_{avg}} \cdot \pd{P_{avg}}{t}.
 \end{gathered}
\end{equation}

Относительные фазовые проницаемости определяются в~работе в~соответствии с~
приближением Стоуна~\cite{Aziz-Settari}:
\begin{equation} \label{eq:k_w}
  k_{w}(\overline{S}_w)=
  \begin{cases}
  &\overline{S}_w^\frac{1}{2} \left( 1-\left( 1-\overline{S}_w^\frac{n}{n-1} \right) ^\frac{n-1}{n} \right) ^2,
  \, 0<\overline{S}_w<1 \\
  &1, \,\overline{S}_w\ge 1\\
  &0, \,\overline{S}_w\le 0
\end{cases},
\end{equation}
\begin{equation} \label{eq:k_n}
  k_{n}(\overline{S}_w)=
  \begin{cases}
  &(1-\overline{S}_w)^\frac{1}{2} \left(1-\overline{S}_w^\frac{n}{n-1} \right) ^\frac{2(n-1)}{n},
  \, 0<\overline{S}_w<1\\
  &1, \,\overline{S}_w\le 0\\
  &0, \, \overline{S}_w\ge 1
  \end{cases},
\end{equation}
где $\overline{S}_w$ -- эффективная насыщенность водной фазы:
$\overline{S}_w={\dfrac{S_w-S_{wr}}{1-S_{wr}-S_{nr}}}$, где $S_{wr}$,
$S_{nr}$ -- остаточные насыщенности фаз. Для наглядности функции~\eqref{eq:k_w} и~\eqref{eq:k_n}
представлены на графиках \figref{k_w} и \figref{k_n} при $S_{wr}=S_{nr}=0.05$.
\begin{figure}[H]
\begin{center}
\begin{minipage}[h!]{0.49\textwidth}
\begin{tikzpicture}
  \begin{axis}[axis lines=left, xmax=1, xmin=0, ymax=1, ymin=0, enlargelimits=true, grid=major, width=0.95\textwidth, xlabel={$S_w$}, ylabel={$k_w$}]
    \pgfplotstableread[skip first n=1]{data/permeability.out}{\mytable}
    \addplot [blue, ultra thick] table [x=0, y=1] {\mytable};
  \end{axis}
\end{tikzpicture}
\caption{$k_w(S_w)$}
\label{k_w}
\end{minipage}
\hfill
\begin{minipage}[h!]{0.49\textwidth}
\begin{tikzpicture}
 \begin{axis}[axis lines=left, xmax=1, xmin=0, ymax=1, ymin=0, enlargelimits=true, grid=major, width=0.95\textwidth, xlabel={$S_w$}, ylabel={$k_n$}]
    \pgfplotstableread[skip first n=1]{data/permeability.out}{\mytable}
    \addplot [black, ultra thick] table [x=0, y=2] {\mytable};
  \end{axis}
\end{tikzpicture}
\caption{$k_n(S_w)$}
\label{k_n}
\end{minipage}
\end{center}
\end{figure}
\section{Вычислительный эксперимент}

%%%% Заменить обозначения: лев на left, прав на right здесь и в векторе - правой части СЛАУ (3.4).

\subsection{Постановка тестовой задачи}
\label{test_task}
Рассмотрим одномерную задачу двухфазной двухкомпонентной фильтрации ~(присутствуют только вода и легкая нефть в жидком состоянии)
в однородной изотропной пористой среде с учетом упрощений раздела ~\ref{model_simplification}.
Исследуемая область имеет форму параллелепипеда со сторонами 5м, 1м, 1м, но в силу симметрии задача сводится к одномерной.
Зафиксируем шаг по пространству $h=0.05\text{м}$.
Зададим следующие начальные условия: $S_w = 0.35$, $P_{avg}$ линейно убывает от $3.3\cdot 10^7$Па к $3.0\cdot 10^7$Па слева направо.
Граничные условия: $S_w|_{\text{лев}} = 0.7$, $S_w|_{\text{прав}} = 0.35$,
$P_{avg}|_{\text{лев}} = 3.3\cdot 10^7$Па, $P_{avg}|_{\text{прав}} = 3.0\cdot 10^7$Па.

\begin{table}[H]
\caption{Параметры фаз}
\label{tabular:liquids}
\begin{center}
\begin{tabular}{|c|c|c|}
\hline
Физ. величина & вода, \textit {w} & нефть, \textit {n} \\
\hline
Плотность,  $ {\text{кг}} / {\text{м}^3} $ & 1000 & 850 \\
\hline
Динамическая вязкость, $ \text{Па} \cdot \text{с} $ & $10^{-3}$ & $10^{-2}$ \\
\hline
Сжимаемость, $ \text{Па}^{-1}$ & $4.4 \cdot 10^{-10}$ & $10^{-9}$ \\
\hline
Остаточная насыщенность & 0.05 & 0.05 \\
\hline
\end{tabular}
\end{center}
\end{table}

\begin{table}[H]
\caption{Параметры среды}
\label{tabular:medium}
\begin{center}
\begin{tabular}{|c|c|}
\hline
Пористость & 0.4\\
\hline
Абсолютная проницаемость, $ \text{м}^{2}$ & $6.64 \cdot 10^{-11}$ \\
\hline
\end{tabular}
\end{center}
\end{table}


\subsection{Эталонное решение}
\label{reference sample}
В силу сильной нелинейности системы уравнений ~\eqref{eq:classic_system} найти точное общее решение не представляется возможным.
В этом случае в качестве эталонного решения возьмем решение, полученное с помощью часто используемого и проверенного на практике IMPES-метода с расчетным шагом по
времени $\Delta t = 10^{-4}\text{с}$. Величина шага подобрана с помощью сравнения результатов расчетов при различных шагах.
Дальнейшее уменьшение шага не привело к существенному изменению результатов. Детальные данные представлены в таблице ~\ref{time_step_impes_cor}.

С учетом упрощений тестовой задачи из уравнения ~\eqref{eq:classic_pressure} получим неявное разностное уравнение для давления,
а из ~\eqref{eq:classic_saturation} -- явное для насыщенности.

\begin{equation} \label{eq:scheme_p}
 \begin{aligned} 
  & \dfrac{1}{\Delta x^2} \bigg(\dfrac{1}{\rho_{n_i}} \big( \rho_{n_{i+0.5}} \lambda_{n_{i+0.5}} (P_{i+1}^{m+1} - P_i^{m+1})
  - \rho_{n_{i-0.5}} \lambda_{n_{i-0.5}} (P_i^{m+1} - P_{i-1}^{m+1})\big) + \\
  &+ \dfrac{1}{\rho_{w_i}} \big( \rho_{w_{i+0.5}} \lambda_{w_{i+0.5}} (P_{i+1}^{m+1} - P_i^{m+1})
  - \rho_{w_{i-0.5}} \lambda_{w_{i-0.5}} (P_i^{m+1} - P_{i-1}^{m+1})\big)\bigg) + \\
  & + q_{t_i} = \phi(S_{n_i}c_{n_i}+S_{w_i}c_{w_i}) \dfrac{P_i^{m+1} - P_i^m}{\Delta t}
  + \dfrac{g}{\Delta x} \bigg(\dfrac{1}{\rho_{n_i}} \big( \rho_{n_{i+0.5}}^2 \lambda_{n_{i+0.5}} 
  - \rho_{n_{i-0.5}}^2 \lambda_{n_{i-0.5}} \big) + \\
  &+ \dfrac{1}{\rho_{w_i}} \big( \rho_{w_{i+0.5}}^2 \lambda_{w_{i+0.5}} 
  - \rho_{w_{i-0.5}}^2 \lambda_{w_{i-0.5}} \big)\bigg)
 \end{aligned}
\end{equation}

\begin{equation} \label{eq:scheme_s}
 \begin{aligned} 
  & S_{w_i}^{m+1} = S_{w_i}^{m} - S_{w_i}c_{w_i} (P_i^{m+1} - P_i^m) + \\
  & + \dfrac{\Delta t}{\phi\rho_{w_i}}\Bigg( \dfrac{1}{\Delta x^2} \bigg(\rho_{w_{i+0.5}} \lambda_{w_{i+0.5}} (P_{i+1}^{m+1} - P_i^{m+1})
  - \rho_{w_{i-0.5}} \lambda_{w_{i-0.5}} (P_i^{m+1} - P_{i-1}^{m+1}) \bigg) +\\
  & + q_{w_i} - \dfrac{g}{\Delta x} \bigg(\rho_{w_{i+0.5}}^2 \lambda_{w_{i+0.5}} - \rho_{w_{i-0.5}}^2 \lambda_{w_{i-0.5}} \bigg) \Bigg)
 \end{aligned}
\end{equation}

Запишем уравнение ~\eqref{eq:scheme_p} в матричном виде:
\begin{equation} \label{eq:scheme_p_matrix}
A\overrightarrow{P^{m+1}} = \vec{b}
\end{equation}

\begin{equation} \label{eq:p_matrix}
\begin{aligned}
A = \begin{pmatrix}
1 & 0 & 0 & 0 & \cdots & 0 & 0 \\
a_{10} & a_{11} & a_{12} & 0 & \cdots & 0 & 0 \\
\vdots & \vdots & \ddots & \vdots & \vdots & \vdots & \vdots \\
0 & \cdots & a_{ii-1} & a_{ii} & a_{ii+1} & \cdots & 0 \\
\vdots & \vdots & \vdots & \vdots & \ddots & \vdots & \vdots\\
0 & 0 & \cdots & 0 & a_{N-1 N-2} & a_{N-1 N-1} & a_{N-1 N} \\
0 & 0 & \cdots & 0 & 0 & 0 & 1
\end{pmatrix}, \;
\vec{b} = \begin{pmatrix}
P_{\text{лев}} \\
b_1 \\
\vdots \\
b_i \\
\vdots \\
b_{N-1} \\
P_{\text{прав}}
\end{pmatrix}
\end{aligned}
\end{equation}

\begin{equation} \label{eq:p_matrix_i}
\begin{aligned}
a_{ii-1} = & \dfrac{1}{\Delta x^2} \big(\dfrac{\rho_{n_{i-0.5}} \lambda_{n_{i-0.5}}}{\rho_{n_i}} + \dfrac{\rho_{w_{i-0.5}} \lambda_{w_{i-0.5}}}{\rho_{w_i}} \big) \\
a_{ii} = & -\dfrac{1}{\Delta x^2} \big(\dfrac{\rho_{n_{i-0.5}} \lambda_{n_{i-0.5}} + \rho_{n_{i+0.5}} \lambda_{n_{i+0.5}}}{\rho_{n_i}} 
+ \dfrac{\rho_{w_{i-0.5}} \lambda_{w_{i-0.5}} + \rho_{w_{i+0.5}} \lambda_{w_{i+0.5}}}{\rho_{w_i}} \big) - \\
& - \dfrac{\phi}{\Delta t} (S_{n_i}c_{n_i}+S_{w_i}c_{w_i}) \\
a_{ii+1} = & \dfrac{1}{\Delta x^2} \big(\dfrac{\rho_{n_{i+0.5}} \lambda_{n_{i+0.5}}}{\rho_{n_i}} + \dfrac{\rho_{w_{i+0.5}} \lambda_{w_{i+0.5}}}{\rho_{w_i}} \big) \\
b_i = & - \dfrac{\phi}{\Delta t} (S_{n_i}c_{n_i}+S_{w_i}c_{w_i})P_i^m - q_t + \dfrac{g}{\Delta x} \bigg(\dfrac{1}{\rho_{n_i}} \big( \rho_{n_{i+0.5}}^2 \lambda_{n_{i+0.5}} 
  - \rho_{n_{i-0.5}}^2 \lambda_{n_{i-0.5}} \big) + \\
  &+ \dfrac{1}{\rho_{w_i}} \big( \rho_{w_{i+0.5}}^2 \lambda_{w_{i+0.5}} 
  - \rho_{w_{i-0.5}}^2 \lambda_{w_{i-0.5}} \big)\bigg)
\end{aligned}
\end{equation}

В данном тестовом примере заданы граничные условия первого рода.
Матрица коэффициентов трехдиагональная.
Для решения полученной СЛАУ используется итерационный численный метод.


Зависимости насыщенностей и давления от времени представлены на графиках.
Видно постепенное вытеснение нефти водой вплоть до уровня источника.

\newcommand{\printplotoned}[1]
{
\begin{figure}[h!]
\begin{center}
\begin{minipage}[h!]{0.49\textwidth}
\begin{tikzpicture}
  \begin{axis}[legend entries={$S_w$,$S_n$}, axis lines=left, xmax=5, xmin=0, ymax=1, ymin=0, enlargelimits=true, grid=major, width=1\textwidth, xlabel={$x$, м}, ylabel={$S$}]
    \pgfplotstableread[skip first n=1]{data/impes_#1.0s.out}{\mytable}
    \addplot [blue, ultra thick] table [x=0, y=2] {\mytable};
    \addplot [black, ultra thick] table [x=0, y expr=1-\thisrowno{2}] {\mytable};
  \end{axis}
\end{tikzpicture}
\caption{Насыщенности, t=#1с}
\label{t_10_s}
\end{minipage}
\hfill
\begin{minipage}[h!]{0.49\textwidth}
\begin{tikzpicture}
 \begin{axis}[axis lines=left, xmax=5, xmin=0, ymax=330, ymin=300, enlargelimits=true, grid=major, width=1\textwidth, xlabel={$x$, м}, ylabel={$P$, атм}]
    \pgfplotstableread[skip first n=1]{data/impes_#1.0s.out}{\mytable}
    \addplot [green, ultra thick] table [x=0, y=1] {\mytable};
  \end{axis}
\end{tikzpicture}
\caption{Давление, t=#1с}
\label{t_10_p}
\end{minipage}
\end{center}
\end{figure}
}

\printplotoned{10}
\printplotoned{100}
\printplotoned{200}
\printplotoned{500}

В таблице представлены погрешности расчета IMPES-методом в моменты времени t=30c, t=60c и t=90c. 
при различных $\Delta t$ по сравнению с $\Delta t = 10^{-5}\text{с}$.
Относительная ошибка между величинами $A$ и $B$ вычислялась следующим образом: $\varepsilon=\sqrt{\sum\limits_{i=1}^N\left({\frac{A_i-B_i}{A_i}}\right)^2} / N$.
Погрешность менее $10^{-6}$ считаем несущественной. Обращаем внимание как на погрешность расчета давления, так и насыщенности водной фазы.
Исследуем данные значения в разные моменты времени для исключения быстрого набегания ошибки.
Из чего делаем вывод, что решение при $\Delta t = 10^{-4}\text{с}$
можно взять в качестве эталонного для продолжительных расчетов. При этом $\varepsilon_p$ меняется незначительно, $\varepsilon_s$ возрасла на $0.4 \cdot 10^{-7}$  за 60с.

\begin{table}[H]
\caption{Погрешность IMPES-метода в зависимости от шага по времени}
\label{tabular:impes}
\begin{center}
\begin{tabular}{|c|c|c|c|}
\hline
$\Delta t$,c & $\varepsilon_p, \varepsilon_s$ при t=30с & $\varepsilon_p,\, \varepsilon_s$ при t=60с & $\varepsilon_p,\, \varepsilon_s$ при t=90с \\
\hline
$10^{-3}$ & $1.4 \cdot 10^{-8}, \, 1.4 \cdot 10^{-6}$ & $1.1  \cdot 10^{-8}, \, 1.6 \cdot 10^{-6}$ & $0.9  \cdot 10^{-8}, \, 1.8 \cdot 10^{-6}$ \\
\hline
$5 \cdot 10^{-4}$ & $6.7 \cdot 10^{-9}, \, 6.8 \cdot 10^{-7}$ & $5.5  \cdot 10^{-9}, \, 7.8 \cdot 10^{-7}$ & $4.7  \cdot 10^{-9}, \, 8.9 \cdot 10^{-7}$ \\
\hline
$10^{-4}$ & $1.2 \cdot 10^{-9}, \, 1.3 \cdot 10^{-7}$ & $1.0  \cdot 10^{-9}, \, 1.4 \cdot 10^{-7}$ & $0.9  \cdot 10^{-9}, \, 1.6 \cdot 10^{-7}$ \\
\hline
$5 \cdot 10^{-5}$ & $5.6  \cdot 10^{-10}, \, 5.6 \cdot 10^{-8}$ & $4.5  \cdot 10^{-10}, \, 6.4 \cdot 10^{-8}$ & $3.9  \cdot 10^{-10}, \, 7.2 \cdot 10^{-8}$ \\
\hline
\end{tabular}
\end{center}
\end{table}


\subsection{Явная схема для модифицированной модели}
\label{mod_model_explicit}
Приведем разностную схему для модели ~\eqref{eq:mod_system}.
Уравнению для давления ~\eqref{eq:mod_pressure} соответствует следующее разностное уравнение:
\begin{equation} \label{eq:mod_scheme_p}
 \begin{aligned} 
  & \dfrac{1}{\Delta x^2} \bigg(\dfrac{1}{\rho_{n_i}} \big( \rho_{n_{i+0.5}} \lambda_{n_{i+0.5}} (P_{i+1}^m - P_i^m)
  - \rho_{n_{i-0.5}} \lambda_{n_{i-0.5}} (P_i^m - P_{i-1}^m)\big) + \\
  &+ \dfrac{1}{\rho_{w_i}} \big( \rho_{w_{i+0.5}} \lambda_{w_{i+0.5}} (P_{i+1}^m - P_i^m)
  - \rho_{w_{i-0.5}} \lambda_{w_{i-0.5}} (P_i^m - P_{i-1}^m)\big)\bigg) + q_{t_i} = \\
  &= \phi(S_{n_i}c_{n_i}+S_{w_i}c_{w_i}) \dfrac{P_i^{m+1} - P_i^m}{\Delta t}
  + \dfrac{g}{\Delta x} \bigg(\dfrac{1}{\rho_{n_i}} \big( \rho_{n_{i+0.5}}^2 \lambda_{n_{i+0.5}} 
  - \rho_{n_{i-0.5}}^2 \lambda_{n_{i-0.5}} \big) + \\
  &+ \dfrac{1}{\rho_{w_i}} \big( \rho_{w_{i+0.5}}^2 \lambda_{w_{i+0.5}} 
  - \rho_{w_{i-0.5}}^2 \lambda_{w_{i-0.5}} \big)\bigg)
  - \tau \dfrac{q_{t_i}^m-q_{t_i}^{m-1}}{\Delta t} + \\
  &+ \tau \phi (S_{n_i}c_{n_i}+S_{w_i}c_{w_i}) \dfrac{P_i^{m+1}-2P_i^m+P_i^{m-1}}{\Delta t^2} +\\
  &+ 2 \tau \phi (c_{w_i}-c_{n_i}) \dfrac{S_{w_i}^m-S_{w_i}^{m-1}}{\Delta t} \cdot \dfrac{P_i^{m+1}-P_i^m}{\Delta t}
 \end{aligned}
\end{equation}
А уравнению ~\eqref{eq:mod_saturation} -- разностное уравнение:
\begin{equation} \label{eq:mod_scheme_s}
 \begin{aligned} 
  & \dfrac{1}{\Delta x^2} \bigg(\rho_{w_{i+0.5}} \lambda_{w_{i+0.5}} (P_{i+1}^{m+1} - P_i^{m+1})
  - \rho_{w_{i-0.5}} \lambda_{w_{i-0.5}} (P_i^{m+1} - P_{i-1}^{m+1}) \bigg) + q_{w_i} = \\
  &= \phi\rho_{w_i}S_{w_i}c_{w_i} \dfrac{P_i^{m+1} - P_i^m}{\Delta t} + \phi\rho_{w_i} \dfrac{S_{w_i}^{m+1} - S_{w_i}^m}{\Delta t} +\\ 
  &+ \dfrac{g}{\Delta x} \bigg(\rho_{w_{i+0.5}}^2 \lambda_{w_{i+0.5}} - \rho_{w_{i-0.5}}^2 \lambda_{w_{i-0.5}} \bigg)
  - \tau \dfrac{q_{w_i}^m-q_{w_i}^{m-1}}{\Delta t} + \\
  &+ \tau \phi \rho_{w_i}S_{w_i}c_{w_i} \dfrac{P_i^{m+1}-2P_i^m+P_i^{m-1}}{\Delta t^2}
  + \tau \phi \rho_{w_i} \dfrac{S_{w_i}^{m+1}-2S_{w_i}^m+S_{w_i}^{m-1}}{\Delta t^2} \\
  &+ 2 \tau \phi \rho_{w_i} c_{w_i} \dfrac{S_{w_i}^{m+1}-S_{w_i}^{m}}{\Delta t} \cdot \dfrac{P_i^{m+1}-P_i^m}{\Delta t}
 \end{aligned}
\end{equation}
Где $i$ -- пространственный индекс, а $m$ -- временной. Опущенный временной индекс соответствует текущему шагу.
Фазовые проницаемости определяются <<вверх по потоку>> ~\cite{Kanevskaya}.
\begin{equation}
k_{w_{i+0.5}} =
 \begin{cases}
  &k_w(S_{w_i}),\, P_{i+1}-P_i-\rho_{w_i}g\,\Delta x \le 0
  \\
  &k_w(S_{w_{i+1}}),\, P_{i+1}-P_i-\rho_{w_i}g\,\Delta x > 0
 \end{cases}
\end{equation}
Аналогично вычисляется $k_{n_{i+0.5}}$.
Из схем получаем явные уравнения для давления $P_i^{m+1}$ и насыщенности $S_{w_i}^{m+1}$
в каждом расчетном узле. 


\begin{pspicture}(14,6)
 \psframe(1,1)(13,5)
 \psline[linecolor=black](1,3)(13,3)
 \psline[linecolor=black](4,1)(4,5)
 \psline[linecolor=black](7,1)(7,5)
 \psline[linecolor=black](10,1)(10,5)
 \pscircle[linecolor=black](1,1){0.2}
 \pscircle[linecolor=black](7,1){0.2}
 \pscircle[linecolor=black](13,1){0.2}
 \pscircle[linecolor=black](1,3){0.2}
 \pscircle[linecolor=black](7,3){0.2}
 \pscircle[linecolor=black](13,3){0.2}
 \pscircle[linecolor=black](1,5){0.2}
 \pscircle[linecolor=black](7,5){0.2}
 \pscircle[linecolor=black](13,5){0.2}
 \psline{->}(3.5,4)(4.5,4)
 \psline{->}(4.5,3.5)(3.5,3.5)
 \psline{->}(9.5,4)(10.5,4)
 \psline{->}(10.5,3.5)(9.5,3.5)
 \rput(1,0.5){$i-1$}
 \rput(7,0.5){$i$}
 \rput(13,0.5){$i+1$}
 \rput(4,0.5){$i-0.5$}
 \rput(10,0.5){$i+0.5$}
 \rput(0.25,1){$m-1$}
 \rput(0.25,3){$m$}
 \rput(0.25,5){$m+1$}
 \label{pic:scheme}
\end{pspicture}

Для вычисления потока берем значения величин в <<полуцелых>> узлах как среднее арифметическое
значений в узлах. Например, плотность на границе ячеек:
$\rho_{i+0.5}=\dfrac{\rho_i+\rho_{i+1}}{2},\, \rho_{i-0.5}=\dfrac{\rho_i+\rho_{i-1}}{2}$.
$k_i, \lambda_i$ вычисляются аналогично.

Отметим, что, если положить $\tau=0$, модель ~\eqref{mod_model_explicit} совпадает с моделью ~\eqref{eq:classic_system},
а представленная явная трехслойная разностная схема для модели ~\eqref{mod_model_explicit} совпадает с явной двухслойной
для расчетов по модели ~\eqref{eq:classic_system}.

\subsection{Результаты расчетов}
\label{calc_results}
Для проведения расчетов по моделям ~\eqref{eq:classic_system}, ~\eqref{mod_model_explicit} написана программа на языке Python с использованием научных библиотек NumPy, SciPy.
Параметр $\tau$, при котором расчеты были бы устойчивы, подбирался эмпирически в зависимости от размера шага по времени.
Проведено сравнение решения IMPES-методом модели ~\eqref{eq:classic_system} и решения трехслойной явной разностной схемой
модели ~\eqref{mod_model_explicit}.
Результаты представлены в таблице ~\ref{tabular:results}.

\begin{table}[H]
\caption{Погрешность явной схемы относительно эталонного решения при различных $\tau$ при при t=10с}
\label{tabular:results}
\begin{center}
\begin{tabular}{|c|c|c|c|}
\hline
$\Delta t$,c & $\tau$,с & $\varepsilon_p$ & $\varepsilon_s$  \\
\hline
$ 10^{-5}$ & $5 \cdot 10^{-5}$ & $ 2 \cdot 10^{-6}$ & $ 0.003 $ \\
\hline
$ 10^{-4}$ & $2 \cdot 10^{-4}$ & $ 5 \cdot 10^{-7}$ & $ 0.001 $ \\
\hline
$5 \cdot 10^{-4}$ & $0.01$ & $ 10^{-4}$ & $ 0.003 $ \\
\hline
$0.001$ & $0.025$ & $ 5 \cdot 10^{-4}$ & $ 0.03 $ \\
\hline
\end{tabular}
\end{center}
\end{table}

Например, при $\Delta x=0.05$м явная двухслойная схема становится неустойчивой при $\Delta t \ge 5 \cdot 10^{-5}$с, а явная
трехслойная при $\Delta t=0.001$с, $\tau=0.025$с позволяет проводить расчеты с приемлемой с точки зрения моделирования
погрешностью $\varepsilon_p=0.0005$, $\varepsilon_s=0.03$ на момент времени $t=10$с.
\section{Заключение}

В процессе работы над дипломом:

\begin{itemize}
	\item 	предложены 1D, 2D и 3D-модели течений в пористых средах,
	допускающие реализацию явными численными методами;

	\item поставлены и решены модельные задачи просачивания в однородной 
	пористой среде;

	\item разработан алгоритм расчета;
	
	\item алгоритм распараллелен;
	
	\item написан программный комплекс на языке программирования C/C++ с использованием 
	библиотек MPI, CUDA для решения задач трехфазной неизотермической фильтрации;
	
	\item в необходимом на данном этапе объеме освоена технология 
	программирования на графических платах Nvidia CUDA;

	\item освоены технологии визуализации данных расчетов;

	\item программный комплекс протестирован на нескольких тестовых задачах 
	фильтрации;
	
	\item сделаны три доклада на конференциях~\citemy{MIPT-54,MIPT-55,Pareng-2013}, одна из которых -- международная~\citemy{Pareng-2013};

	\item написаны в соавторстве две статьи~\citemy{CSE,Mathmod-2014};
\end{itemize}


\newpage
\addcontentsline{toc}{section}{\hspace{7mm}Список литературы}
\bibliographystyle{utf8gost705u.bst}
\bibliography{kbib-utf8.bib}
\nocite*{}

\end{document}
