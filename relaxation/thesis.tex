\documentclass[a4paper, draft]{article}
\usepackage[14pt]{extsizes}
\usepackage{cmap} % поиск русских слов в полученном pdf-файле
\usepackage[T2A]{fontenc} % Поддержка русских букв
\usepackage[utf8]{inputenc}
\usepackage{ucs}
\usepackage[english,russian]{babel}
\usepackage{makeidx} % пакет для создания алфавитного указателя
\usepackage{amsthm,amsfonts,amsmath,amssymb,amscd} % эти пакеты необходимы для набора формул
\usepackage{graphicx}
\usepackage{indentfirst}% Красная строка в первом абзаце
\usepackage{tabularx}
\usepackage{newclude}
\usepackage[labelsep=period]{caption} % пакет для управления заголовка рисунков и таблиц
\usepackage{subfig} % пакет для расположения нескольких под рисунков в одном окружении figure
\usepackage{longtable}
\usepackage{cite}
\usepackage{pgfplots}
\usepackage{pgfplotstable}
\usepackage{auto-pst-pdf}
\usepackage{wrapfig}
\usepackage{minted}
\usepackage[overload]{empheq}
\usepackage[nottoc]{tocbibind}
\usepackage{fancyhdr} % пакет для установки колонтитулов
\usepackage{geometry} % Меняем поля страницы
\usepackage{caption}
\usepackage[]{hyperref}
\geometry{left=2cm}% левое поле
\geometry{right=2cm}% правое поле
\geometry{top=3cm}% верхнее поле
\geometry{bottom=2cm}% нижнее поле
\sloppy % Избавляемся от переполнений
\clubpenalty=10000 % Запрещаем разрыв страницы после первой строки абзаца
\widowpenalty=10000 % Запрещаем разрыв страницы после последней строки абзаца
\captionsetup{figurewithin=none}
\linespread{1.0} % интервал
\graphicspath{{fig/}}
\makeindex % сделаем именной указатель, но печатаем его отдельной командой(если печатаем)
\frenchspacing %пробел после запятой не увеличивается - так принято в России
\righthyphenmin=2 % можно оставлять два символа на строке при переносе
\tolerance=400
\pagenumbering{arabic}

\DeclareCaptionLabelFormat{labelright}{\hfill {#1}{ }#2}
\captionsetup[table]{position=top, name={Таблица}, labelfont=it, labelsep = newline, labelformat=labelright, aboveskip=0px, belowskip=0px, justification=centering}
\captionsetup[figure]{name={Рис.}, labelfont=it, aboveskip=6px, belowskip=15px}


%%==========================================================================================
\begin{document}

\newcommand{\includepic}[2][]{\includegraphics[#1]{#2.png}}
\newcommand{\figref}[1]{рис.~\ref{#1}}

\renewcommand{\contentsname}{\hfill Оглавление \hfill}
\renewcommand{\refname}{Библиографический список}

\pagestyle{empty}
\begin{titlepage}
\begin{center}
\vspace{7mm}
{МОСКОВСКИЙ ФИЗИКО-ТЕХНИЧЕСКИЙ ИНСТИТУТ \\
(ГОСУДАРСТВЕННЫЙ УНИВЕРСИТЕТ)} \\
\vspace{45mm}
{Кафедра философии \par}
\vspace{20mm}
{РЕФЕРАТ ПО ИСТОРИИ НАУКИ \par}
\vspace{15mm}
{\bf \large ИСТОРИЯ РАЗВИТИЯ И ОСНОВНЫЕ ПОНЯТИЯ ПОДЗЕМНОЙ ГИДРОМЕХАНИКИ \par}

\end{center}

\vspace{35mm}
\hspace{15mm}
{Аспирант -- Люпа Анастасия Александровна\par}
\vspace{10mm}
\hspace{15mm}
{Научный руководитель аспиранта \underline{\hspace{2.5cm}} Четверушкин Б.Н.\par}
\hspace{100mm}
{\tiny подпись\par}
\vspace{8mm}
\hspace{15mm}
{Преподаватель кафедры философии --  Семенов Ю.И.\par}

\vspace{\fill}

\begin{center}
{г. Москва 2014}
\end{center}

\clearpage
\end{titlepage}
\newpage
\include*{intro}
\include*{model}
\include*{calc}
\include*{concl}
\bibliographystyle{utf8gost705u.bst}
\bibliography{kbib-utf8.bib}
%\nocite*{}
\tableofcontents

\end{document}
