\begin{titlepage}
\begin{center}

{РОССИЙСКАЯ АКАДЕМИЯ НАУК \\
ОРДЕНА ЛЕНИНА \\
ИНСТИТУТ ПРИКЛАДНОЙ МАТЕМАТИКИ \\
имени М. В. КЕЛДЫША \\
\par}

\vspace{80mm}

{А.А. Люпа, Е.Б. Савенков\par}

\vspace{10mm}

{\bf \large Модель двухфазной фильтрации с релаксацией потока
и анализ эффективности применения явных схем
\par}

\end{center}

\vspace{\fill}

\begin{center}
{Москва, 2016}
\end{center}

\clearpage
\end{titlepage}
\newpage


Люпа А.А., Савенков Е.Б. Модель двухфазной фильтрации с релаксацией потока
и анализ эффективности применения явных схем
\\

{\bf Аннотация.} В работе рассматривается модификация классической модели
изотермической двухфазной фильтрации в пористых средах, полученная 
введением релаксации потока в уравнениях неразрывности.
Представлены результаты численного исследования предложенной модели,
а также применения явных численных методов со сниженным ограничением на
шаг по времени.
\\

{\bf Ключевые слова:} течение жидкости в пористой среде, явные конечно-разностные схемы
\\

{\bf Abstract.} The paper focuses on the modification of the classical model of isothermal two-phase 
flow in porous media by addition of flow relaxation to the continuity equation. 
The results of a numerical study of the proposed model are presented. 
Explicit numerical methods with a reduced restriction on time step are applied.
\\

{\bf Key words and phrases:} fluid flow in a porous medium, explicit finite difference schemes
\\

Работа выполнена при поддержке Российского фонда фундаментальных исследований, проекты 16-29-15095-офи\_м, 15-01-03445-а, 15-01-03654-а.