\pagenumbering{Alph}
\begin{titlepage}
\begin{center}

{\bf Ордена Ленина \\
ИНСТИТУТ ПРИКЛАДНОЙ МАТЕМАТИКИ \\
имени М.В.~Келдыша \\
Российской академии наук \\
\par}

\vspace{50mm}

{\bf \large А.А.~Люпа, Е.Б.~Савенков\par}

\vspace{10mm}

{\bf \Large Модель двухфазной фильтрации с~релаксацией потока
и анализ эффективности применения явных схем
\par}

\end{center}

\vspace{\fill}

\begin{center}
{\bf Москва --- 2016}
\end{center}

\clearpage
\end{titlepage}
\pagenumbering{arabic}
\newpage

{\noindent \textit{ \textbf {Люпа~А.А., Савенков~Е.Б.}}} \medskip

{\bf Модель двухфазной фильтрации с~релаксацией потока
и анализ эффективности применения явных схем} \bigskip

В работе рассматривается модификация классической модели
изотермической двухфазной фильтрации в пористых средах, полученная 
введением релаксации потока в уравнениях неразрывности.
Представлены результаты численного исследования предложенной модели,
а также применения явных численных методов со сниженным ограничением на
шаг по времени. \bigskip

{\textit{ \textbf {Ключевые слова:}}} течение жидкости в пористой среде, явные конечно-разностные схемы 
\bigskip \bigskip \bigskip

{\noindent \textit{ \textbf { Anastasiya Alexandrovna Lyupa, Evgeny Borisovich Savenkov}}} \medskip

{\bf Two-phase flow model with flow relaxation
and effectiveness analysis of the explicit schemes application} \bigskip

The paper focuses on the modification of the classical model of isothermal two-phase 
flow in porous media by addition of flow relaxation to the continuity equation. 
The results of a numerical study of the proposed model are presented. 
Explicit numerical methods with a reduced restriction on time step are applied. \bigskip

{\textit{ \textbf {Key words and phrases:}}} fluid flow in a porous medium, explicit finite difference schemes 
\bigskip \bigskip \bigskip

Работа выполнена при поддержке Российского фонда фундаментальных исследований, проекты 16-29-15095-офи\_м, 15-01-03445-а, 15-01-03654-а.
\bigskip \bigskip \bigskip