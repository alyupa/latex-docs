\newcommand*{\pd}[3][]{\ensuremath{\dfrac{\partial^{#1} #2}{\partial #3^{#1}}}}
\newcommand*{\dd}[3][]{\ensuremath{\dfrac{\mathrm d^{#1} #2}{\mathrm d #3^{#1}}}}

\section{Модель}
\subsection{Классическая модель}
\label{classic_model}
Система уравнений для описания двухфазной неизотермической фильтрации в пористой среде
имеет вид:
\begin{subequations} \label{eq:classic_system}
  \begin{align}[left = \empheqlbrace\,]
    &\pd {(\phi \rho_i S_i)}{t} + div(\rho_i \overrightarrow{u_i}) = q_i, \label{eq:classic_system1} \\
    &\overrightarrow{u_i} = -\frac{K k_i}{\mu_i}(\nabla P_i - {\rho}_i g \nabla z), \label{eq:classic_system2} \\
    &P_n = P_w + P_{c}(\overline{S}_w), \label{eq:classic_system3} \\
    &S_w + S_n = 1, \label{eq:classic_system4} \\
    &k_i = k_i(\overline{S}_w), \label{eq:classic_system5} \\
    &\rho_i = \rho_i(P_i), \label{eq:classic_system6} \\
    &i = w,n \nonumber .
  \end{align}
\end{subequations}

Здесь 
$P_i$ -- давление фазы,
$P_c$ -- капиллярное давление на границе раздела фаз,
$S_i$ -- насыщенность фазы,
${\rho}_i$ -- плотность фазы,
$q_i$ -- источниковые члены,
$K$ -- абсолютная проницаемость среды,
$\phi$ -- пористость среды,
$\overrightarrow{u_i}$ -- фазовая скорость,
$\mu_i$ -- вязкость фазы,
$k_i$ -- относительная фазовая проницаемость,
$g$ -- ускорение свободного падения,
\eqref{eq:classic_system1} -- уравнения неразрывности,
\eqref{eq:classic_system2} -- закон Дарси,
\eqref{eq:classic_system3} -- связь давлений фаз,
\eqref{eq:classic_system4} -- связь насыщенностей фаз,
\eqref{eq:classic_system5} -- уравнения относительных фазовых проницаемостей,
\eqref{eq:classic_system6} -- уравнения состояния.

\subsection{Принятые обозначения}
\label{syms}
Пусть $q_t = \dfrac{q_n}{\rho_n} + \dfrac{q_w}{\rho_w}$,
$c_n = \dfrac{1}{\rho_n} \dd{\rho_n}{P_n}$,
$c_w = \dfrac{1}{\rho_w} \dd{\rho_w}{P_w}$,
$P_{avg} = \dfrac{P_n + P_w}{2}$,
$P_n = P_{avg} + \dfrac{P_c}{2}$,
$P_w = P_{avg} - \dfrac{P_c}{2}$,
$\lambda_n = -\dfrac{K k_n}{\mu_n}$,
$\lambda_w = -\dfrac{K k_w}{\mu_w}$,
$\overrightarrow{u_t} = \overrightarrow{u_n} + \overrightarrow{u_w}$,
$f_w = \dfrac{\lambda_w}{\lambda_n + \lambda_w}$,
$h_w = - \dfrac{\lambda_n\lambda_w}{\lambda_n + \lambda_w} \dd{P_c}{S_w}$.


\subsection{Разделение переменных}
\label{var_division}
Для дальнейшего использования метода IMPES (Implicit Pressure Explicit Saturation) -- метода неявного по давлению,
явного по насыщенности, проводится преобразование уравнений модели ~\eqref{eq:classic_system}.
Разложим член $\pd {(\phi \rho_i S_i)}{t}$ на сумму слагаемых:
\begin{equation}
 \pd {(\phi \rho_i S_i)}{t} = \rho_i S_i \dd{\phi}{P_{avg}} \cdot \pd{P_{avg}}{t}
 + \phi S_i \dd{\rho_i}{P_i} \cdot \pd{P_i}{t} + \phi \rho_i \pd{S_i}{t}.
\end{equation}
Разделим уравнения неразрывности ~\eqref{eq:classic_system1} на $\rho_i$ 
и сложим их. Также используем принятые в ~\ref{syms} обозначения.
Получим следующее уравнение для определения среднего давления:

\begin{equation} \label{eq:classic_pressure}
 \begin{aligned}
  &\left[ \frac{1}{\rho_n} div (\rho_n\lambda_n) + \frac{1}{\rho_w} div (\rho_w\lambda_w) \right] \nabla P_{avg} + 
  \frac{1}{2} \left[ \frac{1}{\rho_n} div (\rho_n\lambda_n) + \frac{1}{\rho_w} div (\rho_w\lambda_w) \right] \nabla P_c \\
  & + q_t = \left[ \dd{\phi}{P_{avg}} + \phi (S_n c_n + S_w c_w) \right] \pd{P_{avg}}{t} +
  \frac{1}{2} \phi (S_n c_n - S_w c_w) \pd{P_{c}}{t} \\
  & + g \left[ \frac{1}{\rho_n} div (\rho_n^2\lambda_n) + \frac{1}{\rho_w} div (\rho_w^2\lambda_w) \right] \nabla z
 \end{aligned}
\end{equation}

Обратим внимание, что оператор $div$ действует на всю правую часть слагаемого, к которому он применяется.

Получим уравнение для нахождения насыщенности $S_w$, исходя из предположения,
что давление $P_{avg}$ к этому моменту уже известно.
Из ~\eqref{eq:classic_system3} следует, что $\nabla P_c = \nabla P_n - \nabla P_w$ или
$\lambda_n\lambda_w\nabla P_c = \lambda_n\lambda_w\nabla P_n - \lambda_n\lambda_w\nabla P_w$.
Используем ~\eqref{eq:classic_system1}, ~\eqref{eq:classic_system2} и принятые в ~\ref{syms} обозначения.
Тогда
\begin{equation} \label{eq:classic_saturation}
  div(\rho_w h_w \nabla S_w) - div(\rho_w f_w)[\overrightarrow{u_t} + \lambda_n(\rho_w - \rho_n)g \nabla z] + q_w 
  = \pd{(\phi \rho_w S_w)}{t}
\end{equation} 
 
\subsection{Модифицированная модель}

Запишем модель ~\eqref{eq:classic_system} кратко в виде
\begin{equation} \label{eq:short_system}
   \pd{(\phi \rho_i S_i)}{t} + div \overrightarrow{Q_i} = q_i, i = w,n.
\end{equation}

Здесь $\overrightarrow{Q_i}$ -- поток фазы.
Пусть $\overrightarrow{Q_i^D} = \rho_i \overrightarrow{u_i}$ -- поток Дарси.
В модели ~\eqref{eq:classic_system} $\overrightarrow{Q_i} = \overrightarrow{Q_i^D}$.
Мы же введем релаксацию потока, положим 

\begin{equation}
 \overrightarrow{Q_i} = \overrightarrow{Q_i^D} - \tau \pd{\overrightarrow{Q_i}}{t},
\end{equation}
где $\tau$ -- параметр релаксации, довольно малая величина, характеризующая время установления равновесия в системе.

Откуда следует, что
\begin{equation} \label{eq:div_relax}
 div\overrightarrow{Q_i} = div\overrightarrow{Q_i^D} - \tau div\pd{\overrightarrow{Q_i}}{t}.
\end{equation}

Продифференцируем по времени исходные уравнения системы ~\eqref{eq:short_system} и домножим на $\tau$, получим что

\begin{equation}
  \tau \pd[2]{(\phi \rho_i S_i)}{t} + \tau \pd{(div \overrightarrow{Q_i})}{t} = \tau \pd{q_i}{t}.
\end{equation}

Откуда с учетом ~\eqref{eq:div_relax} от ~\eqref{eq:short_system} перейдем к конечному виду модифицированного уравнения неразрывности
путем введения релаксации потока:

\begin{equation} \label{eq:mod_system}
  \tau \pd[2]{(\phi \rho_i S_i)}{t} + \pd{(\phi \rho_i S_i)}{t} + div \overrightarrow{Q_i^D} = q_i + \tau \pd{q_i}{t}.
\end{equation}



\subsection{Разделение переменных в модифицированной модели}

Сделаем некоторые вспомогательные выкладки:
\begin{equation} \label{calcs1}
 \begin{aligned}
  \pd[2]{\phi}{t} &= \dd[2]{\phi}{P_{avg}} \cdot {\left( \pd{P_{avg}}{t}\right) }^2 + \dd{\phi}{P_{avg}} \cdot \pd[2]{P_{avg}}{t}, \\
  \pd[2]{\rho_i}{t} &= \dd[2]{\rho_i}{P_i} \cdot {\left( \pd{P_i}{t}\right) }^2 + \dd{\rho_i}{P_i} \cdot \pd[2]{P_i}{t}, \\
  \pd[2]{(\phi \rho_i S_i)}{t} &= \rho_i S_i \pd[2]{\phi}{t} + \phi S_i \pd[2]{\rho_i}{t} + \phi \rho_i \pd[2]{S_i}{t} +
  2S_i \pd{\phi}{t} \cdot \pd{\rho_i}{t} + 2\rho_i \pd{\phi}{t} \cdot \pd{S_i}{t} \\
  &+ 2\phi \pd{\rho_i}{t} \cdot \pd{S_i}{t} = \rho_i S_i \left(  \dd[2]{\phi}{P_{avg}} \cdot {\left( \pd{P_{avg}}{t}\right) }^2 + \dd{\phi}{P_{avg}} \cdot \pd[2]{P_{avg}}{t} \right) \\
  &+ \phi S_i \left( \dd[2]{\rho_i}{P_i} \cdot {\left( \pd{P_i}{t}\right) }^2 + \dd{\rho_i}{P_i} \cdot \pd[2]{P_i}{t} \right) + \phi \rho_i \pd[2]{S_i}{t} \\
  &+ 2S_i \dd{\phi}{P_{avg}} \cdot \pd{P_{avg}}{t} \cdot \dd{\rho_i}{P_i} \cdot \pd{P_i}{t} + 2\rho_i \dd{\phi}{P_{avg}} \cdot \pd{P_{avg}}{t} \cdot \pd{S_i}{t} \\
  &+ 2\phi \dd{\rho_i}{P_i} \cdot \pd{P_i}{t} \cdot \pd{S_i}{t}, \\
  \pd{q_i}{t} &= \dd{q_i}{P_{avg}} \cdot \pd{P_{avg}}{t}.
 \end{aligned}
\end{equation}

Проведя преобразования, аналогичные выполненным в ~\ref{var_division}, получим с учетом результатов ~\eqref{calcs1} следующие уравнения для определения
среднего давления и насыщенности водной фазы:

\begin{equation} \label{eq:full_pressure}
 \begin{aligned}
  &\left[ \frac{1}{\rho_n} div (\rho_n\lambda_n) + \frac{1}{\rho_w} div (\rho_w\lambda_w) \right] \nabla P_{avg} + 
  \frac{1}{2} \left[ \frac{1}{\rho_n} div (\rho_n\lambda_n) + \frac{1}{\rho_w} div (\rho_w\lambda_w) \right] \nabla P_c \\
  & + q_t = \left[ \dd{\phi}{P_{avg}} + \phi (S_n c_n + S_w c_w) \right] \left( \pd{P_{avg}}{t} + \tau \pd[2]{P_{avg}}{t} \right) \\
  & + \frac{1}{2} \phi (S_n c_n - S_w c_w) \left( \pd{P_{c}}{t} + \tau \pd[2]{P_c}{t} \right)
  + g \left[ \frac{1}{\rho_n} div (\rho_n^2\lambda_n) + \frac{1}{\rho_w} div (\rho_w^2\lambda_w) \right] \nabla z \\
  & + \tau \left[ \dd[2]{\phi}{P_{avg}} + \phi \left( \frac{1}{\rho_n} S_n \dd[2]{\rho_n}{P_n} + \frac{1}{\rho_w} S_w \dd[2]{\rho_w}{P_w} \right)
  + 2(S_n c_n + S_w c_w) \dd{\phi}{P_{avg}} \right] {\left( \pd{P_{avg}}{t}\right) }^2 \\
  & + \tau \left[ \phi \left( \frac{1}{\rho_n} S_n \dd[2]{\rho_n}{P_n} - \frac{1}{\rho_w} S_w \dd[2]{\rho_w}{P_w} \right) 
  + (S_n c_n - S_w c_w) \dd{\phi}{P_{avg}} \right] \pd{P_{avg}}{t} \cdot \pd{P_c}{t} \\
  & + \frac{1}{4} \tau \phi \left( \frac{1}{\rho_n} S_n \dd[2]{\rho_n}{P_n} + \frac{1}{\rho_w} S_w \dd[2]{\rho_w}{P_w} \right) {\left( \pd{P_c}{t}\right) }^2 
  - \tau \left(\frac{1}{\rho_n}\dd{q_n}{P_{avg}} + \frac{1}{\rho_w}\dd{q_w}{P_{avg}}\right) \pd{P_{avg}}{t} \\
  & + 2 \tau \phi \left(c_n \pd{S_n}{t} + c_w \pd{S_w}{t} \right) \pd{P_{avg}}{t}+ \tau \phi \left(c_n \pd{S_n}{t} - c_w \pd{S_w}{t} \right) \pd{P_c}{t},
 \end{aligned}
\end{equation}

\begin{equation} \label{eq:full_saturation}
 \begin{aligned}
  &div(\rho_w h_w \nabla S_w) - div(\rho_w f_w)[\overrightarrow{u_t} + \lambda_n(\rho_w - \rho_n)g \nabla z] + q_w \\
  & = \rho_i S_i \dd{\phi}{P_{avg}} \cdot \pd{P_{avg}}{t}
 + \phi S_i \dd{\rho_i}{P_i} \cdot \pd{P_i}{t} + \phi \rho_i \pd{S_i}{t} \\
  & + \tau \pd[2]{(\phi \rho_w S_w)}{t} - \tau \dd{q_w}{P_{avg}} \cdot \pd{P_{avg}}{t}.
 \end{aligned}
\end{equation} 

\subsection{Упрощение модели для проведения вычислительного эксперимента}

Поскольку целью данной работы является проверка возможности в принципе снизить ограничение на шаг по времени при
использовании явного численного метода для решения уравнения для среднего давления
с помощью введения релаксации потока в уравнения неразрывности ~\eqref{eq:classic_system1}, то на данном этапе
будем исходить из дополнительных предположений, позволяющих значительно упростить уравнения для среднего давления
и насыщенности, не изменяя существенно их характера в целом. В частности, пористость среды положим постояной ($\phi = const$),
уравнения состояния - линейными $\left( \dd[2]{\rho_i}{P_i} = 0 \right)$, капиллярными эффектами пренебрежем ($P_w = P_n = P_{avg}$).
При перечисленных допущениях, действуя аналогично ~\ref{var_division}, получим уравнение для определения давления:
\begin{equation} \label{eq:mod_pressure}
 \begin{aligned}
  &\left[ \frac{1}{\rho_n} div (\rho_n\lambda_n) + \frac{1}{\rho_w} div (\rho_w\lambda_w) \right] \nabla P_{avg}
  + q_t \\
  & = \phi (S_n c_n + S_w c_w) \left( \pd{P_{avg}}{t} + \tau \pd[2]{P_{avg}}{t} \right)
  + g \left[ \frac{1}{\rho_n} div (\rho_n^2\lambda_n) + \frac{1}{\rho_w} div (\rho_w^2\lambda_w) \right] \nabla z \\
  & -  \tau \left(\frac{1}{\rho_n}\dd{q_n}{P_{avg}} + \frac{1}{\rho_w}\dd{q_w}{P_{avg}}\right) \pd{P_{avg}}{t} \\
  & + 2 \tau \phi (c_n \pd{S_n}{t} + c_w\pd{S_w}{t}) \pd{P_{avg}}{t}.
 \end{aligned}
\end{equation}

Несмотря на то, что полностью избавиться от насыщенности в уравнении не получилось, в силу малости слагаемого
$2 \tau \phi (c_n \pd{S_n}{t} + c_w\pd{S_w}{t}) \pd{P_{avg}}{t}$
при вычислениях будем считать допустимым задание $\pd{S_i}{t}$ явно, т.е. будем использовать данные
значения с предыдущего временного слоя.

А для определения насыщенности достаточно использовать исходное уравнение неразрывности ~\eqref{eq:mod_system},
раскрыв слагаемые, в которых присутствуют частные производные по времени от произведения переменных. Таким образом,

\begin{equation} \label{eq:mod_saturation}
 \begin{aligned}
  & div (\rho_w\lambda_w) \nabla P_{avg} + q_w = g \cdot div (\rho_w^2\lambda_w) \nabla z + \phi \rho_w S_w c_w  \pd{P_{avg}}{t}
  + \phi \rho_w \pd{S_w}{t} \\
  & + \tau \phi \rho_w \pd[2]{S_w}{t} + \tau \phi \rho_w S_w c_w \pd[2]{P_{avg}}{t}
  + 2 \tau \phi \rho_w c_w \pd{P_{avg}}{t} \cdot \pd{S_w}{t} - \tau \dd{q_w}{P_{avg}} \cdot \pd{P_{avg}}{t}.
 \end{aligned}
\end{equation}

\subsection{Относительные фазовые проницаемости}
Относительные фазовые проницаемости определяются в~работе в~соответствии с~
приближением Стоуна\cite{Aziz-Settari}:

\begin{equation*}
  k_{w}(\overline{S}_w)=
  \begin{cases}
  &\overline{S}_w^\frac{1}{2} \left( 1-\left( 1-\overline{S}_w^\frac{n}{n-1} \right) ^\frac{n-1}{n} \right) ^2
  \text{ , $0<\overline{S}_w<1$}\\
  &1 \text{ , $\overline{S}_w\ge 1$}\\
  &0 \text{ , $\overline{S}_w\le 0$}
\end{cases} 
\end{equation*}
\\
\begin{equation*}
  k_{n}(\overline{S}_w)=
  \begin{cases}
  &(1-\overline{S}_w)^\frac{1}{2} \left(1-\overline{S}_w^\frac{n}{n-1} \right) ^\frac{2(n-1)}{n}
  \text{ , $0<\overline{S}_w<1$}\\
  &1 \text{ , $\overline{S}_w\le 0$}\\
  &0 \text{ , $\overline{S}_w\ge 1$}
  \end{cases}\text { , }
\end{equation*}

$\overline{S}_w$ -- эффективная насыщенность водной фазы:
$\overline{S}_w={\dfrac{S_w-S_{wr}}{1-S_{wr}-S_{nr}}}$, где $S_{wr}$,
$S_{nr}$ -- остаточные насыщенности фаз.