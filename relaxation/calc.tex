\section{Вычислительный эксперимент}

\subsection{Постановка тестовой задачи}
\label{test_task}
Рассмотрим одномерную задачу двухфазной двухкомпонентной фильтрации ~(присутствуют только вода и легкая нефть в жидком состоянии)
в однородной изотропной пористой среде с учетом упрощений раздела ~\ref{model_simplification}.
Исследуемая область имеет форму параллелепипеда со сторонами 5м, 1м, 1м, но в силу симметрии задача сводится к одномерной.
Зафиксируем шаг по пространству $h=0.05\text{м}$.
Зададим следующие начальные условия: $S_w = 0.35$, $P_{avg}$ линейно убывает от $3.3\cdot 10^7$Па к $3.0\cdot 10^7$Па слева направо.
Граничные условия: $S_w|_{\text{лев}} = 0.7$, $S_w|_{\text{прав}} = 0.35$,
$P_{avg}|_{\text{лев}} = 3.3\cdot 10^7$Па, $P_{avg}|_{\text{прав}} = 3.0\cdot 10^7$Па.

\begin{table}[H]
\caption{Параметры фаз}
\label{tabular:liquids}
\begin{center}
\begin{tabular}{|c|c|c|}
\hline
Физ. величина & вода, \textit {w} & нефть, \textit {n} \\
\hline
Плотность,  $ {\text{кг}} / {\text{м}^3} $ & 1000 & 850 \\
\hline
Динамическая вязкость, $ \text{Па} \cdot \text{с} $ & $10^{-3}$ & $10^{-2}$ \\
\hline
Сжимаемость, $ \text{Па}^{-1}$ & $4.4 \cdot 10^{-10}$ & $10^{-9}$ \\
\hline
Остаточная насыщенность & 0.05 & 0.05 \\
\hline
\end{tabular}
\end{center}
\end{table}

\begin{table}[H]
\caption{Параметры среды}
\label{tabular:medium}
\begin{center}
\begin{tabular}{|c|c|}
\hline
Пористость & 0.4\\
\hline
Абсолютная проницаемость, $ \text{м}^{2}$ & $6.64 \cdot 10^{-11}$ \\
\hline
\end{tabular}
\end{center}
\end{table}


\subsection{Эталонное решение}
\label{reference sample}
В силу сильной нелинейности системы уравнений ~\ref{eq:classic_system} найти точное общее решение не представляется возможным.
В этом случае в качестве эталонного решения возьмем часто используемый и проверенный на практике IMPES-метод с расчетным шагом по
времени $\Delta t = 10^{-4}\text{с}$. Величина шага подобрана с помощью сравнения результатов расчетов при различных шагах.
Дальнейшее уменьшение шага не привело к существенному изменению результатов. Детальные данные представлены в ~\ref{time_step_impes_cor}.
Зависимости насыщенностей и давления от времени представлены на графиках. Видно постепенное вытеснение нефти водой вплоть до уровня источника.

\newcommand{\printplotoned}[1]
{
\begin{figure}[h!]
\begin{center}
\begin{minipage}[h!]{0.49\textwidth}
\begin{tikzpicture}
  \begin{axis}[legend entries={$S_w$,$S_n$}, axis lines=left, xmax=5, xmin=0, ymax=1, ymin=0, enlargelimits=true, grid=major, width=1\textwidth, xlabel={$x$, м}, ylabel={$S$}]
    \pgfplotstableread[skip first n=1]{data/impes_#1.0s.out}{\mytable}
    \addplot [blue, ultra thick] table [x=0, y=2] {\mytable};
    \addplot [black, ultra thick] table [x=0, y expr=1-\thisrowno{2}] {\mytable};
  \end{axis}
\end{tikzpicture}
\caption{Насыщенности, t=#1с}
\label{t_10_s}
\end{minipage}
\hfill
\begin{minipage}[h!]{0.49\textwidth}
\begin{tikzpicture}
 \begin{axis}[axis lines=left, xmax=5, xmin=0, ymax=330, ymin=300, enlargelimits=true, grid=major, width=1\textwidth, xlabel={$x$, м}, ylabel={$P$, атм}]
    \pgfplotstableread[skip first n=1]{data/impes_#1.0s.out}{\mytable}
    \addplot [green, ultra thick] table [x=0, y=1] {\mytable};
  \end{axis}
\end{tikzpicture}
\caption{Давление, t=#1с}
\label{t_10_p}
\end{minipage}
\end{center}
\end{figure}
}

\printplotoned{10}
\printplotoned{100}
\printplotoned{200}
\printplotoned{500}

\subsection{Зависимость точности IMPES-метода от шага по времени}
\label{time_step_impes_cor}
В таблице представлены погрешности расчета IMPES-методом в моменты времени t=30c, t=60c и t=90c. 
при различных $\Delta t$ по сравнению с $\Delta t = 10^{-5}\text{с}$.
Относительная ошибка между величинами $A$ и $B$ вычислялась следующим образом: $\varepsilon=\sqrt{\sum\limits_{i=1}^N\left({\frac{A_i-B_i}{A_i}}\right)^2} / N$.
Погрешность менее $10^{-6}$ считаем несущественной. Обращаем внимание как на погрешность расчета давления, так и насыщенности водной фазы.
Исследуем данные значения в разные моменты времени для исключения быстрого набегания ошибки.
Из чего делаем вывод, что решение при $\Delta t = 10^{-4}\text{с}$
можно взять в качестве эталонного для продолжительных расчетов. При этом $\varepsilon_p$ меняется незначительно, $\varepsilon_s$ возрасла на $0.4 \cdot 10^{-7}$  за 60с.

\begin{table}[H]
\caption{Погрешность IMPES-метода в зависимости от шага по времени}
\label{tabular:impes}
\begin{center}
\begin{tabular}{|c|c|c|c|}
\hline
$\Delta t$,c & $\varepsilon_p, \varepsilon_s$ при t=30с & $\varepsilon_p,\, \varepsilon_s$ при t=60с & $\varepsilon_p,\, \varepsilon_s$ при t=90с \\
\hline
$10^{-3}$ & $1.4 \cdot 10^{-8}, \, 1.4 \cdot 10^{-6}$ & $1.1  \cdot 10^{-8}, \, 1.6 \cdot 10^{-6}$ & $0.9  \cdot 10^{-8}, \, 1.8 \cdot 10^{-6}$ \\
\hline
$5 \cdot 10^{-4}$ & $6.7 \cdot 10^{-9}, \, 6.8 \cdot 10^{-7}$ & $5.5  \cdot 10^{-9}, \, 7.8 \cdot 10^{-7}$ & $4.7  \cdot 10^{-9}, \, 8.9 \cdot 10^{-7}$ \\
\hline
$10^{-4}$ & $1.2 \cdot 10^{-9}, \, 1.3 \cdot 10^{-7}$ & $1.0  \cdot 10^{-9}, \, 1.4 \cdot 10^{-7}$ & $0.9  \cdot 10^{-9}, \, 1.6 \cdot 10^{-7}$ \\
\hline
$5 \cdot 10^{-5}$ & $5.6  \cdot 10^{-10}, \, 5.6 \cdot 10^{-8}$ & $4.5  \cdot 10^{-10}, \, 6.4 \cdot 10^{-8}$ & $3.9  \cdot 10^{-10}, \, 7.2 \cdot 10^{-8}$ \\
\hline
\end{tabular}
\end{center}
\end{table}


\subsection{Явная схема для модифицированной модели}
\label{mod_model_explicit}
Приведем разностную схему для модели ~\ref{eq:mod_system}.
\begin{eqnarray*} \label{eq:mod_scheme_p}
  & \dfrac{1}{dx^2} \bigg(\dfrac{1}{\rho_{n_i}} \big( \rho_{n_{i+0.5}} \lambda_{n_{i+0.5}} (P_{i+1}^m - P_i^m)
  - \rho_{n_{i-0.5}} \lambda_{n_{i-0.5}} (P_i^m - P_{i-1}^m)\big) + \\
  &+ \dfrac{1}{\rho_{w_i}} \big( \rho_{w_{i+0.5}} \lambda_{w_{i+0.5}} (P_{i+1}^m - P_i^m)
  - \rho_{w_{i-0.5}} \lambda_{w_{i-0.5}} (P_i^m - P_{i-1}^m)\big)\bigg) + q_{t_i} = \\
  &= \phi(S_{n_i}c_{n_i}+S_{w_i}c_{w_i}) \dfrac{P_i^{m+1} - P_i^m}{dt}
  + \dfrac{g}{dx} \bigg(\dfrac{1}{\rho_{n_i}} \big( \rho_{n_{i+0.5}}^2 \lambda_{n_{i+0.5}} 
  - \rho_{n_{i-0.5}}^2 \lambda_{n_{i-0.5}} \big) + \\
  &+ \dfrac{1}{\rho_{w_i}} \big( \rho_{w_{i+0.5}}^2 \lambda_{w_{i+0.5}} 
  - \rho_{w_{i-0.5}}^2 \lambda_{w_{i-0.5}} \big)\bigg)
  - \tau \dfrac{q_{t_i}^m-q_{t_i}^{m-1}}{dt} + \\
  &+ \tau \phi (S_{n_i}c_{n_i}+S_{w_i}c_{w_i}) \dfrac{P_i^{m+1}-2P_i^m+P_i^{m-1}}{dt^2} +\\
  &+ 2 \tau \phi (c_{w_i}-c_{n_i}) \dfrac{S_{w_i}^m-S_{w_i}^{m-1}}{dt} \cdot \dfrac{P_i^{m+1}-P_i^m}{dt}
\end{eqnarray*}
\begin{eqnarray*} \label{eq:mod_scheme_p}
  & \dfrac{1}{dx^2} \bigg(\rho_{w_{i+0.5}} \lambda_{w_{i+0.5}} (P_{i+1}^{m+1} - P_i^{m+1})
  - \rho_{w_{i-0.5}} \lambda_{w_{i-0.5}} (P_i^{m+1} - P_{i-1}^{m+1}) \bigg) + q_{w_i} = \\
  &= \phi\rho_{w_i}S_{w_i}c_{w_i} \dfrac{P_i^{m+1} - P_i^m}{dt} + \phi\rho_{w_i} \dfrac{S_{w_i}^{m+1} - S_{w_i}^m}{dt} +\\ 
  &+ \dfrac{g}{dx} \bigg(\rho_{w_{i+0.5}}^2 \lambda_{w_{i+0.5}} - \rho_{w_{i-0.5}}^2 \lambda_{w_{i-0.5}} \bigg)
  - \tau \dfrac{q_{w_i}^m-q_{w_i}^{m-1}}{dt} + \\
  &+ \tau \phi \rho_{w_i}S_{w_i}c_{w_i} \dfrac{P_i^{m+1}-2P_i^m+P_i^{m-1}}{dt^2}
  + \tau \phi \rho_{w_i} \dfrac{S_{w_i}^{m+1}-2S_{w_i}^m+S_{w_i}^{m-1}}{dt^2} \\
  &+ 2 \tau \phi \rho_{w_i} c_{w_i} \dfrac{S_{w_i}^{m+1}-S_{w_i}^{m}}{dt} \cdot \dfrac{P_i^{m+1}-P_i^m}{dt}
\end{eqnarray*}
Где $i$ -- пространственный индекс, а $m$ -- временной. Опущенный временной индекс соответствует текущему шагу.
Фазовые проницаемости определяются <<вверх по потоку>>.
\begin{equation}
k_{w_{i+0.5}} =
 \begin{cases}
  &k_w(S_{w_i}),\, P_{i+1}-P_i-\rho_{w_i}g\,dx \le 0
  \\
  &k_w(S_{w_{i+1}}),\, P_{i+1}-P_i-\rho_{w_i}g\,dx > 0
 \end{cases}
\end{equation}
Аналогично вычисляется $k_{n_{i+0.5}}$.
Из схем получаем явные уравнения для давления $P_i^{m+1}$ и насыщенности $S_{w_i}^{m+1}$
в каждом расчетном узле. 


\begin{pspicture}(14,6)
 \psframe(1,1)(13,5)
 \psline[linecolor=black](1,3)(13,3)
 \psline[linecolor=black](4,1)(4,5)
 \psline[linecolor=black](7,1)(7,5)
 \psline[linecolor=black](10,1)(10,5)
 \pscircle[linecolor=black](1,1){0.2}
 \pscircle[linecolor=black](7,1){0.2}
 \pscircle[linecolor=black](13,1){0.2}
 \pscircle[linecolor=black](1,3){0.2}
 \pscircle[linecolor=black](7,3){0.2}
 \pscircle[linecolor=black](13,3){0.2}
 \pscircle[linecolor=black](1,5){0.2}
 \pscircle[linecolor=black](7,5){0.2}
 \pscircle[linecolor=black](13,5){0.2}
 \psline{->}(3.5,4)(4.5,4)
 \psline{->}(4.5,3.5)(3.5,3.5)
 \psline{->}(9.5,4)(10.5,4)
 \psline{->}(10.5,3.5)(9.5,3.5)
 \rput(1,0.5){$i-1$}
 \rput(7,0.5){$i$}
 \rput(13,0.5){$i+1$}
 \rput(4,0.5){$i-0.5$}
 \rput(10,0.5){$i+0.5$}
 \rput(0.25,1){$m-1$}
 \rput(0.25,3){$m$}
 \rput(0.25,5){$m+1$}
 \label{pic:scheme}
\end{pspicture}

Для вычисления потока берем значения величин в <<полуцелых>> узлах как среднее арифметическое
значений в узлах. Например, плотность на границе ячеек:
$\rho_{i+0.5}=\dfrac{\rho_i+\rho_{i+1}}{2},\, \rho_{i-0.5}=\dfrac{\rho_i+\rho_{i-1}}{2}$.
$k_i, \lambda_i$ вычисляются аналогично.

Отметим, что разностная схема для модели ~\ref{eq:classic_system} совпадает с ~\ref{mod_model_explicit}, если положить $\tau=0$.

\subsection{Результаты расчетов}
\label{calc_results}
Для проведения расчетов написана программа на языке Python с использованием научных библиотек NumPy, SciPy.
Параметр $\tau$, при котором расчеты были бы устойчивы, подбирался эмпирически в зависимости от размера шага по времени.
Результаты представлены в таблице ~\ref{tabular:results}

\begin{table}[H]
\caption{Погрешность явной схемы относительно эталонного решения при различных $\tau$ при при t=10с}
\label{tabular:results}
\begin{center}
\begin{tabular}{|c|c|c|c|}
\hline
$\Delta t$,c & $\tau$,с & $\varepsilon_p$ & $\varepsilon_s$  \\
\hline
$ 10^{-5}$ & $5 \cdot 10^{-5}$ & $ 2 \cdot 10^{-6}$ & $ 0.003 $ \\
\hline
$ 10^{-4}$ & $2 \cdot 10^{-4}$ & $ 5 \cdot 10^{-7}$ & $ 0.001 $ \\
\hline
$5 \cdot 10^{-4}$ & $0.01$ & $ 10^{-4}$ & $ 0.003 $ \\
\hline
$0.001$ & $0.025$ & $ 5 \cdot 10^{-4}$ & $ 0.03 $ \\
\hline
\end{tabular}
\end{center}
\end{table}

Например, при $h=0.05$м явная двухслойная схема становится неустойчивой при $\Delta t \ge 5 \cdot 10^{-5}$с, а явная
трехслойная при $\Delta t=0.001$с, $\tau=0.025$с позволяет проводить расчеты с приемлимой с точки зрения моделирования
погрешностью $\varepsilon_p=0.0005$, $\varepsilon_s=0.03$ на момент времени $t=10$с.