\section{Введение}
Все большее распространение получает моделирование физических явлений
с учетом времени распространения возмущений и времени выравнивания таких макропараметров,
как давление, температура, плотность.
Множество трудов посвящено применению этого подхода в моделировании динамики
жидкостей и газов. В книге Хасанова~\ref{Hasanov} рассмотрен этот подход к моделированию
двухфазной двухкомпонентной фильтрации несжимаемых жидкостей, то есть классической задачи Баклея-Леверетта.
В данной работе вводится время релаксации потока при моделировании тоже двухфазной двухкомпонентной
фильтрации, но в случае слабосжимаемых жидкостей. Также проведено сравнение используемых в расчетах численных методов.
Исследована возможность снизить ограничение на шаг по времени при расчетах явными схемами
модифицированной модели в сравнении с классической.