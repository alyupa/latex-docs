\section{Введение}
Все большее распространение получает моделирование физических явлений
с учетом времени распространения возмущений и времени выравнивания таких макропараметров,
как давление, температура, плотность.
Множество трудов посвящено применению этого подхода в моделировании динамики
жидкостей и газов. В книге~\cite{Hasanov} этот подход рассмотрен в применении к моделированию
двухфазной двухкомпонентной фильтрации несжимаемых жидкостей, то есть для решения классической задачи Баклея-Леверетта.
В данной работе время релаксации потока вводится при моделировании двухфазной двухкомпонентной
фильтрации слабосжимаемых жидкостей. На примере решения тестовой задачи проведено сравнение используемых в расчетах численных методов,
исследована возможность снизить ограничение на шаг по времени при расчетах явными схемами
модифицированной модели в сравнении с классической.