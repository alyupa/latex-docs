\section{Заключение}
Результаты расчетов показывают, что, используя предложенную модификацию модели, удается существенно ускорить
численное решение, а также сделать параллельные вычисления удобнее. При этом погрешность вычисений остается
приемлимой. В частности, на примере рассмотренной тестовой задачи удалось снизить на порядок ограничение на шаг по времени
при расчетах явными методами.
Также можно добиться линейного ускорения явного метода решения с помощью использования многопроцессорных систем. 
Тогда как в случае IMPES-метода другими авторами были получены максимальные ускорения порядка ...