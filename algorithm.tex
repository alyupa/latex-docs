 \section{Алгоритм решения задачи}  
 \subsection{Последовательность расчетов}
Численное решение полученной системы уравнений разбито на этапы. После
применения начальных условий на каждом шаге по времени выполняется следующая
последовательность действий: 
\begin{enumerate} 
\item Вычисление давлений $P_n$,
$P_g$ через $P_w$ и капиллярные давления. 
\item Вычисление плотностей фаз. 
\item
Нахождение относительных фазовых проницаемостей фаз. 
\item Вычисление скоростей
фаз независимо по всем измерениям. 
\item 
\label{roS} 
Нахождение ${\rho}_iS_i$ на
следующем шаге по времени из уравнения неразрывности явным численным методом.
\item 
\label
{Newton} Решение системы трех нелинейных уравнений методом Ньютона,
в результате чего находим $P_w$, $S_w$, $S_n$ на следующем шаге по времени.
\item Применение граничных условий. 
\item Сохранение полученных значений переменных в текстовый файл
в формате, подходящем для визуализации (если нужно).
\item Обмены данными при многопроцессорных вычислениях.
\end{enumerate} 

\subsection{Численный метод}
Остановимся подробнее на численном методе, используемом на шаге \ref{roS}
описанного выше алгоритма. С целью наилучшего распараллеливания действий при
решении рассматриваемого класса задач выбрана явная двухслойная схема с направленными разностями. 
В данном случае он может быть записан следующим
образом:
\begin{equation*}
  \begin{aligned}
    &\text{Пусть } k, l, m \text{-- текущие индексы по осям $x$, $y$, $z$, соответственно,}\\
    &j \text{-- текущий индекс слоя по времени,}\\
    &i=w,n,g, \Delta t \text{-- шаг по времени,}\\
    &h_x, h_y, h_z \text{-- шаги по пространству по 
    направлениями осей $x$, $y$, $z$};\\
  \end{aligned}
\end{equation*}
\begin{equation*}
  \begin{aligned}
    &x_{1i}= - \frac{P_{i_{k,l,m}}-P_{i_{k-1,l,m}}}{h_x} , \;
    x_{2i}= - \frac{P_{i_{k+1,l,m}}-P_{i_{k,l,m}}}{h_x} ;\\
    &y_{1i}= - \frac{P_{i_{k,l,m}}-P_{i_{k,l-1,m}}}{h_y} + \rho_{i_{k,l,m}} g,\;
    y_{2i}= - \frac{P_{i_{k,l+1,m}}-P_{i_{k,l,m}}}{h_y} + \rho_{i_{k,l,m}} g;\\
    &\text{если задача двумерная, то} \\
    &z_{1i}= 0 ,\;
    z_{2i}= 0 ;\\
    &\text{если задача трехмерная, то} \\
    &z_{1i}= - \frac{P_{i_{k,l,m}}-P_{i_{k,l,m-1}}}{h_z} ,\;
    z_{2i}= - \frac{P_{i_{k,l,m+1}}-P_{i_{k,l,m}}}{h_z} ;\\
    &\chi_{i_{k,l,m}}=-K\frac{k_i(S_{i_{k,l,m}})}{\mu_i} ;\\
  \end{aligned}
\end{equation*}
\begin{eqnarray*}
    &f_{xi}= \left( \dfrac{x_{2i}+|x_{2i}|}{2} - \dfrac{x_{1i}+|x_{1i}|}{2} \right)(-\chi_{i_{k,l,m}} \rho_{i_{k,l,m}}) / h_x
    - \dfrac{x_{1i}+|x_{1i}|}{2} (-\chi_{i_{k-1,l,m}} \rho_{i_{k-1,l,m}}) / h_x \\
    & +  \dfrac{x_{2i}-|x_{2i}|}{2} (-\chi_{i_{k+1,l,m}} \rho_{i_{k+1,l,m}}) / h_x;
\end{eqnarray*}
\begin{eqnarray*}
    &f_{yi}= \left( \dfrac{y_{2i}+|y_{2i}|}{2} - \dfrac{y_{1i}+|y_{1i}|}{2} \right)(-\chi_{i_{k,l,m}} \rho_{i_{k,l,m}}) / h_y
    - \dfrac{y_{1i}+|y_{1i}|}{2} (-\chi_{i_{k,l-1,m}} \rho_{i_{k,l-1,m}}) / h_y \\
    & +  \dfrac{y_{2i}-|y_{2i}|}{2} (-\chi_{i_{k,l+1,m}} \rho_{i_{k,l+1,m}}) / h_y;
\end{eqnarray*}
\begin{eqnarray*}
    &f_{zi}= \left( \dfrac{z_{2i}+|z_{2i}|}{2} - \dfrac{z_{1i}+|z_{1i}|}{2} \right)(-\chi_{i_{k,l,m}} \rho_{i_{k,l,m}}) / h_z
    - \dfrac{z_{1i}+|z_{1i}|}{2} (-\chi_{i_{k,l,m-1}} \rho_{i_{k,l,m-1}}) / h_z \\
    & +  \dfrac{z_{2i}-|z_{2i}|}{2} (-\chi_{i_{k,l,m+1}} \rho_{i_{k,l,m+1}}) / h_z;
\end{eqnarray*}
\begin{equation*}
    (\rho_i S_i)^{j+1}_{k,l,m}=(\rho_i S_i)^{j}_{k,l,m}-\frac{\Delta t}{m}(-q_i+f_{xi}+f_{yi}+f_{zi}).
\end{equation*}

Данный метод обладает первым порядком аппроксимации по времени и пространству, 
является монотонным и устойчив при $\Delta t < h_{min}^2$. 

\newpage
\subsection{Решение системы методом Ньютона} На шаге \ref{Newton}
предложенного алгоритма в каждом узле расчетной сетки возникает следующая система 
уравнений: 
\begin{equation}
\left\{
  \begin{aligned}
    &\rho_{0w}[1+\beta_w (P_w-P_0)]\, S_w=(\rho_w S_w) \\
    &\rho_{0n}[1+\beta_n (P_w+P_{cnw}(S_w)-P_0)]\, S_n=(\rho_n S_n) \\
    &\rho_{0g}\frac{(P_w+P_{cnw}(S_w)+P_{cgn}(1-S_w-S_n))}{P_0}\, (1-S_w-S_n)=(\rho_g S_g) \\
  \end{aligned}
\right.
\end{equation}
$(\rho_i S_i)$,\; $i=w,n,g$ -- известные на данном этапе значения, полученные на
предыдущем шаге алгоритма.

Ее решение проводится методом Ньютона.
До достижения заданной точности выполняем требуемое число итераций метода. Каждая итерация 
состоит из следующей последовательности действий:

Пусть
\begin{equation}
  \begin{aligned}
    F_1=&\rho_{0w}[1+\beta_w (P_w-P_0)]\, S_w-(\rho_w S_w) \\
    F_2=&\rho_{0n}[1+\beta_n (P_w+P_{cnw}(S_w)-P_0)]\, S_n-(\rho_n S_n) \\
    F_3=&\rho_{0g}(P_w+P_{cnw}(S_w)+P_{cgn}(1-S_w-S_n))\, (1-S_w-S_n)/P_0-(\rho_g S_g) \\
  \end{aligned}
\end{equation}
\begin{equation}
A=
\begin{pmatrix}
\dfrac{\partial{F_1}}{\partial{P_w}} & \dfrac{\partial{F_1}}{\partial{S_w}} & \dfrac{\partial{F_1}}{\partial{S_n}}\\[3mm]
\dfrac{\partial{F_2}}{\partial{P_w}} & \dfrac{\partial{F_2}}{\partial{S_w}} & \dfrac{\partial{F_2}}{\partial{S_n}}\\[3mm]
\dfrac{\partial{F_3}}{\partial{P_w}} & \dfrac{\partial{F_3}}{\partial{S_w}} & \dfrac{\partial{F_3}}{\partial{S_n}}\\[3mm]
\end{pmatrix}
\end{equation}

Тогда

\begin{equation}
\begin{pmatrix}
P_w\\
S_w\\
S_g\\
\end{pmatrix}^{new}
=
\begin{pmatrix}
P_w\\
S_w\\
S_g\\
\end{pmatrix}
-A^{-1}
\begin{pmatrix}
F_1\\
F_2\\
F_3\\
\end{pmatrix}
\end{equation}\\

При необходимости сохраняем полученные значения переменных в текстовый файл
в формате, подходящем для визуализации, делаем все необходимые обмены при 
многопроцессорных вычислениях, переходим к расчетам на следующем шаге по 
времени.
