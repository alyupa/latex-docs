 \section{Алгоритм решения задачи}  
 \subsection{Последовательность расчетов}
Численное решение полученной системы уравнений разбито на~ этапы. После
применения начальных условий на каждом шаге по времени выполняется следующая
последовательность действий: 
\begin{enumerate} 
\item Применение граничных условий.
\item Вычисление давлений $P_n$,
$P_g$ через $P_w$ и капиллярные давления. 
\item Вычисление плотностей фаз. 
\item
Нахождение относительных фазовых проницаемостей фаз и~ вязкости.
\item Определение
коэффициентов в~ законе Дарси для~ фаз. 
\item 
\label{roS} 
Нахождение ${\rho}_iS_i$ на~
следующем шаге по времени из~ уравнения неразрывности (\ref{mass}) явным численным методом.
\item 
\label{roE} 
Нахождение внутренней энергии системы на~
следующем шаге по~ времени из~ уравнения сохранения энергии (\ref{Energy_law}) явным численным методом.
\item 
\label
{Newton} Решение нелинейной системы из~ пяти уравнений методом Ньютона,
в результате чего находим $P_w$, $S_w$, $S_n$, $S_g$, $T$ на следующем шаге по~ времени.
\item Сохранение полученных значений переменных в~ текстовый файл
в~ формате, подходящем для~ визуализации.
\item Обмены данными при~ многопроцессорных вычислениях.
\end{enumerate} 

\subsection{Численный метод}
Остановимся подробнее на численном методе, используемом на шаге \ref{roS}
описанного выше алгоритма. Выбран класс явных двухслойных схем
на равномерных декартовых сетках,
допускающих эффективное распараллеливание решения.
Рассмотрены две различные схемы:
\begin{itemize}
\item с направленными разностями для уравнения неразрывности~\ref{mass};
\item с центральными разностями для модифицированного уравнения неразрывности~\ref{mass_mod}.
\end{itemize}
Для описания схем используем обозначения из \cite{Kalitkin}:
\begin{equation*}
  \begin{aligned}
    &\text{Пусть } n, m, k \text{ -- текущие индексы по осям $x$, $y$, $z$, соответственно,}\\
    &\widehat{\qquad} \text{ -- обозначает значение величины на следующем слое по времени,}\\
    &\tau \text{ -- шаг по времени,}\\
    &h_x, h_y, h_z \text{ -- шаги по пространству по 
    вдоль направлений осей $x$, $y$, $z$};\\
  \end{aligned}
\end{equation*}


\subsubsection*{Схема с направленными разностями}
Представим уравнение (\ref{mass}) в виде:
 \begin{equation}
 	 m \frac{\partial (\rho_i S_i)}{\partial t}+ div(\rho_i \chi_i (grad P_i - {\rho}_i\overrightarrow{g})) = \rho_i q_i,
 \end{equation}
 $$\chi_i=-K\frac{k_i}{\mu_i}.$$
Разностная схема для этого уравнения может быть записана следующим
образом:

\begin{equation*}
  \begin{aligned}
    &x_{li}= \frac{P_{i_{n,m,k}}-P_{i_{n-1,m,k}}}{h_x} , \;
    x_{ri}=  \frac{P_{i_{n+1,m,k}}-P_{i_{n,m,k}}}{h_x} ;\\
    &y_{li}= \frac{P_{i_{n,m,k}}-P_{i_{n,m-1,k}}}{h_y} - \rho_{i_{n,m,k}} g,\;
    y_{ri}=  \frac{P_{i_{n,m+1,k}}-P_{i_{n,m,k}}}{h_y} - \rho_{i_{n,m,k}} g;\\
    &z_{li}= \frac{P_{i_{n,m,k}}-P_{i_{n,m,k-1}}}{h_x} , \;
    z_{ri}=  \frac{P_{i_{n,m,k+1}}-P_{i_{n,m,k}}}{h_x} ;
  \end{aligned}
\end{equation*}
\begin{eqnarray*}
  \begin{aligned}
    f_{xi} =& \frac{1}{2h_x} \bigl{(} (x_{ri} - |x_{ri}| - x_{li} - |x_{li}|) \chi_{i_{n,m,k}} \rho_{i_{n,m,k}} \\
    &- (x_{li} - |x_{li}|) \chi_{i_{n-1,m,k}} \rho_{i_{n-1,m,k}} \\
    &+ (x_{ri} - |x_{ri}|) \chi_{i_{n+1,m,k}} \rho_{i_{n+1,m,k}} \bigr{)};
  \end{aligned}
\end{eqnarray*}
\begin{eqnarray*}
  \begin{aligned}
    f_{yi} =& \frac{1}{2h_y} \bigl{(} (y_{ri} - |y_{ri}| - y_{li} - |y_{li}|) \chi_{i_{n,m,k}} \rho_{i_{n,m,k}} \\
    &- (y_{li} - |y_{li}|) \chi_{i_{n,m-1,k}} \rho_{i_{n,m-1,k}} \\
    &+ (y_{ri} - |y_{ri}|) \chi_{i_{n,m+1,k}} \rho_{i_{n,m+1,k}} \bigr{)};
  \end{aligned}
\end{eqnarray*}
\begin{eqnarray*}
  \begin{aligned}
    f_{zi} =& \frac{1}{2h_z} \bigl{(} (z_{ri} - |z_{ri}| - z_{li} - |z_{li}|) \chi_{i_{n,m,k}} \rho_{i_{n,m,k}} \\
    &- (z_{li} - |z_{li}|) \chi_{i_{n,m,k-1}} \rho_{i_{n,m,k-1}} \\
    &+ (z_{ri} - |z_{ri}|) \chi_{i_{n,m,k+1}} \rho_{i_{n,m,k+1}} \bigr{)};
  \end{aligned}
\end{eqnarray*}
\begin{equation*}
    (\widehat{\rho_i S_i})_{k,l,m}=(\rho_i S_i)_{k,l,m}+\frac{\tau}{m}(\rho_i q_i - f_{xi} - f_{yi} - f_{zi}).
\end{equation*}

Если рассматривается меньшее количество измерений,
соответствующее слагаемое $f_i$ обнуляется.

Данный метод обладает первым порядком аппроксимации по времени
и пространству и устойчив при $\tau < 0.25 h_{min}^2$.

Аналогично строится схема для нахождения $\widehat{E}$.

\subsubsection*{Схема с центральными разностями}

Представим уравнение (\ref{mass_mod}) в виде:
 \begin{equation}
 	 m \frac{\partial (\rho_i S_i)}{\partial t}+ div(\rho_i \chi_i (grad P_i - {\rho}_i\overrightarrow{g})) = \rho_i q_i + l c_i \cdot div(grad(\rho_i S_i)),
 \end{equation}
 $$\chi_i=-K\frac{k_i}{\mu_i}.$$
Разностная схема для~ этого уравнения может быть записана следующим
образом:

\begin{eqnarray*}
  \begin{aligned}
    f_{xi} =& \chi_{i_{n,m,k}} \rho_{i_{n,m,k}} \dfrac{P_{i_{n+1,m,k}} - 2P_{i_{n,m,k}} + P_{i_{n-1,m,k}}}{h_x^2} \\
    &+ \dfrac{P_{i_{n+1,m,k}}-P_{i_{n-1,m,k}}}{2h_x} \cdot \dfrac{\chi_{i_{n+1,m,k}} \rho_{i_{n+1,m,k}}-\chi_{i_{n-1,m,k}} \rho_{i_{n-1,m,k}}}{2h_x} \\
    &- lc_i\dfrac{\rho_{i_{n+1,m,k}}S_{i_{n+1,m,k}} - 2\rho_{i_{n,m,k}}S_{i_{n,m,k}} + \rho_{i_{n-1,m,k}}S_{i_{n-1,m,k}}}{h_x^2};
  \end{aligned}
\end{eqnarray*}
\begin{eqnarray*}
  \begin{aligned}
    f_{yi} =& \chi_{i_{n,m,k}} \rho_{i_{n,m,k}} \dfrac{P_{i_{n,m+1,k}} - 2P_{i_{n,m,k}} + P_{i_{n,m-1,k}}}{h_y^2} \\
    &+ \dfrac{P_{i_{n,m+1,k}}-P_{i_{n,m-1,k}}}{2h_y} \cdot \dfrac{\chi_{i_{n,m+1,k}} \rho_{i_{n,m,k}}-\chi_{i_{n,m-1,k}} \rho_{i_{n,m-1,k}}}{2h_y} \\
    &- lc_i\dfrac{\rho_{i_{n,m+1,k}}S_{i_{n,m+1,k}} - 2\rho_{i_{n,m,k}}S_{i_{n,m,k}} + \rho_{i_{n,m-1,k}}S_{i_{n,m-1,k}}}{h_y^2};
  \end{aligned}
    \end{eqnarray*}
\begin{eqnarray*}
  \begin{aligned}
    f_{zi} =& \chi_{i_{n,m,k}} \rho_{i_{n,m,k}} \dfrac{P_{i_{n,m,k+1}} - 2P_{i_{n,m,k}} + P_{i_{n,m,k-1}}}{h_z^2} \\
    &+ \dfrac{P_{i_{n,m,k+1}}-P_{i_{n,m,k-1}}}{2h_z} \cdot \dfrac{\chi_{i_{n,m,k+1}} \rho_{i_{n,m,k+1}}-\chi_{i_{n,m,k-1}} \rho_{i_{n,m,k-1}}}{2h_z} \\
    &- lc_i\dfrac{\rho_{i_{n,m,k+1}}S_{i_{n,m,k+1}} - 2\rho_{i_{n,m,k}}S_{i_{n,m,k}} + \rho_{i_{n,m,k-1}}S_{i_{n,m,k-1}}}{h_z^2};
  \end{aligned}
\end{eqnarray*}
\begin{equation*}
    (\widehat{\rho_i S_i})_{k,l,m}=(\rho_i S_i)_{k,l,m}+\frac{\tau}{m}(\rho_i q_i - f_{xi} - f_{yi} - f_{zi}).
\end{equation*}

Данный метод позволяет увеличить точность и~ снизить ограничение
на~ шаг по~ времени. Благодаря наличию дополнительного члена в~ правой
части уравнения он является устойчивым.

\newpage
\subsection{Решение системы методом Ньютона} На~ шаге \ref{Newton}
предложенного алгоритма в каждом узле расчетной сетки возникает 
нелинейная система уравнений.
Ее решение проводится методом Ньютона\cite{Kalitkin}, выполняется семь
итераций метода. Каждая итерация 
состоит из~ следующей последовательности действий:
\begin{eqnarray*}
  \begin{aligned}
    F_1=\ &\rho_w(P_w, T) S_w - (\widehat{\rho_w S_w}) \\
    F_2=\ &\rho_n(P_w+P_{cnw}(S_w)) S_n - (\widehat{\rho_n S_n}) \\
    F_3=\ &\rho_g(P_w+P_{cnw}(S_w)+P_{cgn}(S_g)) S_g - (\widehat{\rho_g S_g}) \\
    F_4=\ & m \Big{(}S_w \big{(}\rho_w(P_w, T) H_w(T) - P_w\big{)} \\
	 &+ S_n \big{(}\rho_n(P_w+P_{cnw}(S_w), T) H_n(T) - P_w - P_{cnw}(S_w)\big{)} \\
	 &+ S_g \big{(}\rho_g(P_w+P_{cnw}(S_w)+P_{cgn}(S_g), T) H_g(T) - P_w - P_{cnw}(S_w) - P_{cgn}(S_g)\big{)}
	 \Big{)}\\
	 &+ (1-m) (\rho_r H_r - P_w) - \widehat{E} \\
    F_5=\ &S_w+S_n+S_g - 1
  \end{aligned}
\end{eqnarray*}
\begin{equation}
A=
\begin{pmatrix}
\dfrac{\partial{F_1}}{\partial{P_w}} & \dfrac{\partial{F_1}}{\partial{S_w}} & \dfrac{\partial{F_1}}{\partial{S_n}} & \dfrac{\partial{F_1}}{\partial{S_g}} & \dfrac{\partial{F_1}}{\partial{T}}\\[3mm]
\dfrac{\partial{F_2}}{\partial{P_w}} & \dfrac{\partial{F_2}}{\partial{S_w}} & \dfrac{\partial{F_2}}{\partial{S_n}} & \dfrac{\partial{F_2}}{\partial{S_g}} & \dfrac{\partial{F_2}}{\partial{T}}\\[3mm]
\dfrac{\partial{F_3}}{\partial{P_w}} & \dfrac{\partial{F_3}}{\partial{S_w}} & \dfrac{\partial{F_3}}{\partial{S_n}} & \dfrac{\partial{F_3}}{\partial{S_g}} & \dfrac{\partial{F_3}}{\partial{T}}\\[3mm]
\dfrac{\partial{F_4}}{\partial{P_w}} & \dfrac{\partial{F_4}}{\partial{S_w}} & \dfrac{\partial{F_4}}{\partial{S_n}} & \dfrac{\partial{F_4}}{\partial{S_g}} & \dfrac{\partial{F_4}}{\partial{T}}\\[3mm]
\dfrac{\partial{F_5}}{\partial{P_w}} & \dfrac{\partial{F_5}}{\partial{S_w}} & \dfrac{\partial{F_5}}{\partial{S_n}} & \dfrac{\partial{F_5}}{\partial{S_g}} & \dfrac{\partial{F_5}}{\partial{T}}\\[3mm]
\end{pmatrix}
\end{equation}

Тогда

\begin{equation}
\begin{pmatrix}
P_w\\
S_w\\
S_n\\
S_g\\
T
\end{pmatrix}^{new}
=
\begin{pmatrix}
P_w\\
S_w\\
S_n\\
S_g\\
T
\end{pmatrix}
-A^{-1}
\begin{pmatrix}
F_1\\
F_2\\
F_3\\
F_4\\
F_5
\end{pmatrix}
\end{equation}\\

Для обращения матрицы $A$ используется метод Гаусса с~ выбором
главного элемента\cite{Kalitkin}.
