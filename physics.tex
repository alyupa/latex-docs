%
\section{Физическая постановка задачи}
%
%
	Законы движения флюидов в пористых средах базируются на сохранении
массы, энергии, импульса. С практической точки зрения безнадежно в настоящее
время приложить эти законы непосредственно к рассматриваемым задачам. Основные
физические свойства пористой среды могут быть связаны с элементарным
объемом среды, который должен быть достаточно большим по сравнению с размером
пор. Подземные пространства характеризуются сложной системой пор, каналов,
трещин, размеры которых малы по сравнению с характерными размерами среды, и по
которым может происходить течение жидкости или газа. Количественной
характеристикой пористости среды
служит отношение объема пор к общему объему:
%
	$$m=V_n/V$$
%	 	
(где $m$ -- коэффициент пористости, $V_n$ -- объем пор, $V$ -- общий объем
данного
элемента среды).
%
Течение через такие пористые тела, при котором сила трения флюида
(жидкости, газа) о скелет играет определяющую роль, называется фильтрацией.
Часть системы, все компоненты которой имеют
одинаковые физические и химические свойства, называется фазой. 

Главными характеристиками движения многофазной системы являются насыщенности и
скорости фильтрации каждой из фаз. Насыщенностью $S_i$  порового пространства
$i$-ой фазой называется доля объема пор, занятая этой фазой в элементарном
объеме:
%
\begin{equation} 
S_i=\frac{\Delta V_i}{\Delta V}, i=1,2\ldots n,{\quad}\sum_{i=1}^{n}S_i=1 
\end{equation}
%
Для каждой фазы существует предельная насыщенность, такая, что при меньших
значениях насыщенности эта фаза неподвижна. Эти значения носят название остаточных 
насыщенностей. Обозначаем их с помощью нижнего индекса $r$ у насыщенностей. Таким 
образом, совместное течение двух фаз имеет место лишь в интервале насыщенностей.
Для работы в данном интервале насыщенностей вводится понятие эффективных 
насыщенностей фаз, их обозначаем с помощью символа верхней черты. Эффективная насыщенность 
$i$-ой фазы трехфазной жидкости может быть записана в виде:
$$\overline{S_i}={\frac{S_i-S_{ir}}{1-\sum_{j=1}^{3}S_{jr}}}$$

Проекция скорости фильтрации $i$-ой фазы в некоторой точке
$\overrightarrow{u_i}$ на некоторое направление равна отношению объемного
расхода данной фазы к площадке, перпендикулярной к указанному направлению.
Основное соотношение теории фильтрации -- закон фильтрации(закон Дарси) -- устанавливает 
связь между вектором скорости фильтрации и тем полем давления, которое вызывает 
фильтрационное течение. Закон Дарси в теории фильтрации заменяет собой уравнение 
движения. Для $i$-ой фазы при учете силы тяжести его можно записать в виде:
\begin{equation}
\label{Darcy}
  \overrightarrow{u_i}=-K \frac{k_i(S_i)}{{\mu}_i}(grad P_i - {\rho}_i\overrightarrow{g}),
\end{equation}
где $K$ -- характеристика пористой среды, называемая
абсолютной проницаемостью, определяемая по данным о фильтрации однородной
жидкости и не зависящая от свойств жидкости; $\mu_i$ -- коэффициент динамической
вязкости $i$-ой фазы; $k_i(S_i)$ -- относительная фазовая проницаемость $i$-ой фазы(может зависеть 
не только от насыщенности $i$-ой фазы), определяемые
экспериментально; $P_i$ -- давление в $i$-ой фазе; ${\rho}_i$ -- плотность $i$-ой фазы;
$\overrightarrow{g}$ -- вектор ускорения свободного падения.
В процессе исследований отечественных и зарубежных специалистов показано, что 
можно выделить верхнюю и нижнюю границы применимости закона Дарси и соответствующие 
им две основные группы причин. Верхняя граница определяется группой причин, связанных 
с проявлением инерционных сил при достаточно высоких скоростях фильтрации. Нижняя --
с взаимодействием жидкости с твердым скелетом пористой среды при достаточно малых 
скоростях фильтрации.

Еще одно фундаментальное соотношение в теории фильтрации - уравнение неразрывности. 
Для $i$-ой фазы оно принимает вид:
 \begin{equation}
 \label{mass}
 	 \frac{\partial (m \rho_i S_i)}{\partial t}+ div(\rho_i \overrightarrow{u_i}) = 0 
 \end{equation}
 в отсутствие объемных источников $i$-ой фазы. При их наличии появляется
соответсвующее слагаемое в правой части уравнения -- $q_i$.

При малых размерах области фильтрации и малых скоростях капиллярные силы могут
превзойти внешний перепад давления, и их необходимо учитывать.
Капиллярные эффекты обусловлены межмолекулярными взаимодействиями двух различных
фаз. Эти силы приводят к появлению угла смачивания на границе раздела двух фаз и
к разрыву давления на этой границе. Разность фазовых давлений есть так
называемое \textit{капиллярное давление}. Соотношения между капиллярными
давлениями и насыщенностями обычно получают по опытным данным как функции насыщенностей.

Для замыкания системы дополнительно вводятся уравнения состояния рассматриваемого флюида
и пористой среды.
