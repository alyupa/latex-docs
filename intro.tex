\section{Введение}
Теория фильтрации, изучающая законы движения жидкостей, газов и их смесей в
пористой среде, имеет обширное практическое применение. На протяжении многих лет
фильтрационные расчеты занимают очень важное место при разработке технологий
добычи нефти и газа, при проектировании, постройке и эксплуатации
гидротехнических  и мелиоративных сооружений, в горном деле, в решении
экологических проблем. В настоящее время, с появлением мощных расчётных систем,
обладающих возможностями обрабатывать огромные объемы информации, задачи фильтрации
становятся все более и более актуальными.

Настоящая работа посвящена проблемам, связанным с течениями трехфазной жидкости 
сквозь пористую среду. Процессы напрямую зависят от свойств почв, их неоднородности. 
Три рассматриваемые фазы: вода, газ и NAPL(от английского Non-Aqueous Phase 
Liquids). К NAPL, например, относятся минеральное топливо, растворители, очищающие средства. 
В зависимости от того, как соотносится плотность вещества с плотностью воды, 
NAPL разделяются на легкие и плотные (первые называются Light NAPL, или LNAPL, их плотность меньше плотности
воды; вторые - Dense NAPL, или DNAPL, соответственно их плотность больше
плотности воды). Бензин, например, относится к первому типу, а тетрахлороэтилен
- ко второму. Эти жидкости не смешиваются с водой и с газом, поэтому при моделировании их течения
в почве говорят  о многофазной фильтрации.

Основной целью работы является получение распределения насыщенностей и давлений
трех фаз в пласте среды в зависимости от времени, начальных и граничных
условий, свойств и неоднородностей почв. 

Численные решения таких задач, также как и других задач газо- и гидродинамики,
требуют больших вычислительных затрат. В то же время существует необходимость в
наиболее быстром получении результатов. Использование
высокопроизводительных параллельных компьютеров с распределенной памятью
позволяет проводить вычисления в течение разумного времени. Поэтому созданные
вычислительные алгоритмы были адаптированы к расчетам на многопроцессорных
вычислительных комплексах. Для вычислений используются стандартная
библиотека MPI и технология Nvidia CUDA(для задействования в расчетах 
высокопроизводительных графических плат). При расчетах применяются явные разностные 
схемы, допускающие эффективное распараллеливание. Вычисления проводятся
на гибридном вычислительном комплексе K-100, ИПМ им. М.В.Келдыша РАН.
