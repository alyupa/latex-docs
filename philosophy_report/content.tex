\section{Введение}

Подземная гидромеханика (или гидравлика) -- наука о движении жидкостей, газов и их смесей в пористых и трещиноватых горных породах. Она является теоретической основой разработки нефтяных и газовых месторождений
и имеет обширные области приложения в гидрогеологии, гидротехнике, инженерной геологии. Поскольку эта наука изучает разновидность механического движения, то ее часто считают разделом механики сплошных сред.
Но в силу широкого практического применения она выделилась в отдельную научную отрасль. Подземная гидромеханика тесно связана с гидродинамикой, газодинамикой, геологией, химией, термодинамикой, экологией, 
методами математического моделирования, теорией дифференциальных уравнений, исследованием численных методов решения полученных дифференциальных уравнений, статистической обработкой данных, использованием мощных
вычислительных ресурсов.
Подземная гидромеханика содержит в себе бесчисленное множество задач из самых разных областей человеческого знания.
Применение комплексного мультидисциплинарного подхода стало особенно актуальным на современном этапе, характеризующимся, с одной стороны, существенным ухудшением структуры запасов нефти и газа, с другой -- созданием
принципиально новых технологий в области исследования и моделирования геологического строения пласта, бурения и закачивания скважин, использованием новых быстродействующих компьютеров для проведения сложных вычислений,
геологического и гидродинамического моделирования. Практическое применение достижений подземной гидромеханики имеет очень важное значение для энергетического сектора, общественной жизни, мировой экономики.

\newpage

\section{Основные понятия теории фильтрации}

Природные жидкости, находящиеся в подземных пустотах -- порах и трещинах горных пород, такие как нефть, газ, вода, принято называть флюидами. Устанавливается содержание понятия <<пористой среды>>, как
носителя жидкостей. В пористой среде имеется бесчисленное количество пустот различной величины и формы, образующих <<поровое пространство>>. Каждая такая пора соединена узким каналом с другими, образуя
в целом полностью сообщающуюся между собой сквозными каналами сложную систему отверстий - ячеек. По этим ячейкам может перемещаться заключенная в среде жидкость. Здесь мы имеем дело с проводящими каналами,
сложенными множеством мельчайших соединяющихся пор, ограниченных пространственно непроницаемыми перемычками или же геометрией проводящей системы. В этом заключается фундаментальная особенность пористой
проводящей системы.
Течение через такие пористые тела, при котором сила трения флюида (жидкости, газа) о скелет играет определяющую роль, называется фильтрацией.
Теория фильтрации -- теоретическая основа подземной гидромеханики, описывающая движение флюидов с позиции механики сплошной среды.

Свойства горной породы (пласта) вмещать и пропускать через себя флюиды называются фильтрационно - емкостными свойствами пласта (ФЕСП), среди которых: пористость, проницаемость, скорость фильтрации и т. д.
Свойства породы определяются путем исследования кернов. Керн -- образец горной породы, извлеченный из скважины посредством специально предназначенного для этого вида бурения.
Часто представляет собой цилиндрическую колонку (столбик) горной породы достаточно прочной, чтобы сохранять монолитность.

В механике сплошной среды вводится понятие элементарного объема -- объем жидкости (газа), настолько малый, что параметры состояния во всех его точках можно считать одинаковыми.
Под <<элементарным объемом>> в теории фильтрации понимают такой физически бесконечно малый объем, в котором заключено большое число пор и зерен, настолько что он достаточно велик по сравнению с размерами пор и зерен породы. 
Для такого элементарного объема вводятся локальные усредненные характеристики системы флюид - пористая среда. В применении к меньшим объемам выводы теории фильтрации становятся неприменимыми.

Количественной характеристикой пористости среды служит отношение объема пор к общему объему: 
\begin{equation}
m=V_n/V, 
\end{equation}
 где $m$ -- коэффициент пористости, $V_n$ -- объем пор, $V$ -- общий объем данного элемента среды.
Наряду с пористостью $m$ иногда вводится понятие <<просветности>>, определяемой для каждого сечения, проходящего через данную точку, как отношение площади активных пор в сечении ко всей площади сечения.
Значение просветности и пористости может изменяться в пределах от 0 до 1.
Наиболее непосредственный способ измерения пористости состоит в том, что сначала измеряется общий объем образца, затем образец разрушается, все поры уничтожаются и измеряется объем твердой фазы. Таким образом
измеряется общая пористость. А самый распространенный метод измерения активной пористости заключается в том, что образец помещают в сосуд, наполненный воздухом или газом. Затем сосуд, в который помещен образец, соединяют с другим эвакуированным
сосудом и измеряют изменение давления. Поровый объем можно подсчитать, используя закон Бойля-Мариотта. Также используются методы пропитки, нагнетания ртути, статистический метод.

Часть системы, все компоненты которой имеют одинаковые физические и химические свойства, называется фазой.
Для многофазных систем вводится понятие насыщенности фазы. Насыщенностью $S_i$  порового пространства
$i$-ой фазы называется доля объема пор, занятая этой фазой в элементарном объеме:
\begin{equation} 
S_i=\frac{\Delta V_i}{\Delta V}, \quad \sum_{i=1}^{n}S_i=1.
\end{equation}
Для каждой фазы существует предельная насыщенность, такая, что при меньших значениях насыщенности эта фаза неподвижна. Эти значения носят название остаточных насыщенностей.
Таким образом, совместное течение нескольких фаз имеет место лишь в определенном интервале насыщенностей. Опишем несколько широко распространенных методов измерения насыщенности. Метод объемного баланса: если образец
пористого объема, пористость которого известна, первоначально не содержит жидкости, а затем в него вводится некоторый объем этой жидкости, то насыщенность непосредственно определяется из условия сохранения объема.
Метод электрического сопротивления: если пористый материал, являющийся плохим проводником, частично заполнен жидкостью, хорошо проводящей электрический ток, то насыщенность материала этой жидкостью может быть
определена путем измерения электрического сопротивления. Этот метод основан на законе Арчи. Метод поглощения рентгеновских лучей основан на падении интенсивности лучей по экспоненциальному закону с коэффициентом,
зависящим от материала.

Проницаемость -- способность горных пород фильтровать сквозь себя флюиды при наличии перепада давления. Вводятся понятия абсолютной, эффективной и относительной проницаемостей.
Проницаемость образца керна, насыщенного одним флюидом, инертным по отношению к породе, зависит целиком и полностью от свойств породы, а не от насыщающего флюида.
Как правило, абсолютной проницаемостью называют проницаемость керна по азоту или по воздуху. Проницаемость породы для отдельно взятого флюида при числе присутствующих в породе фаз, большем единицы, называется эффективной
проницаемостью. Она зависит от степени насыщения флюидами (флюидонасыщенностей) и их физико-химических свойств. Отношение эффективной проницаемости к абсолютной называется относительной проницаемостью.
Определяемая законом Дарси проницаемость представляет собой макроскопическую характеристику пористого материала. Поэтому имеет смысл говорить о проницаемости только больших образцов,
содержащих достаточно много пор. Проблема усреднения проницаемости, и особенно относительных фазовых проницаемостей, является очень сложной и до сих пор остается областью активных научных исследований.

Проекция скорости фильтрации в некоторой точке на некоторое направление равна отношению объемного расхода данной фазы к площадке, перпендикулярной к указанному направлению.

Если две несмешивающиеся жидкости соприкасаются, то на границе между ними возникает скачок давления, величина которого зависит от кривизны поверхности раздела, называемый капиллярным давлением.
Смачивающая жидкость (с меньшей удельной свободной энергией поверхности раздела, интерпретируемой как поверхностное натяжение) стремится вытеснить несмачивающую. Равновесие наступает тогда, когда смачивающая жидкость 
соберется в тех порах и щелях, которые обеспечивают наибольшую кривизну поверхности раздела между жидкостями. Наличие капиллярного давления влияет на процесс фильтрации, поэтому многими учеными исследовались
способы нахождения зависимости капиллярных давлений от насыщенностей.

Выделяют типы течений в пористых средах аналогично гидравлике. Ламинарное течение характеризуется фиксированным набором линий тока. Это означает, что элементы жидкости, проходящие через одну и ту же точку пространства,
следуют по одной и той же траектории. В противоположность этому в турбулентном потоке между траекториями элементов жидкости имеется лишь частичная связь. Вязкость является мерой внутреннего трения в жидкости.

Законы движения флюидов в пористых средах базируются на фундаментальных законах сохранения массы, энергии, импульса.

\newpage

\section{Исторические этапы развития подземной гидромеханики}

Историю развития подземной гидромеханики можно разделить на два периода. В течение первого периода, начавшегося в середине XIX века и окончившегося в 20-30-x~гг. XX века,
подземная гидромеханика развивалась почти исключительно под влиянием запросов техники водоснабжения, ирригации и гидротехнического строительства.
В это время решались общие задачи теории фильтрации, движения естественных подземных водных потоков, притока воды к грунтовым колодцам, артезианским скважинам, водосборным галереям, дренажным каналам и т. д.
В 20-30-x~гг. XX века начался второй период, который длится и по настоящее время. Толчком к исследованиям в области подземной гидромеханики послужило мощное развитие нефтяной промышленности.
Ставились задачи вытеснения нефти водой и газом из пласта в скважины, проблема движения газированной нефти в пористой среде, специфические задачи размещения нефтяных и газовых скважин.

Началом систематического изучения особенностей процессов фильтрации жидкости в пористой среде принято считать труды французского инженера-гидравлика Анри Дарси в середине XIX века.
Под его руководством в~г. Дижоне была создана первая в Европе система городских очистных сооружений с различными фильтрационными засыпками, что значительно улучшило облик города.
Дарси обосновал закон, связывающий скорость фильтрации жидкости в пористой среде с градиентом давления, получивший его имя. Также именем Дакси названа единица измерения проницаемости пористой среды.
Исследования Дарси изложены в его сочинении <<Recherches expérimentales relatives au mouvement de l’eau dans les tuyaux>> (Париж, 1857~г.).
До появления этого труда Дарси инженеры пользовались единственно формулами Прони, которые основаны на небольшом числе опытов и произведены большей частью над трубами диаметром не более двух дюймов.
Дарси доказал, в противоположность господствовавшему мнению, что диаметр трубы, большая или меньшая гладкость ее стенок, оказывают существенное влияние на скорость протекающей по ней жидкости.

Французский инженер, механик и экономист Жюль Дюпюи исследовал дифференциальное уравнение, описывающее движение грунтовых вод. В своей монографии он впервые изложил гидравлическую теорию движения грунтовых вод,
получил формулы для определения объемного расхода скважин, дебитов колодцев и дрен. Эти формулы, названные его именем, нашли широкое применение в практических расчетах. Дренаж — это разветвленная система взаимосвязанных труб (дрен),
которые укладывают с уклоном в сторону водоприемника (канала, кювета, водоёма, дренажного колодца) вокруг или вдоль сооружения (участка). Каждая из дрен имеет на стенках специальную сеть отверстий, расположенных на 
определенном расстоянии друг от друга. Такая система труб впитывает воду из грунта и отводит её за пределы участка. Использование дренажных труб позволяет решить проблему защиты территории и находящихся на ней строений от 
повреждений, связанных с избыточной влажностью (мерзлотой, образованием плесени, луж и весенних наледей), предотвратить затопление подвальных и цокольных помещений, загнивание корневой системы растений.

В современной трактовке закон Дарси может быть представлен в следующем виде:
\begin{equation}
\label{Darcy}
  \overrightarrow{u}=-\frac{k}{\mu}(grad \ p),
\end{equation}
где $\overrightarrow{u}$ -- скорость фильтрации, $k$ -- коэффициент проницаемости, $\mu$ -- динамическая вязкость флюида, $p$ -- приведенное давление.
Границы применимости закона Дарси способы его уточнения являются предметом активных исследований и споров и по сей день.

Форхаймер предложил использовать квадратичную зависимость в законе фильтрации, которая дает хорошее совпадение с экспериментальными данными зависимости между градиентом давления и скоростью фильтрации при б\'{о}льших скоростях,
чем скорости фильтрации, рассчитанные по характерному размеру пор для числа Рейнольдса порядка единицы.
Таким образом, обобщением закона Дарси на случай достаточно больших скоростей фильтрующейся жидкости является закон Форхаймера
\begin{equation}
\label{Darcy}
  k\overrightarrow{u} + k_1|\overrightarrow{u}|\overrightarrow{u}=-grad \ p,
\end{equation}
где $k, \ k1$ -- некоторые коэффициенты.

Существенный вклад в развитие теории напорного и безнапорного движения грунтовых вод внесли также Ж. Буссинеск и Ф. Форхгеймер.
Уравнение Буссинеска описывает форму свободной поверхности жидкости при её течении в пористом грунте. Ф. Форхгеймер -- немецкий профессор -- рассмотрел гидравлические сопротивления, волны перемещения, колебания горизонтов 
воды в уравнительных резервуарах ГЭС.

Основы моделирования пористых сред заложены американским гидрогеологом Чарльзом Слихтером, рассмотревшим модели идеального и фиктивного грунта.
Фиктивным грунтом называется модель пористой среды, состоящей из шариков одинакового диаметра. Под идеальным грунтом (капиллярным) понимается модель пористой среды, поровые каналы которой представляют пучок тонких 
цилиндрических трубок (капилляров) с параллельными осями. Ч. Слихтер развил упрощенную теорию фильтрации, позволяющую сравнивать движение жидкости по поровым каналам с течением жидкости по цилиндрическим трубкам.
По идее Слихтера, все шарообразные частицы, образующие данную пористую среду, уложены во всем ее объеме одинаковым образом по элементам из восьми шаров.
В силу более сложных пространственных соотношений, чем принимал Слихтер, его теория не нашла широкого практического применения. Однако, он показал, что пористость и просветленность фиктивного грунта зависят не от диаметра
частиц, а лишь от плотности их укладки.

Известно много попыток создать теорию, связывающую геометрическую структуру пористого материала с проницаемостью. Весьма полный обзор этих теорий сделал Шейдеггер. Также он успешно применил методы математической
статистики при изучении структуры парового пространства и фильтрации в нем флюидов. В теории Козени пористая среда представляется в виде
связки капиллярных трубок равной длины. Исходя из решения классических гидродинамических уравнений для медленного установившегося течения, Козени показал, что проницаемость этой системы записывается в виде
\begin{equation}
K=\frac{cm^3}{\Sigma^2}, 
\end{equation}
 где $K$ -- проницаемость, $m$ -- коэффициент пористости, $c$ -- безразмерная постоянная, зависящая только от геометрической формы поперечного сечения капиллярных трубок, $\Sigma$ -- удельная поверхность.
Удельная поверхность пористого материала определяется как площадь внутренних поверхностей пор, приходящаяся на единицу объема материала. Впоследствии были предложены многочисленные усовершенствования уравнения Козени.
В одном из которых предполагается учесть искривленность трубок тока в пористой среде, следовательно, путь, проходимый жидкими частицами, больше длины образца.

Основоположниками отечественной школы теории фильтрации являются профессор Н. Е. Жуковский, академики Н. Н. Павловский, Л. С. Лейбензон. Эти выдающиеся ученые, их многочисленные ученики и последователи
создали фундаментальную основу развития теории фильтрации в нашей стране.

В 1889~г. Н. Е. Жуковский опубликовал первую работу по теории фильтрации <<Теоретическое исследование о движении подпочвенных вод>>. Им впервые выведены общие дифференциальные уравнения теории фильтрации, показано, что
напор как функция координат удовлетворяет уравнению Лапласа, указано на математическую аналогию теплопроводности и фильтрации. Им исследованы также вопросы капиллярного поднятия воды в пористой среде, решен ряд задач
о притоке воды к скважинам.

В конце второго десятилетия XX века начал свои исследования в области фильтрации академик Н. Н. Павловский. Павловскому принадлежит определяющая роль в развитии теории фильтрации в гидротехническом направлении.
В опубликованной в 1922~г. докторской диссертации Н. Н. Павловский разработал гидромеханическую модель явления фильтрации, позволившую выполнить строгий вывод дифференциальных уравнений движения жидкости в пористой среде. 
Впервые многие задачи фильтрации жидкости под плотинами были сформулированы
Павловским как краевые задачи математической физики. Решение этих задач в таком аспекте открыло новую эпоху в гидротехнике. Павловский впервые предложил использовать параметр Рейнольдса как критерий существования закона
Дарси. До Павловского Крёбер (в 1884~г.) и Мазони (в 1895~г.) провели ряд исследований с целью установить пределы применимости основного закона фильтрации, но их выводы были принципиально неверны. Крёбер связывал предел
применимости закона Дарси только с величиной диаметра зерна грунта, Мазони -- только с пьезометрическим уклоном. Другие исследователи пытались установить критическую скорость фильтрации, при которой якобы (при всех условиях)
закон Дарси нарушается. 

После ознакомления с работами Павловского Н. Е. Жуковский вернулся к исследованиям в области фильтрации и разработал иной метод (чем у Павловского) решения задачи о фильтрации воды под плотинами при наличии напорной 
поверхности и о фильтрации воды со свободной поверхностью.

Эти идеи Павловского и Жуковского были в последующем развиты в трудах обширной советской школы исследователей: Аравина, Биндемана, Замарина, Ведерникова, Веригина, Галина, Гиринского, Девисона, Козлова, Мелещенко,
Нельсон-Скорнякова, Ненько, Нумерова, Полубариновой-Кочиной, Фальковича и др. Большинство из этих работ связано только с проблемами фильтрации воды под плотинами, через тело земляных плотин, с проблемами притока воды к
ирригационным и дренажным сооружениям и т. д.

С 1921~г. в Баку начались теоретические и экспериментальные исследования академика Л. С. Лейбензона -- основателя советской школы ученых, работающих в области подземной гидравлики именно в связи с проблемами добычи нефти
и газа. Лейбензоном впервые выведены дифференциальные уравнения движения газа и газированной жидкости в пористой среде, выяснены особенности работы газовых скважин, подвергнуты математическому исследованию кривые 
производительности и режим работы нефтяных скважин и пластов, методы подсчета запасов нефти и газа в пластах, проблема вытеснения нефти и газа водой и т. д. В 1934~г. была опубликована капитальная монография Лейбензона --
<<Подземная гидравлика воды, нефти, газа>>, где впервые в мировой литературе систематично изложены основы общей теории фильтрации, а также все важнейшие исследования в области собственно подземной нефтяной гидравлики.

В 1935~г. была издана сводная монография профессора Г. Н. Каменского <<Основы динамики подземных вод>>. Работы Каменского и его школы имели большое значение для популяризации основных идей подземной гидравлики среди 
гидрогеологов и позволили нефтяникам перенять опыт гидрогеологов в области исследования скважин и многих других вопросов.

Немногим позже, в 1937~г., в США вышла монография американского физика М. Маскета -- <<Течение однородных жидкостей в пористой среде>>. В работе при широком использовании математического аппарата были подвергнуты глубокому 
анализу вопросы гидромеханического обоснования основных законов фильтрации, методы определения физических констант горных пород (проницаемость, пористость), вывод дифференциальных уравнений движения однородных жидкостей
(воды, нефти и газа), радиальное и нерадиальное плоское движение жидкостей к стокам, фильтрация под плотинами, теория совершенных и несовершенных скважин, движение жидкости в условиях гравитационного потока, теория движения
жидкости в среде с неоднородной проницаемостью, теория одновременного движения в пласте двух жидкостей, анализ движения водонефтяного контакта и явления конусообразования, теория интерференции скважин, неустановившееся
движение жидкости в пористой среде, движение сжимаемой жидкости или проблема упругого режима, теория газонефтяного фактора и т. д. Многие разделы книги Маскета представляют собой компиляцию и критику работ европейских
ученых, работавших в области теории фильтрации, но из советских ученых Маскет ссылается только на Ведерникова и Б. Девисона. В некоторых вопросах в этой книге Маскет придерживается устаревших или опровергнутых на тот момент 
времени теорий. Также им совершенно не учитываются вопросы морфологии нефтяного коллектора, капиллярные и поверхностно-молекулярные явления. А в 1949~г. была издана вторая фундаментальная монография 
Маскета -- <<Физические основы технологии добычи нефти>>, в которой излагаются физические основы разработки нефтяных и газоконденсатных месторождений, освещается состояние в США вопроса расстановки скважин для месторождений
с различным режимом работы в то время.

Что касается исследования продвижения нефти к забоям скважин, с 1865~г. господствовала американская теория Бриггса, согласно которой единственной силой, продвигавшей нефть в пласте к забоям скважин, могла быть сила
упругости газа, а влияние работы каждой скважины может распространяться в пласте не далее строго определенного расстояния, которое в расчетах принималось равным 600 футам.
Но в 1930-х~гг. школой Н. Т. Линдропа, а также в работах Герольда были приведены факты взаимодействия скважин на очень больших расстояниях друг от друга, была разработана теория режимов нефтеносных месторождений.
На основании этой теории развились представления о балансе пластовых водонапорных систем и о необходимости учитывать гидравлическую связь области разработки пласта с областью напора и областью питания.

В 1940~г. Л. С. Лейбензон возглавил организованную профессором Б. Б. Лапуком группу ученых и инженеров разных специальностей, целью которой была выработка научно обоснованной методики проектирования рациональной
системы разработки нефтяных месторождений. Предполагалось, что решить эту проблему можно лишь на базе комплексного геологического-гидродинамического-технического-экономического анализа. После начала войны группа была 
реорганизована в Проектно-исследовательское бюро при Московском нефтяном институте. Данной группой был получен ряд значительных результатов. 

В 1942-1945~гг. П. Я. Полубаринова-Кочина решила ряд сложных задач подземной гидравлики: о притоке жидкости к скажинам в неоднородной среде, об определении размеров пластовой водонапорной системы и проницаемости пласта
на основании известных дебитов скважин (решение так называемых обратных задач подземной гидравлики), о перемещении контура нефтеносности.

В 1941-1944~гг. ГрозНИИ удалось провести на промыслах Грознефти весьма тщательные исследования скважин и поведения пластов после массовой остановки и пуска скажин. Это дало уточнить гидродинамический анализ различных
методов исследования скважин, улучшить методику их исследования, выяснить особенности неустановившихся процессов перераспределения пластового давления, дало твердые доказательства большого влияния объемной упругости
жидкостей и горной породы на поведение скважин и режим пласта. Несколько позже были выведены дифференциальные уравнения движения упругой жидкости в упругом пласте.

Профессор Б. Б. Лапук доказал, что в условиях установившегося и неустановившегося радиального движения газа к скважинам средневзвешенное по объему пластовое давление может быть с высокой степенью точности приравнено
контурному давлению. Основанный на этом факте приближенный метод позволил Лапуку весьма просто и с высокой степенью точности решить как многие из ранее известных сложных, так и новых задач: некоторые проблемы работы скважин в условиях
гравитационного режима, режима растворенного газа и упругого режима, проблемы разработки газовых месторождений.

Основы теории двухфазной фильтрации, предложенные С. Бакли и М. Левереттом, получили широкое распространение и представляют собой основное содержание модели двухфазной фильтрации. Модель Бакли-Леверетта описывает
упрощенный процесс вытеснения одной жидкости другой. В результате возникает уравнение относительно функции насыщенности, представляющей собой отношение скорости фильтрации (или расхода) вытесняющей фазы (воды) и
суммарной скорости, равное объемной доле воды в потоке двух фаз. При гидродинамических расчетах двухфазных потоков эта функция определяет полноту вытеснения и характер распределения насыщенности по пласту.

Выделяется работа Р. Коллинза, посвященная теории течения жидкостей через пористые материалы. Книга содержит достаточное количество решенных примеров, задач и упражнений, весьма полезных для усвоения 
материала, и может служить хорошим введением в теорию фильтрации. Автор стремится выявить основы теории и продемонстрировать наиболее полезные математические приемы решения рассматриваемых задач. При этом главное внимание
уделяется ясности постановок задач.

Французский гидромеханик А. Упер выполнил несколько работ, посвященных применению вероятностно-статистических методов для решения фильтрационных задач, теории фильтрации жидкостей и газов при нелинейном законе.

Важное место в теории фильтрации занимают вопросы описания механики и теплофизики многофазных сред (газовзвесен, пузырьковых жидкостей, газо- и парожидкостных потоков, смесей взаимонерастворимых жидкостей в пористых телах),
методы описания межфазного взаимодействия в дисперсных средах, теория волн в многофазных средах.
Выдающуюся работу в этом направлении провел Р. И. Нигматулин в 70-80-х~г.г. XX века. В своих трудах он рассматривает нестационарные, в том числе волновые, вибрационные и фильтрационные, а также стационарные движения различных 
гетерогенных, или многофазных смесей. Эффекты неоднофазности существенно осложняют исследование и проявляются с наибольшей полнотой при распространении волн, возникающих при воздействии ударных и вибрационных нагрузок. 
Нигматулиным рассмотрен некоторый класс фильтрационных движений многофазной многокомпонентной смеси нескольких взаиморастворимых жидкостей в пористой среде с образованием кинематических волн применительно к анализу одного
из перспективных методов повышения нефтеотдачи -- метода мицеллярно-полимерного заводнения пласта. Подбор материала в книге иллюстрирует, как на основе современной механики сплошной среды происходит интеграция различных 
разделов механики и физики.

В конце XX - начале XXI веков ряд ученых заинтересовал вопрос возможности получения закона фильтрации жидкости в пористой среде, исходя из уравнений Стокса и Навье-Стокса, которые являются одними из важнейших в гидродинамике 
и применяются в математическом моделировании многих природных явлений и технических задач.

В своем труде, посвященном вопросу осреднения процессов в периодических средах, изданном в 1984~г., Н. С. Бахвалов и Г. П. Панасенко рассматривают вывод закона Дарси с помощью осреднения систем уравнений Стокса и Навье-Стокса.
В результате чего главным членом в осредненной системе, как и в линейном случае, являются уравнения Дарси.

Академиком Б. Н. Четверушкиным в монографии, изданной в 2004~г., изложен новый подход к построению моделей и вычислительных методов в газовой динамике, согласно которому из математической модели исключаются масштабы, меньшие чем это необходимо для выполнения 
исходных физических предположений. Непосредственной основой метода служат модели для одночастичной функции распределения. Используется понятие бесконечно малого объема, когда его минимальный размер определяется
из условия наличия в нем достаточного количества молекул, с тем чтобы выполнялось приближение сплошной среды. Этот подход можно распространить и на теорию фильтрации. Закон Дарси может быть также получен из простых 
размерностных соображений, если представить движение флюида через пористую среду в виде движения вязкой жидкости в плоском канале с расстоянием между стенками равным $d$. Тогда с помощью классической формулы Пуазейля получается, 
что
\begin{equation}
\label{Darcy}
  {\mu}\overrightarrow{u}/d^2\ \sim \ -grad \ p.
\end{equation}
Тем самым видно, что коэффициент $k$ в законе Дарси определяется как через вязкость флюида $\mu$, так и через свойства породы. Закон Дарси вытекает непосредственно из уравнения движения системы Навье-Стокса в пренебрежении
конвективными членами и его осреднении на характерном масштабе среды. Рассматривается функция распределения для фильтрующейся жидкости, кинетическое уравнение Энскога, закон сохранения массы. После чего получается новое
дифференциальное уравнение. Результатом являются дополнительные вычислительные возможности при моделировании процессов фильтрации.

Э. Г. Шифриным и Н. А. Гусевым в 2010~г. была опубликована работа, в которой дается строгий вывод уравнений стационарной фильтрации несжимаемой сильно вязкой жидкости в ограниченной трехмерной области
конечно дисперсной неподвижной пористой среды. В соответствии с теоремой об аппроксимации, доказанной в приложении, предполагается, что течение в поровом пространстве при большой вязкости описывается системой Стокса.
Макроскопические скорость и давление определяются как усреднения по Стеклову соответствующих полей жидкости, продолженных нулем на твердое тело. Из интегральных законов сохранения массы и импульса выводится неоднородная
система Стокса для макровеличин, правая часть которой содержит силу гидродинамического сопротивления твердого тела внутри области, по которой производится усреднение. Определяются понятия изотропности и однородности пористой
среды по отношению к фильтрационному течению. Для этих сред полученные уравнения фильтрации замкнуты относительно макровеличин. В результате доказана неэквивалентность этих уравнений и системы Дарси. При этом закон Дарси
является первым членом асимптотического разложения стационарных уравнений Стокса или Навье-Стокса.

В завершение следует отметить, что также пристального внимания при моделировании фильтрации в пористых средах заслуживают вопросы построения расчетных сеток,
аппроксимации дифференциальных уравнений, выбора соответствующих численных методов, нахождения вычислительных алгоритмов решения получаемых систем уравнений, представления и анализа результатов, обработки больших
объемов геологических данных и т. д., но это уже выходит за рамки данной работы.

\newpage
\section{Заключение}

В последнее время остро встал вопрос о повышении степени извлечения углеводородов из недр, в связи с чем ведутся интенсивные исследования. Это обусловлено исчерпанием легкодоступных, подвижных запасов нефти и газа,
усложнением горно-геологических и термобарических условий разработки месторождений. Фундаментальные гидродинамические аспекты этой проблемы успешно развиваются с помощью привлечения
методов термодинамики, физики, химии, а также современного аппарата математического описания сложных фильтрационных процессов.

Необходимо отметить, что в последние годы создано несколько новых направлений для решения сложных фильтрационных задач. Развитие исследований показало, что традиционные задачи гидродинамической теории фильтрации можно 
сформулировать как стохастические в средах со случайными неоднородностями. В связи с чем активно развивается направление, которое можно назвать стохастической теорией фильтрационных процессов.

В подземной гидромеханике, с одной стороны, внимание исследователей привлекают еще не до конца изученные теоретические аспекты движения флюидов в пористых средах, такие как уточнение и определение границ применимости
закона Дарси, моделирование турбулентных течений, механика течений с учетом фазовых переходов, поведение границы раздела фаз при движении несмешивающихся флюидов, способы опредения параметров пористых сред, вопросы осреднения
величин и т. д.
С другой -- нужды промышленности, необходимость создания математических моделей, которые одновременно описывали бы фильтрационные процессы с достаточной точностью и позволяли бы получать результаты за приемлимое время
и не требовали бы больших затрат.
Здесь приходим к вопросу взаимодействия теории и практики. При всем своем различии эмпирический и теоретический уровни познания взаимосвязаны, граница между ними условна и подвижна. 
Эмпирическое исследование, выявляя с помощью наблюдений и экспериментов новые данные, стимулирует теоретическое познание (которое их обобщает и объясняет), ставит перед ним новые, более сложные задачи. 
С другой стороны, теоретическое познание, развивая и конкретизируя на базе эмпирии новое собственное содержание, открывает новые, более широкие горизонты для эмпирического познания, ориентирует и 
направляет его в поисках новых фактов, способствует совершенствованию его методов и средств и т.п. История развития подземной гидромеханики это наглядно демонстрирует.