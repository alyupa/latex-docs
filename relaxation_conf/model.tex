\newcommand*{\pd}[3][]{\ensuremath{\dfrac{\partial^{#1} #2}{\partial #3^{#1}}}}
\newcommand*{\dd}[3][]{\ensuremath{\dfrac{\mathrm d^{#1} #2}{\mathrm d #3^{#1}}}}
\newcommand{\grad}{\mathop{\mathrm{grad}}\nolimits}
\newcommand{\diver}{\mathop{\mathrm{div}}\nolimits}

Приведем классическую модель. Система уравнений для описания двухфазной изотермической фильтрации в пористой среде
может быть представлена в виде~\cite{Aziz-Settari},~\cite{Basniev}:
\begin{subequations} \label{eq:classic_system}
  \begin{align}[left = \empheqlbrace\,]
    &\pd {(\phi \rho_i S_i)}{t} + \diver{(\rho_i \overrightarrow{u_i})} = q_i, \label{eq:classic_system1} \\
    &\overrightarrow{u_i} = -\frac{K k_i}{\mu_i}(\grad {P_i} - {\rho}_i g \grad {z}), \label{eq:classic_system2} \\
    &P_n = P_w + P_{c}({S}_w), \label{eq:classic_system3} \\
    &S_w + S_n = 1, \label{eq:classic_system4} \\
    &k_i = k_i({S}_w), \label{eq:classic_system5} \\
    &\rho_i = \rho_i(P_i), \label{eq:classic_system6} \\
    &i = w,n \nonumber .
  \end{align}
\end{subequations}

Здесь 
$P_i$ -- давление фазы,
$P_c$ -- капиллярное давление на границе раздела фаз,
$S_i$ -- насыщенность фазы,
${\rho}_i$ -- плотность фазы,
$q_i$ -- источниковые члены,
$K$ -- абсолютная проницаемость среды,
$\phi$ -- пористость среды,
$\overrightarrow{u_i}$ -- скорость фильтрации фазы,
$\mu_i$ -- вязкость фазы,
$k_i$ -- относительная фазовая проницаемость,
$g$ -- ускорение свободного падения,
\eqref{eq:classic_system1} -- уравнения неразрывности,
\eqref{eq:classic_system2} -- закон Дарси,
\eqref{eq:classic_system3} -- связь давлений фаз,
\eqref{eq:classic_system4} -- связь насыщенностей фаз,
\eqref{eq:classic_system5} -- уравнения относительных фазовых проницаемостей,
\eqref{eq:classic_system6} -- уравнения состояния.

Для удобства дальнейших выкладок введем следующие обозначения.
Пусть $q_t = \dfrac{q_n}{\rho_n} + \dfrac{q_w}{\rho_w}$,
$c_n = \dfrac{1}{\rho_n} \dd{\rho_n}{P_n}$,
$c_w = \dfrac{1}{\rho_w} \dd{\rho_w}{P_w}$,
$P_{avg} = \dfrac{P_n + P_w}{2}$,
$P_n = P_{avg} + \dfrac{P_c}{2}$,
$P_w = P_{avg} - \dfrac{P_c}{2}$,
$\lambda_n = -\dfrac{K k_n}{\mu_n}$,
$\lambda_w = -\dfrac{K k_w}{\mu_w}$,
$\overrightarrow{u_t} = \overrightarrow{u_n} + \overrightarrow{u_w}$,
$f_w = \dfrac{\lambda_w}{\lambda_n + \lambda_w}$,
$h_w = - \dfrac{\lambda_n\lambda_w}{\lambda_n + \lambda_w} \dd{P_c}{S_w}$.

Для последующего использования численного метода IMPES (Implicit Pressure Explicit Saturation) --- метода неявного по давлению,
явного по насыщенности, проводится преобразование уравнений модели~\eqref{eq:classic_system} ~\cite{Peaceman}.

Разложим член $\pd {(\phi \rho_i S_i)}{t}$ на сумму слагаемых:
\begin{equation}
 \pd {(\phi \rho_i S_i)}{t} = \rho_i S_i \dd{\phi}{P_{avg}} \cdot \pd{P_{avg}}{t}
 + \phi S_i \dd{\rho_i}{P_i} \cdot \pd{P_i}{t} + \phi \rho_i \pd{S_i}{t}.
\end{equation}

Разделим уравнения неразрывности~\eqref{eq:classic_system1} на $\rho_i$ 
и сложим их. Также используем принятые выше обозначения.
Получим следующее уравнение для определения среднего давления:
\begin{equation} \label{eq:classic_pressure}
 \begin{gathered}
  \frac{1}{\rho_n} \diver{ (\rho_n\lambda_n\grad{ P_{avg}})} + \frac{1}{\rho_w} \diver{ (\rho_w\lambda_w\grad{ P_{avg}})} +\\
  + \frac{1}{2} \left[ \frac{1}{\rho_n} \diver{ (\rho_n\lambda_n\grad{ P_c})} + \frac{1}{\rho_w} \diver{ (\rho_w\lambda_w\grad{ P_c})}\right] + q_t =\\
  = \left[ \dd{\phi}{P_{avg}} + \phi (S_n c_n + S_w c_w) \right] \pd{P_{avg}}{t} +
  \frac{1}{2} \phi (S_n c_n - S_w c_w) \pd{P_{c}}{t} +\\
  + g \left[ \frac{1}{\rho_n} \diver{ (\rho_n^2\lambda_n\grad{ z})} + \frac{1}{\rho_w} \diver{ (\rho_w^2\lambda_w\grad{ z})} \right].
 \end{gathered}
\end{equation}

Получим уравнение для нахождения насыщенности $S_w$, исходя из предположения,
что давление $P_{avg}$ к этому моменту уже известно.
Из~\eqref{eq:classic_system3} следует, что $\grad{ P_c} = \grad{ P_n} - \grad{ P_w}$ или
$\lambda_n\lambda_w\grad{ P_c} = \lambda_n\lambda_w\grad{ P_n} - \lambda_n\lambda_w\grad{ P_w}$.
Используем~\eqref{eq:classic_system1}, ~\eqref{eq:classic_system2} и принятые обозначения.
Тогда
\begin{equation} \label{eq:classic_saturation}
 \begin{gathered}
  \diver{(\rho_w h_w \grad{ S_w})} - \diver{(\rho_w f_w[\overrightarrow{u_t} + \lambda_n(\rho_w - \rho_n)g \grad{ z}])} + \\
  + q_w = \pd{(\phi \rho_w S_w)}{t}.
 \end{gathered}
\end{equation} 
 
Рассмотрим модифицированную модель. Запишем уравнение~\eqref{eq:classic_system1} кратко в виде
\begin{equation} \label{eq:short_system}
   \pd{(\phi \rho_i S_i)}{t} + \diver{ \overrightarrow{Q_i}} = q_i, i = w,n,
\end{equation}
где $\overrightarrow{Q_i}$ -- поток фазы.
Пусть $\overrightarrow{Q_i^D} = \rho_i \overrightarrow{u_i}$ -- поток Дарси.
В модели~\eqref{eq:classic_system} $\overrightarrow{Q_i} = \overrightarrow{Q_i^D}$.
Введем релаксацию потока, положим 

\begin{equation}
 \overrightarrow{Q_i} = \overrightarrow{Q_i^D} - \tau \pd{\overrightarrow{Q_i}}{t},
\end{equation}
где $\tau$ -- параметр релаксации, довольно малая величина, характеризующая время установления равновесия в системе. Откуда следует, что
\begin{equation} \label{eq:div_relax}
 \diver{\overrightarrow{Q_i}} = \diver{\overrightarrow{Q_i^D}} - \tau \diver{\pd{\overrightarrow{Q_i}}{t}}.
\end{equation}

Продифференцируем по времени уравнения~\eqref{eq:short_system} и домножим на $\tau$:
\begin{equation}
  \tau \pd[2]{(\phi \rho_i S_i)}{t} + \tau \pd{(\diver{ \overrightarrow{Q_i}})}{t} = \tau \pd{q_i}{t}.
\end{equation}

С учетом~\eqref{eq:div_relax} от~\eqref{eq:short_system} перейдем к конечному виду модифицированного уравнения неразрывности
путем введения релаксации потока:

\begin{equation} \label{eq:mod_system}
  \tau \pd[2]{(\phi \rho_i S_i)}{t} + \pd{(\phi \rho_i S_i)}{t} + \diver{ \overrightarrow{Q_i^D}} = q_i + \tau \pd{q_i}{t}.
\end{equation}

Поскольку целью данной работы является проверка возможности в принципе снизить ограничение на шаг по времени при
использовании явного численного метода для решения уравнения для среднего давления
с помощью введения релаксации потока в уравнения неразрывности~\eqref{eq:classic_system1}, то на данном этапе
будем исходить из дополнительных предположений, позволяющих значительно упростить уравнения для среднего давления
и насыщенности, не изменяя существенно их характера в целом. В частности, пористость среды положим постояной,
уравнения состояния - линейными, капиллярными эффектами пренебрежем:
\begin{equation}
\phi = const,\; \dd[2]{\rho_i}{P_i} = 0,\;P_w = P_n = P_{avg}.
\end{equation}
При перечисленных допущениях, действуя как для IMPES, получим уравнение для определения давления:
\begin{equation} \label{eq:mod_pressure}
 \begin{gathered}
  \frac{1}{\rho_n} \diver{ (\rho_n\lambda_n\grad{ P_{avg}})} + \frac{1}{\rho_w} \diver{ (\rho_w\lambda_w\grad{ P_{avg}})} + q_t =\\
  = \phi (S_n c_n + S_w c_w) \left( \pd{P_{avg}}{t} + \tau \pd[2]{P_{avg}}{t} \right) +\\
  + g \left[ \frac{1}{\rho_n} \diver{ (\rho_n^2\lambda_n\grad{ z})} + \frac{1}{\rho_w} \diver{ (\rho_w^2\lambda_w\grad{ z})} \right]  -\\
  -  \tau \left(\frac{1}{\rho_n}\dd{q_n}{P_{avg}} + \frac{1}{\rho_w}\dd{q_w}{P_{avg}}\right) \pd{P_{avg}}{t} +\\
  + 2 \tau \phi \left(c_n \pd{S_n}{t} + c_w\pd{S_w}{t}\right) \pd{P_{avg}}{t}.
 \end{gathered}
\end{equation}

Несмотря на то, что полностью избавиться от насыщенности в уравнении не получилось, в силу малости слагаемого
\begin{equation*}
 2 \tau \phi (c_n \pd{S_n}{t} + c_w\pd{S_w}{t}) \pd{P_{avg}}{t}
\end{equation*}
при вычислениях будем считать допустимым задание $\pd{S_i}{t}$ явно --- будем использовать данные
значения с предыдущего временного слоя.

А для определения насыщенности достаточно использовать исходное уравнение неразрывности~\eqref{eq:mod_system},
раскрыв слагаемые, в которых присутствуют частные производные по времени от произведения переменных:
\begin{equation} \label{eq:mod_saturation}
 \begin{gathered}
  \diver{ (\rho_w\lambda_w \grad{ P_{avg}})} + q_w = g \cdot \diver{ (\rho_w^2\lambda_w \grad{ z})} + \phi \rho_w S_w c_w  \pd{P_{avg}}{t} +\\
  + \phi \rho_w \pd{S_w}{t} + \tau \phi \rho_w \pd[2]{S_w}{t} + \tau \phi \rho_w S_w c_w \pd[2]{P_{avg}}{t} +\\
  + 2 \tau \phi \rho_w c_w \pd{P_{avg}}{t} \cdot \pd{S_w}{t} - \tau \dd{q_w}{P_{avg}} \cdot \pd{P_{avg}}{t}.
 \end{gathered}
\end{equation}

Относительные фазовые проницаемости определяются в~работе в~соответствии с~
приближением Стоуна~\cite{Aziz-Settari}.
