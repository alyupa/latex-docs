В работе предложена модификация классической модели для описания изотермической
двухфазной двухкомпонентной фильтрации слабосжимаемых жидкостей в пористых средах, 
полученная введением релаксации потока в уравнениях неразрывности. 
Для получения качественных результатов при моделировании фильтрации сложных
сред необходимо учитывать эффекты релаксации~\cite{Hasanov}.
Обычно при описании релаксационной фильтрации учитывается запаздывание в соотношениях
между скоростью фильтрации и градиентом давления и в уравнениях состояния.
В книге М.М. Хасанова, Г.Т. Булгаковой~\cite{Hasanov} рассмотрены 
некоторые модели релаксационной фильтрации, например, модель двухфаной
двухкомпонентной фильтрации несжимаемых жидкостей для решения классической задачи Баклея-Леверетта.
В данной работе вводится время релаксации потока при моделировании двухфазной двухкомпонентной
фильтрации слабосжимаемых жидкостей. 
