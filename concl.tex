\section{Заключение}

В процессе работы над дипломом:

\begin{itemize}
	\item 	предложены 2D и 3D-модели течений в пористых средах,
	допускающие реализацию явными численными методами;

	\item поставлены и решены задачи просачивания  под действием силы тяжести, а 
	также задачи нефтедобычи, в которых течения жидкостей вызваны
	работой нагнетательных и добывающих скважин, в однородной 
	пористой среде;

	\item 	создан алгоритм расчета;
	
	\item 	алгоритм распараллелен и протестирован на нескольких тестовых задачах 
	фильтрации;
	
	\item	написана программа на языке программирования C/C++ с использованием 
	библиотек MPI, CUDA для решения задач трехфазной фильтрации, 
	представляющая собой подключаемый модуль к комплексу решения задач 
	фильтрации;
	
	\item в необходимом на данном этапе объеме освоена технология 
	программирования на графических платах Nvidia CUDA.

\end{itemize}

В дальнейшем планируется:

\begin{itemize}
	\item	ставить и решать задачи трехфазной фильтрации в
		неоднородных, анизотропных средах.
		
	\item 	улучшить геометрическое деление области при многопроцессорных
	вычислениях.

	\item 	оптимизировать использование графических плат.

	\item 	применить разностные схемы с лучшими свойствами.

\end{itemize}
