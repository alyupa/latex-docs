\documentclass[a4paper,12pt]{article}
%
%\usepackage{cmap} % поиск русских слов в полученном pdf-файле
\usepackage[T2A]{fontenc} % Поддержка русских букв
\usepackage[utf8x]{inputenc}
\usepackage{ucs}
\usepackage[english,russian]{babel}
\usepackage{makeidx} % пакет для создания алфавитного указателя
\usepackage{amsthm,amsfonts,amsmath,amssymb,amscd} % эти пакеты необходимы для набора формул
\usepackage{graphicx}
\usepackage{indentfirst}% Красная строка в первом абзаце
\usepackage{tabularx}
\usepackage[labelsep=period]{caption} % пакет для управления заголовка рисунков и таблиц
\usepackage{subfig} % пакет для расположения нескольких под рисунков в одном окружении figure
\usepackage{longtable}
\usepackage{cite}
% Параметры страницы 
\usepackage[
	a4paper,% includehead, %если добавить эти строки колонтитул наезжает на самый верхний текст страницы
	%includefoot, mag=1000,
	%headsep=0mm, 
	headheight=15pt,
	left=30mm, right=25mm, top=30mm, bottom=30mm
	]{geometry}%для задания геометрии страницы так все работает
\usepackage[pdftex,unicode=true,bookmarksnumbered=true,
colorlinks,linkcolor=black,anchorcolor=black,citecolor=black,urlcolor=black,
bookmarksopen=true,pdfstartview={FitH},pdfpagemode={UseNone},
pdfdisplaydoctitle=true, % в заголовке окна показывает заголовок, а не имя файла
pdfstartpage={1}]{hyperref}
\sloppy					% Избавляемся от переполнений
\clubpenalty=10000		% Запрещаем разрыв страницы после первой строки абзаца
\widowpenalty=10000		% Запрещаем разрыв страницы после последней строки абзаца
%
\newlength{\picturewidth}
\picturewidth=10cm % ширина всех рисунков
\pagestyle{headings} % стандартные колонтитулы из пакета

%
\hypersetup{pdfauthor={Люпа Анастасия}, pdfsubject={Дипломная работа},
pdftitle={Моделирование течения двухфазной жидкости в неоднородной пористой среде с учетом капиллярных сил и слабой сжимаемости}}
\newcommand{\includepic}[2][]{\includegraphics[#1]{#2.png}}
%
\linespread{1.3} % печать через полтора интервала
\graphicspath{{fig/}}

% Мои окружения
\newenvironment{descr}
{\begin{description}}
{\end{description}}

\newenvironment{citate}
{\begin{quote}\begin{flushright}}
{\end{flushright}\end{quote}}

\makeindex % сделаем именной указатель, но печатаем его отдельной командой(если печатаем)
\frenchspacing %пробел после запятой не увеличивается - так принято в России
\righthyphenmin=2 % можно оставлять два символа на строке при переносе

% Задаем данные об авторах и название
\author{Люпа А.А.}
\title{Моделирование течения двухфазной жидкости в неоднородной пористой среде с учетом капиллярных сил и слабой сжимаемости}
\date{}

%%==========================================================================================
\begin{document}
\pagenumbering{arabic}
\numberwithin{equation}{section}
\renewcommand{\theequation}{\thesection.\arabic{equation}}

\clearpage
\begin{titlepage}
\linespread{1}
\centering
%\begin{center}
МИНИСТЕРСТВО ОБРАЗОВАНИЯ РОССИЙСКОЙ ФЕДЕРАЦИИ \newline  \newline  \newline
МОСКОВСКИЙ ФИЗИКО-ТЕХНИЧЕСКИЙ ИНСТИТУТ \newline 
(государственный университет)
\underline{\hspace{\textwidth}}
ФАКУЛЬТЕТ УПРАВЛЕНИЯ И ПРИКЛАДНОЙ МАТЕМАТИКИ \\
КАФЕДРА МАТЕМАТИЧЕСКОГО МОДЕЛИРОВАНИЯ \\
%
\vspace{20ex}
\large{ВЫПУСКНАЯ РАБОТА БАКАЛАВРА} \\
\vspace{2ex}
Математическое моделирование течения трехфазной жидкости в пористых средах с использованием гибридных суперкомпьютеров
\vspace{20ex}

\begin{tabularx}{\textwidth}{XX} %{YW}
	  % \newline
	&   %\newline
	Студент гр.871  \newline
	Люпа А.А.
	  \\
	\mbox{ } & \mbox{ } \\
%
	&   Научный руководитель,  \newline
	к.ф.-м.н. Чурбанова Н.Г.
\end{tabularx}

\vspace{\fill}
МОСКВА 

2012
%\end{center}
\clearpage
\end{titlepage}

\newpage
\tableofcontents
\newpage

\section{Введение}
Все большее распространение получает моделирование физических явлений
с учетом времени распространения возмущений и времени выравнивания таких макропараметров,
как давление, температура, плотность.
Множество трудов посвящено применению этого подхода в моделировании динамики
жидкостей и газов. В книге~\cite{Hasanov} этот подход рассмотрен в применении к моделированию
двухфазной двухкомпонентной фильтрации несжимаемых жидкостей, то есть для решения классической задачи Баклея-Леверетта.
В данной работе время релаксации потока вводится при моделировании двухфазной двухкомпонентной
фильтрации слабосжимаемых жидкостей. На примере решения тестовой задачи проведено сравнение используемых в расчетах численных методов,
исследована возможность снизить ограничение на шаг по времени при расчетах явными схемами
модифицированной модели в сравнении с классической.
\section{Физическая модель}

Законы движения флюидов в пористых средах базируются на сохранении
массы, энергии, импульса. Основные
физические свойства пористой среды связаны с элементарным
объемом среды, который должен быть достаточно большим по сравнению с размером
пор. Подземные пространства характеризуются сложной системой пор, каналов,
трещин, размеры которых малы по сравнению с характерными размерами среды, и по
которым может происходить течение жидкости или газа. Количественной
характеристикой пористости среды
служит отношение объема пор к общему объему:
%
	$$m=V_n/V,$$
%	 	
где $m$ -- коэффициент пористости, $V_n$ -- объем пор, $V$ -- общий объем
данного
элемента среды.
%
Течение через такие пористые тела, при котором сила трения флюида
(жидкости, газа) о скелет играет определяющую роль, называется фильтрацией.
Часть системы, все компоненты которой имеют
одинаковые физические и химические свойства, называется фазой. 

Главными характеристиками движения многофазной системы являются насыщенности и
скорости фильтрации каждой из фаз. Насыщенностью $S_i$  порового пространства
$i$-ой фазы называется доля объема пор, занятая этой фазой в элементарном
объеме:
%
\begin{equation} 
S_i=\frac{\Delta V_i}{\Delta V}, i=1,2\ldots n,{\quad}\sum_{i=1}^{n}S_i=1. 
\end{equation}
%
Для каждой фазы существует предельная насыщенность, такая, что при меньших
значениях насыщенности эта фаза неподвижна. Эти значения носят название остаточных 
насыщенностей. Обозначим их с помощью нижнего индекса $r$ у насыщенностей. Таким 
образом, совместное течение двух фаз имеет место лишь в интервале насыщенностей.
Для работы в данном интервале насыщенностей вводится понятие эффективных 
насыщенностей фаз, их обозначаем с помощью символа верхней черты. Эффективная насыщенность 
$i$-ой фазы может быть представлена в виде:
$$\overline{S_i}={\frac{S_i-S_{ir}}{1-\sum\limits_{k=1}^{n}S_{kr}}}.$$

Проекция скорости фильтрации $i$-ой фазы в некоторой точке
$\overrightarrow{u_i}$ на некоторое направление равна отношению объемного
расхода данной фазы к площадке, перпендикулярной к указанному направлению.
Основное соотношение теории фильтрации -- закон фильтрации(закон Дарси) -- устанавливает 
связь между вектором скорости фильтрации и тем полем давления, которое вызывает 
фильтрационное течение. Закон Дарси в теории фильтрации заменяет собой уравнение 
движения. Для $i$-ой фазы при учете силы тяжести его можно записать в виде:
\begin{equation}
\label{Darcy}
  \overrightarrow{u_i}=-K \frac{k_i(S_i)}{{\mu}_i}(grad P_i - {\rho}_i\overrightarrow{g}),
\end{equation}
где $K$ -- характеристика пористой среды, называемая
абсолютной проницаемостью, определяемая по~ данным о~ фильтрации однородной
жидкости и~ не~ зависящая от~ свойств жидкости; $\mu_i$ -- коэффициент динамической
вязкости $i$-ой фазы ($\mu_i=\mu_i(T)$); $k_i(S_i)$ -- относительная фазовая проницаемость $i$-ой фазы(может зависеть 
не только от насыщенности $i$-ой фазы), определяемая
экспериментально; $P_i$ -- давление в $i$-ой фазе; ${\rho}_i$ -- плотность $i$-ой фазы;
$\overrightarrow{g}$ -- вектор ускорения свободного падения.
Можно выделить верхнюю и нижнюю границы применимости закона Дарси\cite{Aziz-Settari}. Верхняя граница связана 
с~ проявлением инерционных сил при~ достаточно высоких скоростях фильтрации. Нижняя --
с~ взаимодействием жидкости с~ твердым скелетом пористой среды при достаточно малых 
скоростях фильтрации.

Еще одно фундаментальное соотношение в теории фильтрации - уравнение неразрывности. 
Для $i$-ой фазы оно принимает вид:
 \begin{equation}
 \label{mass}
 	 \frac{\partial (m \rho_i S_i)}{\partial t}+ div(\rho_i \overrightarrow{u_i}) = \rho_i q_i.
 \end{equation}
В отсутствие объемных источников $i$-ой фазы
соответсвующее слагаемое в правой части уравнения (${\rho_i}q_i$) обнуляется.

При малых размерах области фильтрации и малых скоростях капиллярные силы могут
превзойти внешний перепад давления, и их необходимо учитывать.
Капиллярные эффекты обусловлены межмолекулярными взаимодействиями двух различных
фаз. Эти силы приводят к~ появлению угла смачивания на границе раздела двух фаз и~
к~ разрыву давления на~ этой границе. Разность фазовых давлений есть так
называемое \textit{капиллярное давление}. Соотношения между капиллярными
давлениями и насыщенностями обычно получают по~ опытным данным как функции насыщенностей.

При неизотермической фильтрации свойства и~ состав флюидов зависят еще от~ одной переменной -- температуры.
Поэтому для~ замыкания системы уравнений фильтрации необходимо дополнительное уравнение --
закон сохранения энергии. Для~ многофазной системы без~ учета химических реакций закон сохранения энергии 
может быть записан в~ виде
\begin{equation}
\label{Energy_law}
  \frac{\partial \left(m {\sum\limits_{i}{\rho_i S_i E_i}} + (1-m){\rho_r E_r}\right)}{\partial t}
    + div(\sum_{i}{\rho_i H_i \overrightarrow{u_i}}) = div(\lambda_{eff} grad T),
\end{equation}
где суммирование производится по~ всем активным фазам, индекс $r$ обозначает твердую породу,
$\lambda_{eff}$ -- эффективный коэффициент теплопроводности:
\begin{equation}
\lambda_{eff}=m\sum_i{S_i\lambda_i} + (1-m)\lambda_r,
\end{equation}
связь между внутренней энергией $E_i$ и энтальпией $H_i$:
\begin{equation}
E_i=H_i-\frac{P_i}{\rho_i},
\end{equation} для~ нахождения энтальпии необходимо определить зависимость теплоемкости
вещества при~ постоянном давлении от~ температуры:
\begin{equation}
H_i=H_{i0}+\int_{T_0}^{T}{C_P(T)dT}.
\end{equation}
Зависимости коэффициентов теплопроводности и теплоемкостей находятся эмпирически.

Для замыкания системы дополнительно вводятся уравнения состояния рассматриваемого флюида
и пористой среды.

%
\section{Математическая модель}
%
В работе проводится моделирование трехфазной неизотермической фильтрации 
несмешивающихся жидкостей и газа с учетом их сжимаемости. Скелет породы
считаем неподвижным, пористость и плотность породы -- постоянными во всем
рассматриваемом объеме, среду -- изотропной, жидкие фазы -- слабосжимаемыми,
газ -- идеальным, температуру -- единой для всех фаз.

Рассматриваются одномерные, двумерные и трехмерные
области, заданные в декартовых координатах.

Рассмотрим более детально описанные в главе, посвященной физической
модели, законы и принципы с учетом введенных ограничений и опишем
используемые зависимости физических свойств фаз.

Пусть трехфазная система представляет собой две жидкие фазы -- вода и легкий
нефтяной продукт(LNAPL -- от английского Light Non-Aqueous Phase Liquids,
плотность меньше плотности воды), и одну газообразную.
Для удобства введем индексные обозначения для разных фаз: $w$ -- вода, $n$ --
легкая нефть, $g$ -- газ.

В качестве базовых переменных выбираем насыщенности, давления фаз и температуру.
Причем для насыщенностей в силу определения справедливо  $S_w + S_n + S_g = 1$.
А давления $P_n$ и $P_g$ отличаются от $P_w$ на величины капиллярных
давлений, являющихся функциями от насыщенностей.

Вводятся эффективные насыщенности каждой из фаз:
$\overline{S_i}={\dfrac{S_i-S_{ir}}{1-S_{wr}-S_{nr}-S_{gr}}}$, где , ${\quad}i=w,n,g$, $S_{wr}$,
$S_{nr}$, $S_{gr}$ -- остаточные насыщенности фаз.

Для описания
капиллярных давлений выбрана приближенная модель Паркера\cite{Parker}:
$$P_{cnw}(\overline{S_w})=P_n-P_w={\frac{1}{\gamma \delta_{nw}}}
\left( \overline{S_w}^{\frac{n}{1-n}}-1 \right)^\frac{1}{n},\;0<\overline{S_w}<1 $$
$$P_{cgn}(\overline{S_g})=P_g-P_n={\frac{1}{\gamma \delta_{gn}}}
\left( (1-\overline{S_g})^{\frac{n}{1-n}}-1 \right)^\frac{1}{n},\;0<\overline{S_g}<1$$
(где $P_{cnw}(\overline{S_w})$ - и $P_{cgn}(\overline{S_g})$- капиллярные давления на границах вода-нефть и нефть-газ, 
соответственно, $n$ и $\gamma $ -- пареметры породы Ван Генухтена из соответствующего приближения
Ван Генухтена\cite{Genuchten}, $\delta_{nw}$, $\delta_{gn}$ -- коэффициенты, определяемые поверхностным натяжением 
жидкостей). В работе используются следующие значения параметров: $n$ = 3.25, $\gamma $ = 0.00048Па$^{-1}$,
$\delta_{nw}$ = 0.67, $\delta_{gn}$ = 2. На участках, где эффективные насыщенности перестают удовлетворять
указанному интервалу, гладко продалжаем указанные функции капиллярных давлений. Изобразим полученные зависимости
на Рис.~\ref{tikz_P_cnw}, Рис.~\ref{tikz_P_cgn}.

\begin{figure}[h]
\begin{minipage}[h]{0.49\linewidth}
\begin{tikzpicture}
  \begin{axis}[grid=major, axis lines=left, enlargelimits=true, width=1\linewidth, xlabel={$\overline{S_w}$}, ylabel={$P_{cnw}$, Па}]
    \addplot[domain=0.001:1.0, samples=100, ultra thick, magenta]{pw(x,3.25,0.00048,0.67)};
  \end{axis}
\end{tikzpicture}
\caption{Зависимость капиллярного давления на граице нефть-вода
$P_{cnw}$ от эффективной насыщенности водной фазы $\overline{S_w}$}
\label{tikz_P_cnw}
\end{minipage}
\hfill
\begin{minipage}[h]{0.49\linewidth}
\begin{tikzpicture}
  \begin{axis}[grid=major, axis lines=left, enlargelimits=true, width=1\linewidth, xlabel={$\overline{S_g}$}, ylabel={$P_{cgn}$, Па}]
    \addplot[domain=0.001:0.999, samples=100, ultra thick, magenta]{pg(x,3.25,0.00048,2.0)};
  \end{axis}
\end{tikzpicture}
\caption{Зависимость капиллярного давления на границе газ-нефть 
$P_{cgn}$ от эффективной насыщенности газовой фазы $\overline{S_g}$}
\label{tikz_P_cgn}
\end{minipage}
\end{figure}


Относительные фазовые проницаемости определяются в работе в соответствии с
приближением Стоуна в модификации Азиза и Сеттари\cite{Aziz-Settari}:

\begin{equation*}
  k_{w}(\overline{S_w})=
  \begin{cases}
  &\overline{S_w}^\frac{1}{2} \left( 1-\left( 1-\overline{S_w}^\frac{n}{n-1} \right) ^\frac{n-1}{n} \right) ^2
  \text{ , $0<\overline{S_w}<1$}\\
  &1 \text{ , $\overline{S_w}\ge 1$}\\
  &0 \text{ , $\overline{S_w}\le 0$}
\end{cases} 
\end{equation*}
\begin{equation*}
  k_{g}(\overline{S_g})=
  \begin{cases}
  &\overline{S_g}^\frac{1}{2} \left( 1-\left ( 1-\overline{S_g} \right) ^\frac{n}{n-1} \right) ^\frac{2(n-1)}{n}
  \text{ , $0<\overline{S_g}<1$}\\
  &1 \text{ , $\overline{S_g}\ge 1$}\\
  &0 \text{ , $\overline{S_g}\le 0$}
  \end{cases} 
\end{equation*}
\begin{equation*}
  k_{n}(\overline{S_w},\overline{S_n})=
  \begin{cases}
  &\dfrac{\overline{S_n} k_{nw}(\overline{S_w})k_{ng}(\overline{S_n})}{(1-\overline{S_w})(\overline{S_w}+\overline{S_n})}
  \text{ , $\overline{S_w}<1, \quad \overline{S_w}+\overline{S_n} >0$}\\
  &0 \text{ , иначе}
  \end{cases}\text { , где}
\end{equation*}\\
\begin{equation*}
  k_{nw}(\overline{S_w})=
  \begin{cases}
  &(1-\overline{S_w})^\frac{1}{2} \left(1-\overline{S_w}^\frac{n}{n-1} \right) ^\frac{2(n-1)}{n}
  \text{ , $0<\overline{S_w}<1$}\\
  &1 \text{ , $\overline{S_w}\le 0$}\\
  &0 \text{ , $\overline{S_w}\ge 1$}
  \end{cases}\text { , }
\end{equation*}
\begin{equation*}
  k_{ng}(\overline{S_n})=
  \begin{cases}
  &\overline{S_n}^\frac{1}{2} \left( 1-\left( 1-\overline{S_n}^\frac{n}{n-1} \right) ^\frac{n-1}{n} \right) ^2 
  \text{ , $0<\overline{S_n}<1$}\\
  &1 \text{ , $\overline{S_n}\ge 1$}\\
  &0 \text{ ,  $\overline{S_n}\le 0$}
\end{cases}
\end{equation*}
 
Таким образом, относительные проницаемости воды и газа являются функциями одной 
переменной(насыщенности одной из фаз), а нефти -- двух. Данные зависимости
изображены на Рис.~\ref{tikz_k_w}, Рис.~\ref{tikz_k_g}, Рис.~\ref{tikz_k_n},
соответственно.

\begin{figure}[h]
\begin{minipage}[h]{0.49\linewidth}
\begin{tikzpicture}
  \begin{axis}[grid=major, axis lines=left, enlargelimits=true, width=1\linewidth, xlabel={$\overline{S_w}$}, ylabel={$k_w$}]
    \addplot[domain=0.0:1.001, samples=100, ultra thick, red]{kw(x,3.25)};
  \end{axis}
\end{tikzpicture}
\caption{Зависимость относительной фазовой проницаемости воды $k_w$
  от эффективной насыщенности водной фазы $\overline{S_w}$}
\label{tikz_k_w}
\end{minipage}
\hfill
\begin{minipage}[h]{0.49\linewidth}
\begin{tikzpicture}
  \begin{axis}[grid=major, axis lines=left, enlargelimits=true, width=1\linewidth, xlabel={$\overline{S_g}$}, ylabel={$k_g$}]
    \addplot[domain=0.0:1.0, samples=100, ultra thick, red]{kg(x,3.25)};
  \end{axis}
\end{tikzpicture}
\caption{Зависимость относительной фазовой проницаемости газа $k_g$ 
  от эффективной насыщенности газовой фазы $\overline{S_g}$}
\label{tikz_k_g}
\end{minipage}
\end{figure}

\begin{figure}[h]
\begin{center}
\begin{tikzpicture}
  \begin{axis}[xtick={0.0,0.2,...,1.0}, ytick={0.0,0.2,...,1.0}, ztick={0.0,0.2,...,1.0}, grid=major, xlabel={$\overline{S_w}$}, ylabel={$\overline{S_n}$}, zlabel={$k_n$}]
    \addplot3[surf, samples=20, domain=0.00:0.99,y domain=0.00:0.99]{kn(x,y)};
  \end{axis}
\end{tikzpicture}
\caption{Зависимость относительной фазовой проницаемости нефти $k_n$ 
  от эффективных насыщенностей водной и нефтяной фаз $\overline{S_w}$ и $\overline{S_n}$}
\label{tikz_k_n}
\end{center}
\end{figure}

Коэффициенты теплопроводности фаз представлены зависимостями:
\begin{equation}
  \begin{aligned}
    &\lambda_w(T)=0.553\times(1-0.003\times(T-T_0)),\\
    &\lambda_n(T)=0.14\times(1-0.001\times(T-T_0)),\\
    &\lambda_g(T)=0.237\times\left(\frac{T}{T_0}\right)^{0.82},\\
    &\lambda_r(T)=1.0.
  \end{aligned}
\end{equation}
Теплоемкости фаз при постоянном давлении:
\begin{equation}
  \begin{aligned}
    &C_{Pw}(T)= 4194 - 1.15 \times (T-T_0) + 0.015 \times (T-T_0)^2,\\
    &C_{Pn}(T)= 1700 - 3.4 \times (T-T_0),\\
    &C_{Pg}(T)= 1000 - 0.119 \times (T-T_0),\\
    &C_{Pr}(T)= 800 - 0.75 \times (T-T_0).
  \end{aligned}
\end{equation}
Динамические вязкости фаз (см. Рис.~\ref{tikz_mu_w}, Рис.~\ref{tikz_mu_n}, Рис.~\ref{tikz_mu_g}):
\begin{equation}
  \begin{aligned}
    &\mu_w(T)=\frac{1}{29.21 \times T - 7506.64},\\
    &\mu_n(T)=7.256\times10^{-10} \times e^{\frac{4141.9}{T}},\\
    &\mu_g(T)=1.717\times10^{-5} \times \left(\frac{T}{T_0}\right)^{0.683}.
  \end{aligned}
\end{equation}
Здесь $T_0=273K$.

\begin{figure}[h]
\begin{minipage}[h]{0.49\linewidth}
\begin{tikzpicture}
  \begin{axis}[grid=major, axis lines=left, enlargelimits=true, width=1\linewidth, ylabel={$\mu_w$, Па$\cdot$с}, xlabel={$T$, К}]
    \addplot[domain=275:350, samples=100, ultra thick, blue]{mw(x)};
  \end{axis}
\end{tikzpicture}
\caption{Зависимость вязкости воды $\mu_w$ от температуры}
\label{tikz_mu_w}
\end{minipage}
\hfill
\begin{minipage}[h]{0.49\linewidth}
\begin{tikzpicture}
  \begin{axis}[grid=major, axis lines=left, enlargelimits=true, width=1\linewidth, ylabel={$\mu_n$, Па$\cdot$с}, xlabel={$T$, К}]
    \addplot[domain=275:350, samples=100, ultra thick, blue]{mn(x)};
  \end{axis}
\end{tikzpicture}
\caption{Зависимость вязкости нефти $\mu_n$ от температуры}
\label{tikz_mu_n}
\end{minipage}
\end{figure}

\begin{figure}[h]
\begin{center}
\begin{tikzpicture}
  \begin{axis}[grid=major, axis lines=left, enlargelimits=true, width=0.49\linewidth, ylabel={$\mu_g$, Па$\cdot$с}, xlabel={$T$, К}]
    \addplot[domain=275:350, samples=100, ultra thick, blue]{mg(x)};
  \end{axis}
\end{tikzpicture}
\caption{Зависимость вязкости воздуха $\mu_g$ от температуры}
\label{tikz_mu_g}
\end{center}
\end{figure}

Перейдем к уравнениям состояний фаз.
Водную и нефтяную фазы считаем слабосжимаемыми и линейно зависящими от перепада температуры:\\
$${\rho}_i = {\rho}_{i0} {(1 + {\beta}_i (P_i-P_0) - {\alpha}_i (T-T_0))},
{\quad}0<{\beta}_{i}{\ll}1,{\quad}0<{\alpha}_{i}{\ll}1,{\quad}i=w,n,$$
где ${\rho}_{i0}$ -- известное значение плотности $i$-ой фазы, соответсвующее
значению давления $P_0$ и температуры $T_0$.

Для газа предполагаем справедливым уравнение состояния идеального
газа:
$${\rho}_g = {\rho}_{g0}{\frac{P_g}{P_0}}{\frac{T_0}{T}}.$$

В работе используются следующие значения:\\
$\rho_r=2000$ кг/м$^3$, $\rho_{w0}=1000$ кг/м$^3$,
$\rho_{n0}=850$ кг/м$^3$, $\rho_{g0}=1.4$ кг/м$^3$,\\
$\beta_w=4.4\cdot10^{-7}$ 1/Па, $\beta_n=10^{-6}$ 1/Па,
$\alpha_w=1.32\cdot10^{-7}$ 1/К, $\alpha_n=9.2\cdot10^{-7}$ 1/К.

Таким образом, полная система уравнений для описания трехфазной системы
принимает вид:
\begin{equation}
\left\{
  \begin{aligned}
    &\frac{\partial \left(m {\sum\limits_{i}{\rho_i S_i E_i(P_i, T)}} + (1-m){\rho_r E_r(P_w, T)}\right)}{\partial t}
      + div(\sum_{i}{\rho_i H_i(T) \overrightarrow{u_i}}) = div(\lambda_{eff} grad T) \\
    &\frac{\partial (m \rho_i S_i)}{\partial t}+ div(\rho_i \overrightarrow{u_i}) = \rho_i q_i \\
    &\overrightarrow{u_i}=-K \frac{k_i}{{\mu_i(T)}}(grad P_i - {\rho}_i\overrightarrow{g})\\
    &i=w,n,g\\
    &P_n=P_w+P_{cnw}(\overline{S_w})\\
    &P_g=P_w+P_{cnw}(\overline{S_w})+P_{cgn}(\overline{S_g})\\
    &S_w + S_n + S_g=1\\
    &k_w=k_w(\overline{S_w}),\quad k_g=k_g(\overline{S_g}),\quad k_n=k_n(\overline{S_w},\overline{S_n})\\
    &\rho_i=\rho_i(P_i,T)
  \end{aligned}
\right.
\end{equation}

Где $\rho_i q_i$ -- источниковые члены, индекс $r$ обозначает твердую породу.
Используемые параметры породы: $K=6.64\cdot 10^{-11}$ м$^2$, $m$=0.4.
Все значения констант приводятся в системе единиц СИ.

В зависимости от конкретной постановки задачи в работе используются различные
граничные условия. На границе поддерживается постоянное давление, или же оно
определяется из условия непротекания фазы (нормальная компонента скорости
фазы на границе равна 0, $\overrightarrow{u_i}_{n} = 0$).
Для насыщенности ставится условие равенства нулю потока через граничную 
поверхность или же задается поток насыщенности в виде известной функции через 
$ \dfrac{\partial S_i}{\partial n}, \; i=w,n, \;  n \text{-- нормаль к границе} $.
Также для насыщенности может быть задано некоторое постоянное значение на границе.
Для температуры тоже ставятся граничные условия первого или второго рода. 

В начальный момент времени считаем известными распределения давления водной 
фазы, насыщенностей фаз и температуры по всей области.

 \section{Алгоритм решения задачи}  
 \subsection{Последовательность расчетов}
Численное решение полученной системы уравнений разбито на~ этапы. После
применения начальных условий на каждом шаге по времени выполняется следующая
последовательность действий: 
\begin{enumerate} 
\item Применение граничных условий.
\item Вычисление давлений $P_n$,
$P_g$ через $P_w$ и капиллярные давления. 
\item Вычисление плотностей фаз. 
\item
Нахождение относительных фазовых проницаемостей фаз и~ вязкости.
\item Определение
коэффициентов в~ законе Дарси для~ фаз. 
\item 
\label{roS} 
Нахождение ${\rho}_iS_i$ на~
следующем шаге по времени из~ уравнения неразрывности (\ref{mass}) явным численным методом.
\item 
\label{roE} 
Нахождение внутренней энергии системы на~
следующем шаге по~ времени из~ уравнения сохранения энергии (\ref{Energy_law}) явным численным методом.
\item 
\label
{Newton} Решение нелинейной системы из~ пяти уравнений методом Ньютона,
в результате чего находим $P_w$, $S_w$, $S_n$, $S_g$, $T$ на следующем шаге по~ времени.
\item Сохранение полученных значений переменных в~ текстовый файл
в~ формате, подходящем для~ визуализации.
\item Обмены данными при~ многопроцессорных вычислениях.
\end{enumerate} 

\subsection{Численный метод}
Остановимся подробнее на численном методе, используемом на шаге \ref{roS}
описанного выше алгоритма. Выбран класс явных двухслойных схем
на равномерных декартовых сетках,
допускающих эффективное распараллеливание решения.
Рассмотрены две различные схемы:
\begin{itemize}
\item с направленными разностями для уравнения неразрывности~\ref{mass};
\item с центральными разностями для модифицированного уравнения неразрывности~\ref{mass_mod}.
\end{itemize}
Для описания схем используем обозначения из \cite{Kalitkin}:
\begin{equation*}
  \begin{aligned}
    &\text{Пусть } n, m, k \text{ -- текущие индексы по осям $x$, $y$, $z$, соответственно,}\\
    &\widehat{\qquad} \text{ -- обозначает значение величины на следующем слое по времени,}\\
    &\tau \text{ -- шаг по времени,}\\
    &h_x, h_y, h_z \text{ -- шаги по пространству по 
    вдоль направлений осей $x$, $y$, $z$};\\
  \end{aligned}
\end{equation*}


\subsubsection*{Схема с направленными разностями}
Представим уравнение (\ref{mass}) в виде:
 \begin{equation}
 	 m \frac{\partial (\rho_i S_i)}{\partial t}+ div(\rho_i \chi_i (grad P_i - {\rho}_i\overrightarrow{g})) = \rho_i q_i,
 \end{equation}
 $$\chi_i=-K\frac{k_i}{\mu_i}.$$
Разностная схема для этого уравнения может быть записана следующим
образом:

\begin{equation*}
  \begin{aligned}
    &x_{li}= \frac{P_{i_{n,m,k}}-P_{i_{n-1,m,k}}}{h_x} , \;
    x_{ri}=  \frac{P_{i_{n+1,m,k}}-P_{i_{n,m,k}}}{h_x} ;\\
    &y_{li}= \frac{P_{i_{n,m,k}}-P_{i_{n,m-1,k}}}{h_y} - \rho_{i_{n,m,k}} g,\;
    y_{ri}=  \frac{P_{i_{n,m+1,k}}-P_{i_{n,m,k}}}{h_y} - \rho_{i_{n,m,k}} g;\\
    &z_{li}= \frac{P_{i_{n,m,k}}-P_{i_{n,m,k-1}}}{h_x} , \;
    z_{ri}=  \frac{P_{i_{n,m,k+1}}-P_{i_{n,m,k}}}{h_x} ;
  \end{aligned}
\end{equation*}
\begin{eqnarray*}
  \begin{aligned}
    f_{xi} =& \frac{1}{2h_x} \bigl{(} (x_{ri} - |x_{ri}| - x_{li} - |x_{li}|) \chi_{i_{n,m,k}} \rho_{i_{n,m,k}} \\
    &- (x_{li} - |x_{li}|) \chi_{i_{n-1,m,k}} \rho_{i_{n-1,m,k}} \\
    &+ (x_{ri} - |x_{ri}|) \chi_{i_{n+1,m,k}} \rho_{i_{n+1,m,k}} \bigr{)};
  \end{aligned}
\end{eqnarray*}
\begin{eqnarray*}
  \begin{aligned}
    f_{yi} =& \frac{1}{2h_y} \bigl{(} (y_{ri} - |y_{ri}| - y_{li} - |y_{li}|) \chi_{i_{n,m,k}} \rho_{i_{n,m,k}} \\
    &- (y_{li} - |y_{li}|) \chi_{i_{n,m-1,k}} \rho_{i_{n,m-1,k}} \\
    &+ (y_{ri} - |y_{ri}|) \chi_{i_{n,m+1,k}} \rho_{i_{n,m+1,k}} \bigr{)};
  \end{aligned}
\end{eqnarray*}
\begin{eqnarray*}
  \begin{aligned}
    f_{zi} =& \frac{1}{2h_z} \bigl{(} (z_{ri} - |z_{ri}| - z_{li} - |z_{li}|) \chi_{i_{n,m,k}} \rho_{i_{n,m,k}} \\
    &- (z_{li} - |z_{li}|) \chi_{i_{n,m,k-1}} \rho_{i_{n,m,k-1}} \\
    &+ (z_{ri} - |z_{ri}|) \chi_{i_{n,m,k+1}} \rho_{i_{n,m,k+1}} \bigr{)};
  \end{aligned}
\end{eqnarray*}
\begin{equation*}
    (\widehat{\rho_i S_i})_{k,l,m}=(\rho_i S_i)_{k,l,m}+\frac{\tau}{m}(\rho_i q_i - f_{xi} - f_{yi} - f_{zi}).
\end{equation*}

Если рассматривается меньшее количество измерений,
соответствующее слагаемое $f_i$ обнуляется.

Данный метод обладает первым порядком аппроксимации по времени
и пространству и устойчив при $\tau < 0.25 h_{min}^2$.

Аналогично строится схема для нахождения $\widehat{E}$.

\subsubsection*{Схема с центральными разностями}

Представим уравнение (\ref{mass_mod}) в виде:
 \begin{equation}
 	 m \frac{\partial (\rho_i S_i)}{\partial t}+ div(\rho_i \chi_i (grad P_i - {\rho}_i\overrightarrow{g})) = \rho_i q_i + l c_i \cdot div(grad(\rho_i S_i)),
 \end{equation}
 $$\chi_i=-K\frac{k_i}{\mu_i}.$$
Разностная схема для~ этого уравнения может быть записана следующим
образом:

\begin{eqnarray*}
  \begin{aligned}
    f_{xi} =& \chi_{i_{n,m,k}} \rho_{i_{n,m,k}} \dfrac{P_{i_{n+1,m,k}} - 2P_{i_{n,m,k}} + P_{i_{n-1,m,k}}}{h_x^2} \\
    &+ \dfrac{P_{i_{n+1,m,k}}-P_{i_{n-1,m,k}}}{2h_x} \cdot \dfrac{\chi_{i_{n+1,m,k}} \rho_{i_{n+1,m,k}}-\chi_{i_{n-1,m,k}} \rho_{i_{n-1,m,k}}}{2h_x} \\
    &- lc_i\dfrac{\rho_{i_{n+1,m,k}}S_{i_{n+1,m,k}} - 2\rho_{i_{n,m,k}}S_{i_{n,m,k}} + \rho_{i_{n-1,m,k}}S_{i_{n-1,m,k}}}{h_x^2};
  \end{aligned}
\end{eqnarray*}
\begin{eqnarray*}
  \begin{aligned}
    f_{yi} =& \chi_{i_{n,m,k}} \rho_{i_{n,m,k}} \dfrac{P_{i_{n,m+1,k}} - 2P_{i_{n,m,k}} + P_{i_{n,m-1,k}}}{h_y^2} \\
    &+ \dfrac{P_{i_{n,m+1,k}}-P_{i_{n,m-1,k}}}{2h_y} \cdot \dfrac{\chi_{i_{n,m+1,k}} \rho_{i_{n,m,k}}-\chi_{i_{n,m-1,k}} \rho_{i_{n,m-1,k}}}{2h_y} \\
    &- lc_i\dfrac{\rho_{i_{n,m+1,k}}S_{i_{n,m+1,k}} - 2\rho_{i_{n,m,k}}S_{i_{n,m,k}} + \rho_{i_{n,m-1,k}}S_{i_{n,m-1,k}}}{h_y^2};
  \end{aligned}
    \end{eqnarray*}
\begin{eqnarray*}
  \begin{aligned}
    f_{zi} =& \chi_{i_{n,m,k}} \rho_{i_{n,m,k}} \dfrac{P_{i_{n,m,k+1}} - 2P_{i_{n,m,k}} + P_{i_{n,m,k-1}}}{h_z^2} \\
    &+ \dfrac{P_{i_{n,m,k+1}}-P_{i_{n,m,k-1}}}{2h_z} \cdot \dfrac{\chi_{i_{n,m,k+1}} \rho_{i_{n,m,k+1}}-\chi_{i_{n,m,k-1}} \rho_{i_{n,m,k-1}}}{2h_z} \\
    &- lc_i\dfrac{\rho_{i_{n,m,k+1}}S_{i_{n,m,k+1}} - 2\rho_{i_{n,m,k}}S_{i_{n,m,k}} + \rho_{i_{n,m,k-1}}S_{i_{n,m,k-1}}}{h_z^2};
  \end{aligned}
\end{eqnarray*}
\begin{equation*}
    (\widehat{\rho_i S_i})_{k,l,m}=(\rho_i S_i)_{k,l,m}+\frac{\tau}{m}(\rho_i q_i - f_{xi} - f_{yi} - f_{zi}).
\end{equation*}

Данный метод позволяет увеличить точность и~ снизить ограничение
на~ шаг по~ времени. Благодаря наличию дополнительного члена в~ правой
части уравнения он является устойчивым.

\newpage
\subsection{Решение системы методом Ньютона} На~ шаге \ref{Newton}
предложенного алгоритма в каждом узле расчетной сетки возникает 
нелинейная система уравнений.
Ее решение проводится методом Ньютона\cite{Kalitkin}, выполняется семь
итераций метода. Каждая итерация 
состоит из~ следующей последовательности действий:
\begin{eqnarray*}
  \begin{aligned}
    F_1=\ &\rho_w(P_w, T) S_w - (\widehat{\rho_w S_w}) \\
    F_2=\ &\rho_n(P_w+P_{cnw}(S_w)) S_n - (\widehat{\rho_n S_n}) \\
    F_3=\ &\rho_g(P_w+P_{cnw}(S_w)+P_{cgn}(S_g)) S_g - (\widehat{\rho_g S_g}) \\
    F_4=\ & m \Big{(}S_w \big{(}\rho_w(P_w, T) H_w(T) - P_w\big{)} \\
	 &+ S_n \big{(}\rho_n(P_w+P_{cnw}(S_w), T) H_n(T) - P_w - P_{cnw}(S_w)\big{)} \\
	 &+ S_g \big{(}\rho_g(P_w+P_{cnw}(S_w)+P_{cgn}(S_g), T) H_g(T) - P_w - P_{cnw}(S_w) - P_{cgn}(S_g)\big{)}
	 \Big{)}\\
	 &+ (1-m) (\rho_r H_r - P_w) - \widehat{E} \\
    F_5=\ &S_w+S_n+S_g - 1
  \end{aligned}
\end{eqnarray*}
\begin{equation}
A=
\begin{pmatrix}
\dfrac{\partial{F_1}}{\partial{P_w}} & \dfrac{\partial{F_1}}{\partial{S_w}} & \dfrac{\partial{F_1}}{\partial{S_n}} & \dfrac{\partial{F_1}}{\partial{S_g}} & \dfrac{\partial{F_1}}{\partial{T}}\\[3mm]
\dfrac{\partial{F_2}}{\partial{P_w}} & \dfrac{\partial{F_2}}{\partial{S_w}} & \dfrac{\partial{F_2}}{\partial{S_n}} & \dfrac{\partial{F_2}}{\partial{S_g}} & \dfrac{\partial{F_2}}{\partial{T}}\\[3mm]
\dfrac{\partial{F_3}}{\partial{P_w}} & \dfrac{\partial{F_3}}{\partial{S_w}} & \dfrac{\partial{F_3}}{\partial{S_n}} & \dfrac{\partial{F_3}}{\partial{S_g}} & \dfrac{\partial{F_3}}{\partial{T}}\\[3mm]
\dfrac{\partial{F_4}}{\partial{P_w}} & \dfrac{\partial{F_4}}{\partial{S_w}} & \dfrac{\partial{F_4}}{\partial{S_n}} & \dfrac{\partial{F_4}}{\partial{S_g}} & \dfrac{\partial{F_4}}{\partial{T}}\\[3mm]
\dfrac{\partial{F_5}}{\partial{P_w}} & \dfrac{\partial{F_5}}{\partial{S_w}} & \dfrac{\partial{F_5}}{\partial{S_n}} & \dfrac{\partial{F_5}}{\partial{S_g}} & \dfrac{\partial{F_5}}{\partial{T}}\\[3mm]
\end{pmatrix}
\end{equation}

Тогда

\begin{equation}
\begin{pmatrix}
P_w\\
S_w\\
S_n\\
S_g\\
T
\end{pmatrix}^{new}
=
\begin{pmatrix}
P_w\\
S_w\\
S_n\\
S_g\\
T
\end{pmatrix}
-A^{-1}
\begin{pmatrix}
F_1\\
F_2\\
F_3\\
F_4\\
F_5
\end{pmatrix}
\end{equation}\\

Для обращения матрицы $A$ используется метод Гаусса с~ выбором
главного элемента\cite{Kalitkin}.

%
\section{Тестовые задачи}
%
В соответствии с~ описанной математической моделью были проведены расчеты
нескольких тестовых задач, результаты которых качественно верно описывают
рассматриваемые явления.
Используемые параметры модели описаны
в разделе ~\ref{math_section}. Во~ всех тестовых задачах
пористость и~ абсолютная проницаемость породы постоянны во~ всем объеме,
остаточные насыщенности полагаются равными нулю, размер области 1 м $\times$ 1 м $\times$ 1 м.
\section{Использование высокопроизводительных вычислительных систем}

\subsection{Вычислительные системы}
Одним из наиболее интересных направлений развития современной вычислительной
техники являются многопроцессорные системы. В настоящее время происходит фантастически
быстрый рост мощности
вычислительной техники, что дает реальную возможность
приступить к моделированию тех задач, которые ранее были недоступны для
численного решения. Но, с другой стороны, пришлось столкнуться с той ситуацией,
когда высокопроизводительная многопроцессорная вычислительная техника используется 
лишь на небольшую часть своих потенциальных возможностей. Эта проблема в первую очередь 
связана с трудностями адаптации алгоритмов решения к архитектуре многопроцессорных 
ЭВМ с распределенной памятью. 
 
Многопроцессорные системы можно разделить на 2 класса - системы с общей памятью
и системы с распределенной памятью.
\begin{enumerate}
\item  Системы с общей памятью. С~точки зрения программиста привлекательно
выглядят системы с~общей памятью.
Разбитая на взаимодействующие процессы (нити) программа в~большинстве таких
систем автоматически распределяется по~доступным процессорам системы. Достаточно
широкий круг последовательных алгоритмов может быть успешно адаптирован к~
функционированию на~системах с~общей, или, что тоже самое, разделяемой памятью.
Однако у систем с~общей памятью есть ряд существенных недостатков:
\begin{itemize}
\item относительно небольшое число процессоров;
\item отсутствие возможности наращивания числа процессоров - масштабируемости;
\item пиковая производительность систем с общей памятью ниже пиковой
производительности систем с раздельной памятью;
\item высокая, относительно аналогичных по производительности систем с
раздельной памятью, стоимость.
\end{itemize}
Сказанное выше приводит к тому, что сконструированная система, как правило, не
предусматривает возможности существенного наращивания числа процессорных узлов и
обуславливает крайне высокую, относительно рассматриваемых систем с раздельной
памятью, стоимость.
 
\item Системы с~распределенной памятью. Масштабируемые системы массового параллелизма
конструируют на~основе объединения каналами передачи данных процессорных узлов,
обладающих своей локальной оперативной памятью, недоступной другим процессорам. Обмен данными
между процессорами при таком подходе возможен лишь с помощью сообщений,
передаваемых по каналам связи. Такая схема обладает рядом преимуществ по
сравнению с системами, построенными на основе общей памяти. Подчеркнем основные
преимущества систем с распределенной памятью:
\begin{itemize}
\item сравнительно низкая стоимость;
\item масштабируемость (возможность построения систем требуемой
производительности и наращивания их мощности за счет установки дополнительных
процессоров).
\end{itemize}
Системы с раздельной памятью, по-видимому, всегда будут лидировать по показателю
пиковой производительности, поскольку любые новые однопроцессорные (или
многопроцессорные на основе общей памяти) системы могут быть легко объединены
сетью и использованы в качестве многопроцессорных комплексов с раздельной
памятью.
Но, к сожалению, эффективное использование систем с распределенной памятью
требует значительных усилий со стороны разработчиков прикладного обеспечения и
возможно далеко не для всех типов задач. Для широкого круга хорошо
зарекомендовавших себя последовательных алгоритмов не удается построить
эффективные параллельные аналоги.
\end{enumerate}

\subsection{О технологии CUDA}
В настоящее время графические процессоры (GPU -- Graphics Processing Unit)
являются оптимальной по~соотношению цена-производительность параллельной 
архитектурой с~общей памятью. Причина высокой производительности видеокарт 
в~том, что они изначально были сконструированы для~одновременного применения
одной и~той же шейдерной функции к большому числу пикселей, или, другими 
словами, для~высокопроизводительных параллельных вычислений.

До недавнего времени использовать вычислительные мощности видеокарт было
крайне неудобно, так как программисту необходимо было
овладеть шейдерным языком программирования, соответствующим коду,
выполняемому на GPU. Причем, в графических API полностью отсутствует
возможность взаимодействия между параллельно обрабатываемыми пикселами,
что в графике действительно не нужно, а для вычислительных задач оказывается довольно
желательным. Ситуация изменилась, когда появились средства разработки
GPGPU-приложений(General-Purpose computing on Graphics Processing Units). 
В качестве таких средств выступают CUDA, OpenCL, Dx11 Compute
Shaders. 

Предложенная компанией Nvidia технология CUDA (Compute Unified Device Architecture) -- 
заметно облегчает написание GPGPU-приложений. Данная технология предназначена для
разработки приложений для массивно-параллельных вычислений. Все программы пишутся на
<<расширенном>> языке C, есть хорошая документация, инструменты, набор библиотек,
поддерживается кроссплатформенность. CUDA является полностью бесплатной и уже 
приобрела широкую популярность в~научных кругах~\cite{Boreskov}.

CUDA строится на~концепции, что GPU выступает в~роли массивно-параллельного
сопроцессора к~CPU. При этом последовательный код выполняется на~CPU,
а~для массивно-параллельных вычислений соответствующий код выполняется на~
GPU как набор одновременно выполняющихся нитей(потоков, threads).

Таким образом, GPU рассматривается как специализированное вычислительное устройство,
которое:
\begin{itemize}
  \item является сопроцессором к CPU;
  \item обладает собственной памятью;
  \item обладает возможностью параллельного выполнения огромного количества отдельных
  нитей.
\end{itemize}

В данной работе рассматривается применение технологии CUDA к задачам 
трехфазной фильтрации. Портирование кода происходило на базе
примеров из ~\cite{Sanders-CUDA}.

\subsection{Параллелизм}

В данной работе в~основу параллельной реализации положен принцип геометрического параллелизма.
Геометрический параллелизм является одним из~наиболее распространенных типов
параллелизма, применяемых при решении задач, где~имеется большое число
однородных действий над однородными данными, которые можно так разбить на~
группы, что при обработке каждой группы не потребуется обращений к данным из~
других групп, за~исключением их малого числа.
Последнее свойство будем называть свойством локальности алгоритма. Локальный
алгоритм допускает разбиение данных на~части по~числу процессоров, при~котором
обработку каждой части осуществляет соответствующий процессор. Этот подход
весьма эффективен при~условии, что действия, выполняемые одним процессором,
зависят лишь от~небольшого, ограниченного объема данных, расположенных на~других
процессорах. Желательно, чтобы эти <<чужие>> данные были локализованы на~
ограниченном, небольшом количестве процессоров.
В~настоящее время широкое распространение получил стандарт разработки
параллельных программ MPI (Message Passing Interface - интерфейс передачи
сообщений) для~систем с разделенной памятью. 
В~данной работе он используется для~осуществления передачи
сообщений между процессорами.

\subsection{Практические аспекты реализации программного комплекса}

\subsubsection{Программный комплекс}
Программный комплекс написан на языке
программирования С/C++ с использованием
библиотек MPI и CUDA в~среде Eclipse CDT. Комплекс является
кроссплатформенным и запускается в системах Windows и Linux, что
позволяет использовать его на различных кластерах и пользовательских ПК.

\subsubsection{Индексация массива данных}
Для однородности способа обращения к данным независимо
от размерности задачи используется одномерная индексация массивов.
Например, обращение к элементу $T(i,j,k)$:
$T[i + j * locN_x + k * locN_x * locN_y]$.
$locN_x, locN_y, locN_z$ -- размеры расчетной сетки, обрабатываемой
данным процессором. Продемонстируем универсальность данного
подхода. Допустим, отсутствует измерение, соответвующее $z$, тогда
$locN_z=1$, $k$ <<пробегает>> значения от $0$ до $locN_z-1$, т.е.
в данном случае единственное возможное значение $k=0$, при этом
обращение к элементу становится идентичным $T[i + j * locN_x]$, т.е.
двумерной индексации. Таким образом, учет размерности задачи
происходит автоматически. Однако, в коде при вычислении,
например, градиента функции следует все же проверять, что $locN_z > 2$
(если используется трехточечный шаблон),
прежде чем проводить вычисления.

\subsubsection{Сетка из процессоров}
Расчетная область делится на подобласти в~одном, двух или трех направлениях.
Можно сказать, что процесоры образуют сетку, в~которой
местоположение каждого процессора может быть задано с~помощью
декартовых координат.
Для наглядности приведем пример такого деления на~Рис.~\ref{pic_div}.
\begin{figure}[!h]\center
\includegraphics[width=1\textwidth]{div.png} 
\caption{Деление расчетной области на подобласти в одном, двух или трех направлениях}
\label{pic_div}
\end{figure}
Дополнительно к~стандартным для~MPI понятиям $size$, $rank$
каждым из процессоров запоминаются значения $size_x$, $size_y$, $size_z$,
$rank_x$, $rank_y$, $rank_z$ (аналоги $size$, $rank$ отдельно для каждого из направлений).
$size_x$, $size_y$, $size_z$ определяются алгоритмом деления,
причем $size_x \cdot size_y \cdot size_z = size$.
Ранги определяются в Листинге~\ref{lst_ranks}, причем
$rank_x + rank_y \cdot size_x + rank_z \cdot size_x \cdot size_y = rank$.
\begin{listing}[!h]
\begin{minted}[bgcolor=bg, fontfamily=helvetica]{c}
rankx = rank % sizex;
ranky = (rank / sizex) % sizey;
rankz = (rank / sizex) / sizey;
\end{minted}
\caption{Определение координат процессора}
\label{lst_ranks}
\end{listing}
Данные обозначения вводятся для удобства реализации обменов
данными и~обращения к~расчетной области.

\subsubsection{Алгоритм первоначального распределения области по процессорам}

Для того чтобы предварительно оценить время, необходимое для расчета, выяснить, насколько 
рационально деление области по~различным измерениям, и~эффективно распределить нагрузку
между вычислительными узлами, был разработан и~встроен в~основной код программы
алгоритм первоначального распределения области по~процессорам.

В случае многопроцессорной реализации в~общем времени вычислений можно выделить
три основных составляющих: время непосредственно численного расчета, время
межпроцессорных обменов и~время загрузки/выгрузки данных на~графический ускоритель 
(в~случае гибридных вычислительных систем). Доля, которую вносит каждое из~
слагаемых, зависит как от~вычислительной системы, так и~от~выбора конфигурации разбиения
области. Чем больше направлений разбиения, тем меньше граничных узлов, но
больше количество обменов на границах.

Предлагаемый алгоритм автоматического выбора оптимального разбиения состоит
из трех основных шагов:
\begin{enumerate}
 \item Найти эмпирически время, идущее на вычисления в одной расчетной точке на~
одном временном слое задачи. Следует заметить, что нахождение времени расчета
аналитически в~одной точке не является тривиальной задачей, а экспериментально засечь
время расчета можно только для~некоторого множества точек на~нескольких слоях по~
времени. Для~разного числа точек получаем набор времен и~строим зависимость времени
расчета некоторого числа точек от~их количества. Для~явных схем эту зависимость
можно приближенно считать линейной. В этом случае имеет смысл говорить о~времени
расчета одной точки. Для~неявных схем зависимость нелинейна, с увеличением числа
точек растет время, условно приходящееся на расчет значения в~одной точке.
\item Найти эмпирически время, идущее на коммуникации. Для нахождения времени
пересылки между CPU, принадлежащими разным узлам кластера, измеряются времена
пересылок для~различного объема данных. Предполагается, что время пересылки
состоит из~времени инициализации соединения и времени непосредственно 
отправки-получения данных, которое линейно зависит от объема данных. Из~полученных 
измерений можно найти данную линейную зависимость. Время копирования данных из~памяти
CPU в~память GPU определяется аналогичным способом.
\item Решить задачу поиска минимума общего времени вычислений при ограниченном
числе доступных вычислителей. Эту задачу можно решить, перебрав в цикле все возможные
способы разбиения области при заданном количестве расчетных точек.
\end{enumerate}
Рассмотрим пример применения данного алгоритма. Допустим, что нужно произвести
расчеты на трехмерной сетке размером $500\times100\times50=2.5$ млн точек, при этом в~
распоряжении пользователя имеется 24 ядра CPU. Можно оценить, насколько сильно
меняется общее время вычислений в зависимости от того, каким образом распределена
расчетная сетка. На Рис.~\ref{diagr_div} представлены результаты измерений полученных времен 
расчета некоторой тестовой задачи трехфазного просачивания для различных конфигураций разбиения.
Видно, что времена вычислений могут отличаться как на пять процентов,
так и в полтора раза, что может быть очень существенно.

\begin{figure}[!h]
\begin{center}
\begin{tikzpicture}
  \begin{axis}[ymax=35, ymin=0, ybar, enlarge x limits=true, width=0.6\textwidth, x label style={below=10mm},
    xlabel={конфигурация $(size_x\ size_y \ size_z)$}, ylabel={время расчета, с},
    symbolic x coords={(3 4 2),(24 1 1),(12 2 1),(6 4 1),(8 2 2),(4 3 2),(2 6 2)},
    xtick=data,nodes near coords, nodes near coords align={vertical}, x tick label style={rotate=45,anchor=east},]
     \addplot coordinates {((3 4 2),19.8) ((24 1 1),29.7) ((12 2 1),26.55) ((6 4 1),20.81) ((8 2 2),20.3) ((4 3 2),19.98) ((2 6 2),19.91)};
  \end{axis}
\end{tikzpicture}
\caption{Времена расчета тестовой задачи для различных конфигураций для 24 процессоров}
\label{diagr_div}
\end{center}
\end{figure}

\subsubsection{Локальная и глобальная системы координат}
В работе представлен гибкий и хорошо масштабируемый механизм разделения
расчетной сетки на локальные подсетки. Благодаря такому механизму
нет необходимости хранить расчетную сетку целиком на каком-либо из
процессоров, а преобразование локальных координат в глобальные
позволяет воссоздать общую картину полученных в результате расчетов
данных.
Используем следующие обозначения:
$locN_x, locN_y, locN_z$ -- размеры расчетной (локальной) сетки, обрабатываемой
данным процессором, $N_x, N_y, N_z$ -- размеры общей (глобальной) сетки.
Разделение глобальной сетки на локальные происходит с~помощью
алгоритма, описанного в Листинге~\ref{lst_globallocal}. 
Если сетка по выделенному измерению поровну не делится, то процессоры с $rank_x < N_x mod size_x$ получают нагрузку
на один слой больше. Каждый процессор содержит дополнительный слой в массиве для обмена данными, при 
наличии <<соседа>> (если есть <<соседи>> с обеих сторон -- два дополнительных слоя).
Глобальные границы хранятся как обычные точки.

Переход от локальной
системы координат к глобальной представлен в Листинге~\ref{lst_localglobal}.

\begin{listing}[!h]
\begin{minted}[bgcolor=bg, fontfamily=helvetica]{c}
locNx = Nx / sizex;
if (rankx < Nx % sizex) locNx++;
if (sizex > 1) {
    if ((rankx == 0) || (rankx == sizex - 1)) locNx++;
    else locNx += 2;
}
locNy = Ny / sizey;
if ((ranky < Ny % sizey)) locNy++;
if (sizey > 1) {
    if ((ranky == 0) || (ranky == sizey - 1)) locNy++;
    else locNy += 2;
}
locNz = Nz / sizez;
if (rankz < Nz % sizez) locNz++;
if (sizez > 1) {
    if ((rankz == 0) || (rankz == sizez - 1)) locNz++;
    else locNz += 2;
}
\end{minted}
\caption{Разделение расчетной сетки на подсетки}
\label{lst_globallocal}
\end{listing}

\begin{listing}[!h]
\begin{minted}[bgcolor=bg, fontfamily=helvetica]{c}
int global_index = local_index;
switch (axis)
{
case 'x':
    global_index += rankx * (Nx / sizex) + min(rankx, Nx % sizex) - min(rankx, 1);
    break;
case 'y':
    global_index += ranky * (Ny / sizey) + min(ranky, Ny % sizey) - min(ranky, 1);
    break;
case 'z':
    global_index += rankz * (Nz / sizez) + min(rankz, Nz % sizez) - min(rankz, 1);
    break;
default:
    print_error();
}
return global_index;
\end{minted}
\caption{Переход от локальной системы координат к глобальной}
\label{lst_localglobal}
\end{listing}

\subsubsection{Организация межпроцессорных обменов}

В первую очередь граничные данные для обмена помещаются в~
буферный массив. Затем посредством сообщений процессоры
обмениваются содержимым буферов. Процессоры условно делятся
на четные и нечетные отдельно по всем трем направлениям пересылки.
По каждому направлению процессоры обмениваются в~два этапа:
от четных к нечетным и от нечетных к четным. Таким образом,
полный обмен граничными данными со всеми <<соседями>> происходит
за шесть итераций. Реализация обменов описанным способом 
приводится в Листингах~\ref{lst_exchange}, \ref{lst_exchanges}. 
Обмен завершается копированием
полученных данных из буфера в~локальную расчетную область процессора.

При проведении расчетов на GPU к обмену сообщениями и~использованию
буфера на хосте
добавляется копирование данных для обмена из памяти видеокарты
в~память хоста.

\begin{listing}[!h]
\begin{minted}[bgcolor=bg, fontfamily=helvetica]{c}
void send_recv(double* arr, char axis, double* buf, int size, int dst_rank, int id) {
  MPI_Status status;
  load_exchange_data_part(arr, axis);
  MPI_Sendrecv_replace(buf, size, MPI_DOUBLE, dst_rank, id, dst_rank, id + 1,
		       MPI_COMM_WORLD, &status);
  save_exchange_data_part(arr, axis);
}
void recv_send(double* arr, char axis, double* buf, int size, int dst_rank, int id) {
  MPI_Status status;
  load_exchange_data_part(arr, axis);
  MPI_Sendrecv_replace(buf, size, MPI_DOUBLE, dst_rank, id + 1, dst_rank, id,
		       MPI_COMM_WORLD, &status);
  save_exchange_data_part(arr, axis);
}
\end{minted}
\caption{Реализация межпроцессорного двустороннего обмена}
\label{lst_exchange}
\end{listing}
\begin{listing}[!h]
\begin{minted}[bgcolor=bg, fontfamily=helvetica]{c}
void exchange_direct(double* arr, char axis, double* buf) {
  switch(axis) {
  case 'x': if(sizex > 1) {
    if (rankx % 2 == 0) {
      if (rankx != sizex - 1) send_recv(arr, axis, buf, locNy * locNz, rank + 1, 500);
      if (rankx != 0) recv_send(arr, axis, buf, locNy * locNz, rank - 1, 502);
    } else {
      if (rankx != 0) recv_send(arr, axis, buf, locNy * locNz, rank - 1, 500);
      if (rank) != sizex - 1) send_recv(arr, axis, buf, locNy * locNz, rank + 1, 502);
    }
  } break;
  case 'y': if(sizey > 1) {
    if (ranky % 2 == 0) {
      if (ranky != sizey - 1) send_recv(arr, axis, buf, locNx * locNz, rank + sizex, 504);
      if (ranky != 0) recv_send(arr, axis, buf, locNx * locNz, rank - sizex, 506);
    } else {
      if (ranky != 0) recv_send(arr, axis, buf, locNx * locNz, rank - sizex, 504);
      if (ranky != sizey - 1) send_recv(arr, axis, buf, locNx * locNz, rank + sizex, 506);
    }
  } break;
  case 'z': if(sizez > 1) {
    if (rankz % 2 == 0) {
      if (rankz != sizez - 1)
	send_recv(arr, axis, buf, locNx * locNy, rank + sizex * sizey, 508);
      if (rankz != 0) recv_send(arr, axis, buf, locNx * locNy, rank - sizex * sizey, 510);
    } else {
      if (rankz != 0) recv_send(arr, axis, buf, locNx * locNy, rank - sizex * sizey, 508);
      if (rankz != sizez - 1)
	send_recv(arr, axis, buf, locNx * locNy, rank + sizex * sizey, 510);
    }
  } break;
  default: break;
  }
}
\end{minted}
\caption{Организация межпроцессорных пересылок по всем направлениям}
\label{lst_exchanges}
\end{listing}

\subsubsection{Понятие активной точки}
По способу обработки узлы расчетной сети условно можно
разделить на:
\begin{itemize}
 \item активные (значения в которых вычисляются текущим процессором);
 \item вспомогательные (служащие для хранения данных, полученных от
 других процессоров).
\end{itemize}
Для определения, к какому из типов относится данный узел,
используется код, представленный в Листинге~\ref{lst_act_point}.
\begin{listing}[!h]
\begin{minted}[bgcolor=bg, fontfamily=helvetica]{c}
if (((rankx != 0 && i == 0) || (rankx != sizex - 1 && i == locNx - 1 && Nx > 1))
  || ((ranky != 0 && j == 0) || (ranky != sizey - 1 && j == locNy - 1 && Ny > 1))
  || ((rankz != 0 && k == 0) || (rankz != sizez - 1 && k == locNz - 1 && Nz > 1)))
    return 0;
else
    return 1;
\end{minted}
\caption{Определение типа расчетного узла}
\label{lst_act_point}
\end{listing}

\subsubsection{Ввод/Вывод данных}
Сохранение данных занимает важное место в моделировании
физических процессов. Как отмечено в \cite{Yakobovsky},
ввод и вывод исходных данных с внешних дисковых накопителей может осуществляться, как минимум,
двумя разными способами:
\begin{itemize}
\item ввод/вывод данных выполняет один из процессоров, называемый в
дальнейшем <<управляющий>>;
\item ввод/вывод данных выполняется непосредственно каждым процессором.
\end{itemize}
Несмотря на большую универсальность первого способа,
выбран второй способ, так как он является более подходящим
в случае обработки данных, объем которых превышает оперативную память управляющего
процессорного узла.
В этом случае возможны также, по крайней мере, два варианта:
\begin{itemize}
\item запись всеми процессорами одного большого файла -- каждый процессор
пишет свой фрагмент этого файла;
\item запись каждым процессором отдельного файла -- общее число файлов
при этом соответствует числу процессоров.
\end{itemize}
Для лучшей масштабируемости системы, удобства визуализации и обращения
к данным выбран первый способ,
но при этом возникает проблема синхронизации доступа к файлу.
Приведем пример кода, обеспечивающего корректную запись данных в файл
многими процессорами в Листинге~\ref{lst_print_plot}.
\begin{listing}[!h]
\begin{minted}[bgcolor=bg, fontfamily=helvetica]{c}
for (int cpux = 0; cpux < sizex; cpux ++)
    for (int i = 0; i < Nx / sizex + 3; i++)
	for (int cpuy = 0; cpuy < sizey; cpuy ++)
	    for (int j = 0; j < def.Ny / sizey + 3; j++)
		for (int cpuz = 0; cpuz < sizez; cpuz ++)
		{
		    barrier();
		    if ((rank == cpux + cpuy * sizex + cpuz * sizex * sizey)
			  && (i < locNx) && (j < locNy))
		    {
			for (int k = 0; k < locNz; k++)
			    if (is_active_point(i, j, k))
				print_plot_row(i, j, k);
		    }
		}
\end{minted}
\caption{Синхронизация вывода в файл между процессорами}
\label{lst_print_plot}
\end{listing}

\subsection{Результаты расчетов на кластере}

Основные расчеты проводились на кластере К-100 ИПМ им. М.В.Келдыша РАН, 
структура которого изображена на Рис~\ref{pic_k100}.
Ускорения и эффективность вычислений
были подробно проанализированы на примере трехмерной тестовой задачи, описанной в~соответствующем
разделе. Время расчета измерялось с помощью счетчика, встроенного в программный код.

\begin{figure}[!h]\center
\includegraphics[width=0.8\textwidth]{k100.pdf} 
\caption{Структура гибридного кластера К100}
\label{pic_k100}
\end{figure}

\subsubsection{Результаты применения библиотеки MPI}

Исследуем ускорения и эффективности вычислений в зависимости от числа
используемых для расчета процессоров. Результаты измерений приведены
на графиках (Рис.~\ref{mpi_speedup}, Рис.~\ref{mpi_eff}).
Измерялось время расчета 50 шагов по времени на сетке размером 4 миллиона узлов
с помощью счетчика, встроенного в программный код.
Расчетная область может быть поделена между CPU геометрически как в одном, так и в двух или трех направлениях.
Ступенчатая структура ускорений и, следовательно, скачки эффективности наблюдаются
в связи с неодинаковым способом распределения нагрузки между процессорами
в зависимости от их числа, а также погрешностью вычислений. Сохранение высокой
эффективности вычислений (84-95\%) обусловлено значительным превосходством времени, требуемого
непосредственно для расчета, по~сравнению с временем, необходимым для обмена данными
между процессорами.

\begin{figure}[!h]
\begin{center}
\begin{tikzpicture}
  \begin{axis}[axis lines=left, enlargelimits=true, grid=major, width=0.7\textwidth, xlabel={\textit{число процессоров}}, ylabel={\textit{ускорение}}]
    \pgfplotstableread{../data/mpi_times.txt}{\mytable}
    \pgfplotstablegetelem{0}{Y}\of{\mytable}
    \pgfmathsetmacro{\ay}{\pgfplotsretval}
    \addplot [blue, mark=*] table [x=X, y expr=\ay/\thisrow{Y}] {\mytable};
  \end{axis}
\end{tikzpicture}
\caption{Зависимость ускорения расчета тестовой задачи от числа процессоров}
\label{mpi_speedup}
\end{center}
\end{figure}

\begin{figure}[!h]
\begin{center}
\begin{tikzpicture}
  \begin{axis}[axis lines=left, enlargelimits=true, grid=major, width=0.7\textwidth, xlabel={\textit{число процессоров}}, ylabel={\textit{эффективность}}]
    \pgfplotstableread{../data/mpi_times.txt}{\mytable}
    \pgfplotstablegetelem{0}{Y}\of{\mytable}
    \pgfmathsetmacro{\ay}{\pgfplotsretval}
    \addplot [blue,mark=*] table [x=X, y expr=\ay/\thisrow{Y}/\thisrow{X}] {\mytable};
  \end{axis}
\end{tikzpicture}
\caption{Зависимость эффективности расчета тестовой задачи от числа процессоров}
\label{mpi_eff}
\end{center}
\end{figure}


\subsubsection{Результаты использования технологии CUDA}

Рассмотрим, как меняется время решения поставленной задачи при~проведении расчетов на~GPU
с~помощью CUDA. На~Рис.~\ref{cuda_speedup} изображена зависимость укорения тестового расчета на~одном GPU
по~сравнению с~одним CPU. Измерения проводились для~различного числа
узлов расчетной сетки. Как видно из~приведенного на~Рис.~\ref{cuda_speedup}
графика, для~сеток малого размера (8000 узлов) возможно замедление расчетов, но~при~увеличении
размера сетки наблюдается все большее ускорение расчета. При~этом для~большей из~
рассмотренных сеток (2.2 миллиона узлов) полученное ускорение составило 3.5 раза. Небольшое
ускорение обусловлено большой сложностью и~вложенностью вычислений для~каждого из~расчетных
узлов, так~как для GPU доступ к памяти является трудоемкой задачей, а~размер кеша мал. Вероятно,
путем реогранизации программного кода можно достигнуть больших ускорений, но~тогда усложнилась
бы поддержка программы, так как стало бы больше отличий между кодом, написанным для~СPU, и~кодом
для~GPU, а~также уменьшилась бы~читаемость кода программного комплекса.

\begin{figure}[!h]
\begin{center}
\begin{tikzpicture}
  \begin{axis}[axis lines=left, enlargelimits=true, grid=major, width=0.7\textwidth, xlabel={\textit{число узлов сетки}}, ylabel={\textit{ускорение}}]
    \pgfplotstableread[skip first n=1]{../data/cpu_vs_gpu.txt}{\mytable}
    \addplot [blue, mark=*] table [x=0, y expr=\thisrow{1}/\thisrow{2}] {\mytable};
  \end{axis}
\end{tikzpicture}
\caption{Зависимость ускорения расчета тестовой задачи от числа узлов расчетной сетки на одном GPU по сравнению с одним CPU}
\label{cuda_speedup}
\end{center}
\end{figure}

\subsubsection{Результаты совместного применения библиотек CUDA и MPI}

Покажем на~Рис.~\ref{many_gpu}, как для~расчетной сетки размером 8 млн узлов меняется время вычислений
при~использовании сразу нескольких видеокарт (от~трех до~двенадцати). Ускорение расчитывается по~отношению
к~расчету на~трех видеокартах.
Выбрано именно три видеокарты, так как на~каждом из~узлов кластера расположено три GPU. Из приведенного графика можно видеть,
что эффективность вычислений ниже, чем при использовании только MPI (см. Рис.~\ref{mpi_speedup}). Этот эффект можно
объяснить тем, что при обмене данными между процессорами необходимо провести копирование значений в~необходимых
узлах сетки не только в~буфер CPU, но~и~в~буфер GPU (для~копирования между CPU и~GPU) для~дальнейшего обмена между CPU.

\begin{figure}[!h]
\begin{center}
\begin{tikzpicture}
  \begin{axis}[axis lines=left, enlargelimits=true, grid=major, width=0.7\textwidth, xlabel={\textit{число используемых GPU}}, ylabel={\textit{ускорение}}]
    \pgfplotstableread[skip first n=1]{../data/many_gpu.txt}{\mytable}
    \pgfplotstablegetelem{0}{1}\of{\mytable}
    \pgfmathsetmacro{\ay}{\pgfplotsretval}    
    \addplot [blue, mark=*] table [x=0, y expr=\ay/\thisrow{1}] {\mytable};
  \end{axis}
\end{tikzpicture}
\caption{Зависимость ускорения расчета тестовой задачи при использовании различного числа GPU по отношению к расчету на 3 GPU}
\label{many_gpu}
\end{center}
\end{figure}

В заключение следует отметить, что~эффективное использование больших вычислительных
мощностей является трудоемкой задачей и~требует глубокого исследования и~понимания
архитектуры вычислительных систем.

\section{Заключение}
Результаты расчетов показывают, что, используя предложенную модификацию модели, удается существенно ускорить
численное решение, а также сделать параллельные вычисления удобнее.При этом погрешность вычислений остается
приемлемой. В частности, на примере рассмотренной тестовой задачи удалось снизить на порядок ограничение на шаг по времени
при расчетах явными методами.

\makeatletter
\renewcommand{\@biblabel}[1]{#1.\hfil}
\makeatother
\addto\captionsrussian{\def\refname{Список литературы}}
\addcontentsline{toc}{section}{Список литературы}
\nocite{*}
%\bibliographystyle{gost71u}%правильные ссылки по ГОСТ 7.1-84 7.80-00
\bibliographystyle{utf8gost705u} % Оформляем библиографию в соответствии с ГОСТ 7.0.5
\bibliography{kbib-utf8}


\end{document}
