\section{Физическая модель}

Законы движения флюидов в пористых средах базируются на сохранении
массы, энергии, импульса. Основные
физические свойства пористой среды связаны с элементарным
объемом среды, который должен быть достаточно большим по сравнению с размером
пор. Подземные пространства характеризуются сложной системой пор, каналов,
трещин, размеры которых малы по сравнению с характерными размерами среды, и по
которым может происходить течение жидкости или газа. Количественной
характеристикой пористости среды
служит отношение объема пор к общему объему:
%
	$$m=V_n/V,$$
%	 	
где $m$ -- коэффициент пористости, $V_n$ -- объем пор, $V$ -- общий объем
данного
элемента среды.
%
Течение через такие пористые тела, при котором сила трения флюида
(жидкости, газа) о скелет играет определяющую роль, называется фильтрацией.
Часть системы, все компоненты которой имеют
одинаковые физические и химические свойства, называется фазой. 

Главными характеристиками движения многофазной системы являются насыщенности и
скорости фильтрации каждой из фаз. Насыщенностью $S_i$  порового пространства
$i$-ой фазы называется доля объема пор, занятая этой фазой в элементарном
объеме:
%
\begin{equation} 
S_i=\frac{\Delta V_i}{\Delta V}, i=1,2\ldots n,{\quad}\sum_{i=1}^{n}S_i=1. 
\end{equation}
%
Для каждой фазы существует предельная насыщенность, такая, что при меньших
значениях насыщенности эта фаза неподвижна. Эти значения носят название остаточных 
насыщенностей. Обозначим их с помощью нижнего индекса $r$ у насыщенностей. Таким 
образом, совместное течение двух фаз имеет место лишь в интервале насыщенностей.
Для работы в данном интервале насыщенностей вводится понятие эффективных 
насыщенностей фаз, их обозначаем с помощью символа верхней черты. Эффективная насыщенность 
$i$-ой фазы может быть представлена в виде:
$$\overline{S_i}={\frac{S_i-S_{ir}}{1-\sum\limits_{k=1}^{n}S_{kr}}}.$$

Проекция скорости фильтрации $i$-ой фазы в некоторой точке
$\overrightarrow{u_i}$ на некоторое направление равна отношению объемного
расхода данной фазы к площадке, перпендикулярной к указанному направлению.
Основное соотношение теории фильтрации -- закон фильтрации(закон Дарси) -- устанавливает 
связь между вектором скорости фильтрации и тем полем давления, которое вызывает 
фильтрационное течение. Закон Дарси в теории фильтрации заменяет собой уравнение 
движения. Для $i$-ой фазы при учете силы тяжести его можно записать в виде:
\begin{equation}
\label{Darcy}
  \overrightarrow{u_i}=-K \frac{k_i(S_i)}{{\mu}_i}(grad P_i - {\rho}_i\overrightarrow{g}),
\end{equation}
где $K$ -- характеристика пористой среды, называемая
абсолютной проницаемостью, определяемая по~данным о~фильтрации однородной
жидкости и~не~зависящая от~свойств жидкости; $\mu_i$ -- коэффициент динамической
вязкости $i$-ой фазы ($\mu_i=\mu_i(T)$); $k_i(S_i)$ -- относительная фазовая проницаемость $i$-ой фазы(может зависеть 
не только от насыщенности $i$-ой фазы), определяемая
экспериментально; $P_i$ -- давление в $i$-ой фазе; ${\rho}_i$ -- плотность $i$-ой фазы;
$\overrightarrow{g}$ -- вектор ускорения свободного падения.
Можно выделить верхнюю и нижнюю границы применимости закона Дарси\cite{Aziz-Settari}. Верхняя граница связана 
с~проявлением инерционных сил при~достаточно высоких скоростях фильтрации. Нижняя --
с~взаимодействием жидкости с~твердым скелетом пористой среды при достаточно малых 
скоростях фильтрации.

Еще одно фундаментальное соотношение в теории фильтрации - уравнение неразрывности. 
Для $i$-ой фазы оно принимает вид:
 \begin{equation}
 \label{mass}
 	 \frac{\partial (m \rho_i S_i)}{\partial t}+ div(\rho_i \overrightarrow{u_i}) = \rho_i q_i.
 \end{equation}
В отсутствие объемных источников $i$-ой фазы
соответсвующее слагаемое в правой части уравнения (${\rho_i}q_i$) обнуляется.

При малых размерах области фильтрации и малых скоростях капиллярные силы могут
превзойти внешний перепад давления, и их необходимо учитывать.
Капиллярные эффекты обусловлены межмолекулярными взаимодействиями двух различных
фаз. Эти силы приводят к~появлению угла смачивания на границе раздела двух фаз и~
к~разрыву давления на~этой границе. Разность фазовых давлений есть так
называемое \textit{капиллярное давление}. Соотношения между капиллярными
давлениями и насыщенностями обычно получают по~опытным данным как функции насыщенностей.

При неизотермической фильтрации свойства и~состав флюидов зависят еще от~одной переменной -- температуры.
Поэтому для~замыкания системы уравнений фильтрации необходимо дополнительное уравнение --
закон сохранения энергии. Для~многофазной системы без~учета химических реакций закон сохранения энергии 
может быть записан в~виде
\begin{equation}
\label{Energy_law}
  \frac{\partial \left(m {\sum\limits_{i}{\rho_i S_i E_i}} + (1-m){\rho_r E_r}\right)}{\partial t}
    + div(\sum_{i}{\rho_i H_i \overrightarrow{u_i}}) = div(\lambda_{eff} grad T),
\end{equation}
где суммирование производится по~всем активным фазам, индекс $r$ обозначает твердую породу,
$\lambda_{eff}$ -- эффективный коэффициент теплопроводности:
\begin{equation}
\lambda_{eff}=m\sum_i{S_i\lambda_i} + (1-m)\lambda_r,
\end{equation}
связь между внутренней энергией $E_i$ и энтальпией $H_i$:
\begin{equation}
E_i=H_i-\frac{P_i}{\rho_i},
\end{equation} для~нахождения энтальпии необходимо определить зависимость теплоемкости
вещества при~постоянном давлении от~температуры:
\begin{equation}
H_i=H_{i0}+\int_{T_0}^{T}{C_P(T)dT}.
\end{equation}
Зависимости коэффициентов теплопроводности и теплоемкостей находятся эмпирически.

Для замыкания системы дополнительно вводятся уравнения состояния рассматриваемого флюида
и пористой среды.
