\section{Введение}
Теория фильтрации, изучающая законы движения жидкостей, газов и~их~смесей в~
пористой среде, имеет обширное практическое применение. На~протяжении многих лет
фильтрационные расчеты занимают очень важное место при~разработке технологий
добычи нефти и~газа, при~проектировании, постройке и эксплуатации
гидротехнических  и~мелиоративных сооружений, в~горном деле, в~решении
экологических проблем.  Модели неизотермической фильтрации используются для~описания
процессов извлечения нефти
с~применением различных теплоносителей (горячей воды, пара, очага внутрипластового
горения). Эти технологии используются, в~основном, для добычи высоковязких нефтей и~битумов ~\cite{Kanevskaya}.
В~связи с~растущим потреблением нефти и~нефтепродуктов, постепенным истощением ранее разведанных
нефтяных месторождений, сверхвязкие тяжелые нефти становятся особенно востребованными в~экономике. 
Такие нефти активно применяются в строительстве, а~после очистки
их~можно использовать в~химической промышленности -- для~производства клеев и~пластиков.
Поэтому задачи неизотермической фильтрации
становятся все более и~более актуальными.

Настоящая работа посвящена моделированию проблем, связанных с~трехфазной фильтрацией 
в~однородной пористой среде с~учетом тепловых процессов, слабой сжимаемости жидкостей
и~капиллярных сил.
Три рассматриваемые фазы: вода, газ и~NAPL(Non-Aqueous Phase 
Liquids). К~NAPL, например, относятся нефти, битумы, бензин, тетрахлороэтилен, 
минеральное топливо, растворители, очищающие средства. 
Эти жидкости не смешиваются с~водой и с~газом,
поэтому при~моделировании их течения
в~почве говорят о многофазной фильтрации.
Результатом моделирования в~конечном счете являются распределения насыщенностей, давлений
трех фаз и~температуры в~пласте среды в~зависимости от~времени, начальных и~граничных
условий, свойств пористой среды. 
Численные решения таких задач, также как и~других задач газо- и~гидродинамики,
требуют больших вычислительных затрат. В~то же время существует необходимость в~
наиболее быстром получении результатов. Использование
высокопроизводительных вычислительных систем с~распределенной памятью
позволяет проводить вычисления в~течение разумного времени. Поэтому созданные
вычислительные алгоритмы были адаптированы к~расчетам на~многопроцессорных
вычислительных комплексах. Для~параллельных вычислений используются стандартная
библиотека MPI и~технология Nvidia CUDA. При~расчетах применяются явные разностные 
схемы, допускающие эффективное распараллеливание. Вычисления проводятся
на~гибридном вычислительном кластере K-100 ИПМ им.~М.В.Келдыша РАН.
