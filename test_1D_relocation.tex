\subsection{Задача о перераспределении фаз под действием силы тяжести}
Рассмотрим одномерную модельную задачу фильтрации. Пористость и~
абсолютная проницаемость породы постоянны во~ всем объеме:  $m$=0.4,\; $K=6.64\cdot 10^{-11}$ м$^2$.
Начальные условия:\; $S_w=0.4$,\; $S_n=0.3$,\; $S_g=0.3$, 
$P_w=P_\text{атм}+\rho g h$($h$ - глубина, отсчитывается от~ верхнего края области);
$T=285K$. Граничные условия:

$\left.T\right|_{x=0}=320K, \left.T\right|_{x=1}=285K;$
$\left.P\right|_{x=0}=P_{\text{атм}},\quad \Biggl.\dfrac{\partial{P_w}}{\partial{x}}\Biggr|_{x=1}=0;$
$\left.S_w\right|_{x=0}=0.01,\quad \Biggl.\dfrac{\partial{S_w}}{\partial{x}}\Biggr|_{x=1}=0;$
$\left.S_n\right|_{x=0}=0.01,\quad \Biggl.\dfrac{\partial{S_n}}{\partial{x}}\Biggr|_{x=1}=0;$

Результаты расчетов представлены в~ виде зависимостей профилей насыщенностей фаз,
распределения давления и~ температуры от~ времени.

По~ оси $x$ отложено расстояние по вертикали от дна резервуара. По~ оси $y$--
значения насыщенностей каждой из~ фаз на~ расстоянии $x$ от~ дна резервуара
в~ определенный момент расчетного времени.

\begin{figure}
\begin{center}
\begin{minipage}[h]{0.49\textwidth}
\begin{tikzpicture}
  \begin{axis}[legend entries={$S_w$,$S_n$,$S_g$}, axis lines=left, ymax=1, ymin=0, enlargelimits=true, grid=major, width=1\textwidth, xlabel={$x$, м}, ylabel={$S$}]
    \pgfplotstableread[skip first n=3]{data/test2/F-000000004000.tec}{\mytable}
    \addplot [blue, ultra thick] table [x expr=1-\thisrow{0}, y=1] {\mytable};
    \addplot [black, ultra thick] table [x expr=1-\thisrow{0}, y=2] {\mytable};
    \addplot [red, ultra thick] table [x expr=1-\thisrow{0}, y=3] {\mytable};
  \end{axis}
\end{tikzpicture}
\caption{Насыщенности в момент времени $t=4000$с}
\end{minipage}
\hfill
\begin{minipage}[h]{0.49\textwidth}
\begin{tikzpicture}
  \begin{axis}[legend entries={$S_w$,$S_n$,$S_g$}, axis lines=left, ymax=1, ymin=0, enlargelimits=true, grid=major, width=1\textwidth, xlabel={$x$, м}, ylabel={$S$}]
    \pgfplotstableread[skip first n=3]{data/test2/F-000000010000.tec}{\mytable}
    \addplot [blue, ultra thick] table [x expr=1-\thisrow{0}, y=1] {\mytable};
    \addplot [black, ultra thick] table [x expr=1-\thisrow{0}, y=2] {\mytable};
    \addplot [red, ultra thick] table [x expr=1-\thisrow{0}, y=3] {\mytable};
  \end{axis}
\end{tikzpicture}
\caption{Насыщенности в момент времени $t=10000$с}
\end{minipage}
\vfill
\begin{minipage}[h]{0.49\textwidth}
\begin{tikzpicture}
  \begin{axis}[axis lines=left, ymax=105000, ymin=100000, enlargelimits=true, grid=major, width=1\textwidth, xlabel={$x$, м}, ylabel={$P$, Па}]
    \pgfplotstableread[skip first n=3]{data/test2/F-000000004000.tec}{\mytable}
    \addplot [blue, ultra thick] table [x expr=1-\thisrow{0}, y=4] {\mytable};
  \end{axis}
\end{tikzpicture}
\caption{Давление $P_w$ в момент времени $t=4000$с}
\end{minipage}
\hfill
\begin{minipage}[h]{0.49\textwidth}
\begin{tikzpicture}
  \begin{axis}[axis lines=left, ymax=105000, ymin=100000, enlargelimits=true, grid=major, width=1\textwidth, xlabel={$x$, м}, ylabel={$P$, Па}]
    \pgfplotstableread[skip first n=3]{data/test2/F-000000010000.tec}{\mytable}
    \addplot [blue, ultra thick] table [x expr=1-\thisrow{0}, y=4] {\mytable};
  \end{axis}
\end{tikzpicture}
\caption{Давление $P_w$ в момент времени $t=10000$с}
\end{minipage}
\vfill
\begin{minipage}[h]{0.49\textwidth}
\begin{tikzpicture}
  \begin{axis}[axis lines=left, ymax=320, ymin=285, enlargelimits=true, grid=major, width=1\textwidth, xlabel={$x$, м}, ylabel={$T$, К}]
    \pgfplotstableread[skip first n=3]{data/test2/F-000000004000.tec}{\mytable}
    \addplot [blue, ultra thick] table [x expr=1-\thisrow{0}, y=5] {\mytable};
  \end{axis}
\end{tikzpicture}
\caption{Температура в момент времени $t=4000$с}
\end{minipage}
\hfill
\begin{minipage}[h]{0.49\textwidth}
\begin{tikzpicture}
  \begin{axis}[axis lines=left, ymax=320, ymin=285, enlargelimits=true, grid=major, width=1\textwidth, xlabel={$x$, м}, ylabel={$T$, К}]
    \pgfplotstableread[skip first n=3]{data/test2/F-000000010000.tec}{\mytable}
    \addplot [blue, ultra thick] table [x expr=1-\thisrow{0}, y=5] {\mytable};
  \end{axis}
\end{tikzpicture}
\caption{Температура в момент времени $t=10000$с}
\end{minipage}
\end{center}
\end{figure}
