%
\section{Результаты расчетов тестовых задач}
%
В соответствии с описанной математической моделью были проведены расчеты
нескольких тестовых задач, результаты которых качественно верно описывают
рассматриваемые явления. Проверить согласованность результатов расчетов с
экспериментом пока не представляется возможным.

\subsection{Задача о закачке газа под давлением}
Рассмотрим одномерную модельную задачу фильтрации. Пористость и
абсолютная проницаемость породы постоянны во всем объеме:  $m$=0.4,\; $K=6.64\cdot 10^{-11}$ м$^2$.
Начальные условия:\; $S_w=0.4$,\; $S_n=0.3$,\; $S_g=0.3$, 
$P_w=P_\text{атм}+\rho g h$($h$ - глубина, отсчитывается от верхнего края области);
$T=285K$. Граничные условия:

Сравним, как проходит процесс при двух различных граничных
условиях на температуру:
\begin{enumerate}
 \item \label{noT} $\left.T\right|_{x=0}=\left.T\right|_{x=1}=285K$;
 \item \label{T} $\left.T\right|_{x=0}=320K,\quad \Biggl.\dfrac{\partial{T}}{\partial{x}}\Biggr|_{x=1}=0$.
\end{enumerate}


Результаты расчетов представлены в виде зависимостей профилей насыщенностей фаз,
распределения давления и температуры от времени.

По оси $x$ отложено расстояние по вертикали от дна резервуара. По оси $y$--
значения насыщенностей каждой из фаз на расстоянии $x$ от дна резервуара
в определенный момент расчетного времени.

\begin{figure}
\begin{center}
\begin{minipage}[h]{0.6\textwidth}
\begin{tikzpicture}
  \begin{axis}[legend pos=outer north east, legend entries={$S_w$,$S_n$,$S_g$}, axis lines=left, enlargelimits=true, grid=major, width=1\textwidth, xlabel={$x$, м}, ylabel={$S$}]
    \pgfplotstableread[skip first n=3]{data/test1/F-000000001000.tec}{\mytable}
    \addplot [blue, ultra thick] table [x expr=1-\thisrow{0}, y=1] {\mytable};
    \addplot [red, ultra thick] table [x expr=1-\thisrow{0}, y=2] {\mytable};
    \addplot [black, ultra thick] table [x expr=1-\thisrow{0}, y=3] {\mytable};
  \end{axis}
\end{tikzpicture}
\caption{Распределение насыщенностей в момент времени $t=1000$с}
\end{minipage}
\vfill
\begin{minipage}[h]{0.49\textwidth}
\begin{tikzpicture}
  \begin{axis}[legend pos=outer north east, axis lines=left, enlargelimits=true, grid=major, width=1\textwidth, xlabel={$x$, м}, ylabel={$P$, Па}]
    \pgfplotstableread[skip first n=3]{data/test1/F-000000001000.tec}{\mytable}
    \addplot [blue, ultra thick] table [x expr=1-\thisrow{0}, y=4] {\mytable};
  \end{axis}
\end{tikzpicture}
\caption{Распределение давления $P_w$ в момент времени $t=1000$с}
\end{minipage}
\hfill
\begin{minipage}[h]{0.49\textwidth}
\begin{tikzpicture}
  \begin{axis}[legend pos=outer north east, axis lines=left, enlargelimits=true, grid=major, width=1\textwidth, xlabel={$x$, м}, ylabel={$T$, К}]
    \pgfplotstableread[skip first n=3]{data/test1/F-000000001000.tec}{\mytable}
    \addplot [blue, ultra thick] table [x expr=1-\thisrow{0}, y=5] {\mytable};
  \end{axis}
\end{tikzpicture}
\caption{Распределение температуры в момент времени $t=1000$с}
\end{minipage}
\end{center}
\end{figure}

\begin{figure}
\begin{center}
\begin{minipage}[h]{0.6\textwidth}
\begin{tikzpicture}
  \begin{axis}[legend pos=outer north east, legend entries={$S_w$,$S_n$,$S_g$}, axis lines=left, enlargelimits=true, grid=major, width=1\textwidth, xlabel={$x$, м}, ylabel={$S$}]
    \pgfplotstableread[skip first n=3]{data/test1/F-000000004000.tec}{\mytable}
    \addplot [blue, ultra thick] table [x expr=1-\thisrow{0}, y=1] {\mytable};
    \addplot [red, ultra thick] table [x expr=1-\thisrow{0}, y=2] {\mytable};
    \addplot [black, ultra thick] table [x expr=1-\thisrow{0}, y=3] {\mytable};
  \end{axis}
\end{tikzpicture}
\caption{Распределение насыщенностей в момент времени $t=4000$с}
\end{minipage}
\vfill
\begin{minipage}[h]{0.49\textwidth}
\begin{tikzpicture}
  \begin{axis}[legend pos=outer north east, axis lines=left, enlargelimits=true, grid=major, width=1\textwidth, xlabel={$x$, м}, ylabel={$P$, Па}]
    \pgfplotstableread[skip first n=3]{data/test1/F-000000004000.tec}{\mytable}
    \addplot [blue, ultra thick] table [x expr=1-\thisrow{0}, y=4] {\mytable};
  \end{axis}
\end{tikzpicture}
\caption{Распределение давления $P_w$ в момент времени $t=4000$с}
\end{minipage}
\hfill
\begin{minipage}[h]{0.49\textwidth}
\begin{tikzpicture}
  \begin{axis}[legend pos=outer north east, axis lines=left, enlargelimits=true, grid=major, width=1\textwidth, xlabel={$x$, м}, ylabel={$T$, К}]
    \pgfplotstableread[skip first n=3]{data/test1/F-000000004000.tec}{\mytable}
    \addplot [blue, ultra thick] table [x expr=1-\thisrow{0}, y=5] {\mytable};
  \end{axis}
\end{tikzpicture}
\caption{Распределение температуры в момент времени $t=4000$с}
\end{minipage}
\end{center}
\end{figure}

\subsection{Задача просачивания с источником на границе}
Расматриваем двумерную область пористой среды, имеющую форму
прямоугольника. Пористость и абсолютная проницаемость породы постоянны во всем
объеме. Параметры модели такие же, как
в первой тестовой задаче.
Во всей области задаем следующее 
начальное распределение: $S_w=0.4$,\; $S_n=0.35$,\; $S_g=0.25$, 
$P_w=P_\text{атм}+\rho g h$($h$ - глубина, отсчитывается от верхнего края области).
Пусть размер области -- $N_x\cdot N_y$.
Граничную поверхность, кроме части верхней границы, считаем непроницаемой, 
что может быть интерпретировано так,
что среда огорожена непроницаемым резервуаром, имеющим сверху отверстие с
координатами: $(N_x/3, N_y)$, $(2N_x/3, N_y)$. Через отверстие происходит обмен веществом между 
резервуаром и окружающей средой. На отверстии заданы условия: $P_w=P_\text{атм}$,

$ \dfrac{\partial S_w}{\partial t}= 
\begin{cases}
 q, \; S_w<S_{max}\\
 0, \; \text{иначе}
\end{cases}
$,
$ \dfrac{\partial S_n}{\partial t}=-q S_n$.
